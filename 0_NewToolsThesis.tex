%%%%%%%%%%%%%%%%%%%%%%%%%%%%%%%%%%%%%%%%%%%%
%Set Document Type
%%%%%%%%%%%%%%%%%%%%%%%%%%%%%%%%%%%%%%%%%%%%
\documentclass[11pt,a4paper]{report}  

%%%%%%%%%%%%%%%%%%%%%%%%%%%%%%%%%%%%%%%%%%%%
%Packages
%%%%%%%%%%%%%%%%%%%%%%%%%%%%%%%%%%%%%%%%%%%%
\usepackage{MyStyle, graphicx}
\usepackage{xspace}
\usepackage[svgnames]{xcolor}
\usepackage{longtable}
\usepackage{amsmath}     %!  setspace is used to control linepacing
\usepackage[square]{natbib} %! needed for Harvard style of references.
\usepackage{enumerate}  %! used for the library form, but you might find 
\usepackage[hidelinks]{hyperref}
\usepackage[top=2.54cm, left=4cm, right=2.54cm]{geometry}
\usepackage{multicol}
\usepackage{amssymb}
\usepackage{acro}
\usepackage{textgreek}
%%%%%%%%%%%%%%%%%%%%%%%%%%%%%%%%%%%%%%%%%%%%
%Set Tables
%%%%%%%%%%%%%%%%%%%%%%%%%%%%%%%%%%%%%%%%%%%%
\usepackage{lscape}
\usepackage{tablefootnote}
\usepackage{array}
\usepackage{bigfoot}
\usepackage{booktabs}
%%%%%%%%%%%%%%%%%%%%%%%%%%%%%%%%%%%%%%%%%%%%
%Set Font
%%%%%%%%%%%%%%%%%%%%%%%%%%%%%%%%%%%%%%%%%%%%
\usepackage[doublespacing]{setspace}
\usepackage[T1]{fontenc}
\usepackage{newpxtext,newpxmath}
%Set section headers to not be bold
\usepackage[titles]{tocloft}
\renewcommand{\cftsecfont}{\mdseries}
\renewcommand{\cftsecpagefont}{\mdseries}
%%%%%%%%%%%%%%%%%%%%%%%%%%%%%%%%%%%%%%%%%%%%
%Set Figure and Table Caption colour
%%%%%%%%%%%%%%%%%%%%%%%%%%%%%%%%%%%%%%%%%%%%
\usepackage{caption}
\captionsetup[table]{labelsep=period, labelfont={bf}, font={small, stretch=1.5}}     %% change 1.2 as you like
\captionsetup[figure]{labelsep=period, labelfont={bf}, font={small, stretch=1.5}}
\captionsetup{width=0.98\textwidth}
%%%%%%%%%%%%%%%%%%%%%%%%%%%%%%%%%%%%%%%%%%%%
%Set Code Block Formatting
%%%%%%%%%%%%%%%%%%%%%%%%%%%%%%%%%%%%%%%%%%%%
\usepackage{listings}
\usepackage{color}
%%%%%%%%%%%%%%%%%%%%%%%%%%%%%%%%%%%%%%%%%%%%
%Settings
%%%%%%%%%%%%%%%%%%%%%%%%%%%%%%%%%%%%%%%%%%%
%\thesisdraft                       %% Uncomment this if you want a draft
%Chapter Style
%%%%%%%%%%%%%%%%%%%%%%%%%%%%%%
\leftchapter
\usepackage{epigraph} %To add quotes
\usepackage{titlesec}
\definecolor{gray75}{gray}{0.75}
\newcommand{\hsp}{\hspace{10pt}}
\titleformat{\chapter}[hang]{\Huge}{\thechapter\hsp\textcolor{gray75}{|}\hsp}{0pt}{\Huge\bfseries}
\titleformat*{\section}{\Large\bfseries}
\titleformat*{\subsection}{\large\bfseries}
\titleformat*{\subsubsection}{\normalfont\bfseries}

%Paragraph Style
%%%%%%%%%%%%%%%%%%%%%%%%%%%%%%
\doublespacing
\usepackage{blindtext}
\usepackage{lastpage}
\usepackage{fancyhdr}

%Header and Footer Style
%%%%%%%%%%%%%%%%%%%%%%%%%%%%%%
\pagestyle{fancy}
\fancyhf{}
\fancyhead[C]{}
\fancyfoot[R]{\thepage \hspace{1pt} }
\renewcommand{\headrulewidth}{0pt}

%Prevent footnotes from breaking over multiple pages. 
\interfootnotelinepenalty=10000
%%%%%%%%%%%%%%%%%%%%%%%%%%%%%%%%%%%%%%%%                             
%Acronyms
%%%%%%%%%%%%%%%%%%%%%%%%%%%%%%%%%%%%%%%%
                            
%Organism Names
%%%%%%%%%%%%%%%%%%%%%%%%%%%%%%%%%%%%%%%%
\DeclareAcronym{Focub}{
  short = \textit{Foc},
  long  = \textit{Fusarium oxysporum} f. sp. \textit{cubense}}

\DeclareAcronym{tr4}{
  short = TR4,
  long  = Tropical Race 4,
}

\DeclareAcronym{Focub4}{
  short = \textit{Foc} TR4,
  long  = \textit{Fusarium oxysporum} f. sp. \textit{cubense} Tropical Race 4}


\DeclareAcronym{str4}{
  short = STR4,
  long  = Sub-Tropical Race 4,
}

\DeclareAcronym{r1}{
  short = R1,
  long  = Race 1,
}

\DeclareAcronym{Focub1}{
  short = \textit{Foc} R1,
  long  = \textit{Fusarium oxysporum} f. sp. \textit{cubense} Race 1,
}


\DeclareAcronym{r2}{
  short = R2,
  long  = Race 2,
}  
  
\DeclareAcronym{Fo}{
  short = \textit{F. oxysporum},
  long  = \textit{Fusarium oxysporum},
}
  
\DeclareAcronym{Fol}{
  short = \textit{Fol},
  long  = \textit{Fusarium oxysporum} f. sp. \textit{lycopersici},
}

\DeclareAcronym{fsp}{
  short = f. spp.,
  long  = \textit{Formae speciales},
}

\DeclareAcronym{FFSC}{
  short = FFSC,
  long  = \textit{Fusarium fujikuroi} species complex,
}  

\DeclareAcronym{FOSC}{
  short = FOSC,
  long  = \textit{Fusarium oxysporum} species complex,
} 

\DeclareAcronym{FGSC}{
  short = FGSC,
  long  = \textit{Fusarium graminearum} species complex,
} 

\DeclareAcronym{vcg}{
  short = VCG,
  long  = Vegetative Compatibility Group,
} 

\DeclareAcronym{xvm}{
  short = \textit{Xvm},
  long  = \textit{Xanthomonas   campestris }pv. \textit{musacearum},
} 

%Genes, Proteins etc
%%%%%%%%%%%%%%%%%%%%%%%%%%%%%%%%%%%%%%%%
\DeclareAcronym{tef}{
  short = \textit{Tef-1}\textalpha,
  long  = \textit{Translation Elongation Factor-1} \textalpha ,
}

\DeclareAcronym{rbp2}{
  short = \textit{RBPII},
  long  = \textit{RNA polymerase   subunit 2},
}

\DeclareAcronym{mites}{
  short = MITEs,
  long  = Miniature inverted-repeat transposable elements,
}

\DeclareAcronym{te}{
  short = TE,
  long  = transposable element,
}

\DeclareAcronym{mimp}{
  short = \textit{mimp},
  long  = \textit{\underline{m}iniature \underline{imp}ala},
}

\DeclareAcronym{sixg}{
  short = \textit{SIX} gene,
  long  = \textit{Secreted In Xylem  } gene,
}

\DeclareAcronym{sixp}{
  short = SIX protein,
  long  = Secreted In Xylem protein,
}

\DeclareAcronym{vic}{
  short = \textit{Vic},
  long  = \textit{Vegetative Incompatibility}}

\DeclareAcronym{rga2}{
  short = \textit{RGA2},
  long  = \textit{Resistance Gene Analog 2}}



%Biologial Processes & General Terms
%%%%%%%%%%%%%%%%%%%%%%%%%%%%%%%%%%%%%%%%
\DeclareAcronym{AVR}{
short = AVR,
long = avirulence,
}

\DeclareAcronym{ac}{
short = AC,
long = accessory chromosome,
}

\DeclareAcronym{cc}{
short = CC,
long = core chromosome,
}

\DeclareAcronym{bp}{
short = bp,
long = base pair,
}

\DeclareAcronym{tir}{
short = TIR,
long = terminal invert repeat,
}

\DeclareAcronym{prr}{
short = PRRs,
long = pattern recognition receptors,
}

\DeclareAcronym{nlr}{
short = NLR,
long = nucleotide-binding leucine-rich receptor,
}

\DeclareAcronym{pamp}{
short = PAMPs,
long = pathogen-associated molecular patterns,
}

\DeclareAcronym{pti}{
short = PTI,
long = PAMP-triggered immunity,
}


\DeclareAcronym{pcr}{
short = PCR,
long = polymerase chain reaction,
}
  
\DeclareAcronym{eti}{
short = ETI,
long = effector-triggered immunity,
}

\DeclareAcronym{ets}{
short = ETS,
long = effector-triggered susceptibility,
}

\DeclareAcronym{rprot}{
short = R proteins,
long = resistance proteins,
}

\DeclareAcronym{rgene}{
short = R gene,
long = resistance gene,
}

                         
%Bioinf 
%%%%%%%%%%%%%%%%%%%%%%%%%%%%%%%%%%%%%%%%
\DeclareAcronym{blast}{
short = BLAST,
long = Basic Local Alignment Tool,
}

\DeclareAcronym{busco}{
short = BUSCO,
long = Benchmarking Universal Single-Copy Orthologs,
}

\DeclareAcronym{maei}{
short = Maei,
long = \textit{Mimp}-associated effector identification,
}

\DeclareAcronym{ml}{
short = ML,
long = Machine learning,
}

\DeclareAcronym{phibase}{
short = PHI-base,
long = Pathogen-Host Interaction Database,
}               

%Stats
%%%%%%%%%%%%%%%%%%%%%%%%%%%%%%%%%%%%%%%%
\DeclareAcronym{ANOVA}{
short = ANOVA,
long = Analysis of variance,
}

%Experimental Terms and Techniques
%%%%%%%%%%%%%%%%%%%%%%%%%%%%%%%%%%%%%%%%

\DeclareAcronym{dpi}{
short = dpi,
long = Days post-inoculation,
}

\DeclareAcronym{wpi}{
short = wpi,
long = Weeks post-inoculation,
}

\DeclareAcronym{hplc}{
short = HPLC,
long = High-performance liquid chromatography,
}

\DeclareAcronym{lcms}{
short = LCMS,
long = Liquid chromatography-Mass spectrometry,
}

\DeclareAcronym{lamp}{
short = LAMP,
long = Loop-mediated isothermal amplification
}

\DeclareAcronym{pda}{
short = PDA,
long = Potato dextrose agar,
}

\DeclareAcronym{pdb}{
short = PDB,
long = Potato dextrose broth,
}

\DeclareAcronym{rpm}{
short = rpm,
long = Revolutions per minute
}

\DeclareAcronym{um}{
short = UM,
long = untargeted metabolomics,
}

%Organisations
%%%%%%%%%%%%%%%%%%%%%%%%%%%%%%%%%%%%%%%%
\DeclareAcronym{FAO}{
short = FAO,
long = Food and Agriculture Organisation of the United Nations}

\DeclareAcronym{wbf}{
short = WBF,
long = World Banana Forum}

\DeclareAcronym{ncbi}{
short = NCBI,
long = National Center of Biotechnology Information, }

\DeclareAcronym{tnau}{
short = TNAU,
long = Tamil Nadu Agricultural University }

%Misc
%%%%%%%%%%%%%%%%%%%%%%%%%%%%%%%%%%%%%%%%

\DeclareAcronym{rs}{
short = RS,
long = remote sensing }

\DeclareAcronym{uav}{
short = UAV,
long = Unmanned aerial vehicle}

\DeclareAcronym{vi}{
short = VI,
long = Vegetation indices}

\DeclareAcronym{ps2}{
short = PSII,
long = Photosystem II}



\newcommand{\et}{\textit{et al., }}
%%%%%%%%%%%%%%%%%%%%%%%%%%%%%%%%%%%%%%%%                             
%Start Document
%%%%%%%%%%%%%%%%%%%%%%%%%%%%%%%%%%%%%%%%
\begin{document}

%%%%%%%%%%%%%%%%%%%%%%%%%%%%%%%%%%%%%%%%
%Title Pages
%%%%%%%%%%%%%%%%%%%%%%%%%%%%%%%%%%%%%%%%
\begin{titlepage}
    \begin{center}
        \singlespacing
        \Huge
        \textbf{New Tools for the Identification of} \\\vspace*{0.5cm} \textbf{\textit{Fusarium} Wilt in Banana} \\
        \vspace*{1.75cm}
        \normalsize
        by \\ 
        \vspace*{0.2cm}
        \LARGE
        Jamie Pike BSc (Hons) \\ 
        \vfill
        \normalsize
        Submitted in the partial fulfilment of the requirements for the degree of \\
        \Large
        \vspace*{0.3cm}
        Doctor of Philosophy in Life Sciences \\
        \vspace*{1cm}
        \includegraphics[width=0.4\textwidth]{Preamble/crest_black.eps} \\
        \vspace*{1cm}
        \large
        University of Warwick, School of Life Sciences\\
        \vspace*{0.5cm}
        \normalsize
        March 2024 \\
         \vspace*{1cm}   
    \end{center}
    \clearpage
\end{titlepage}

%%%%%%%%%%%%%%%%%%%%%%%%%%%%%%%%%%%%%%%%
%Fromatting
%%%%%%%%%%%%%%%%%%%%%%%%%%%%%%%%%%%%%%%%
\pagenumbering{roman} %! Begins roman numerals start from page i.
\clearpage
\tableofcontents                     
\listoftables                     
\listoffigures                    
%%%%%%%%%%%%%%%%%%%%%%%%%%%%%%%%%%%%%%%%
%Prelimary Sections
%%%%%%%%%%%%%%%%%%%%%%%%%%%%%%%%%%%%%%%%
\begin{thesisacknowledgments}        
\input Preamble/0.1_Acknowledgments.tex       
\end{thesisacknowledgments}

\begin{thesisdeclaration}       
\input Preamble/0.2_Declarations.tex 
\end{thesisdeclaration}

\begin{thesisabstract}              
  \begin{singlespace}    
    \input Preamble/0.3_Abstract.tex      
 \end{singlespace}
\end{thesisabstract}

%\begin{thesisabbreviations}      
\printacronyms[name=Abbreviations]
%\end{thesisabbreviations}

%%%%%%%%%%%%%%%%%%%%%%%%%%%%%%%%%%%%%%%%%%%%
%Main Chapters
%%%%%%%%%%%%%%%%%%%%%%%%%%%%%%%%%%%%%%%%
\pagenumbering{arabic} %! Begins arabic numerals start from page 1.
\renewcommand{\arraystretch}{1.5} %Extend tables so they're easier to read
%Set all instances of text below (e.g. \focub) to italicised full version.

%Set chapter names and .tex files with chapter content.
\setlength\epigraphwidth{.7\textwidth}
\setlength{\epigraphrule}{0pt}

    \vspace*{1cm}
    \epigraph{\protect\singlespace\textit{"The more clearly we can focus our attention on the wonders and realities of the universe about us, the less taste we shall have for destruction."}}{\textemdash \ Rachel Carson, \textit{Silent Spring}}
    
\chapter{Introduction}
    \input Chapters/1_Introduction.tex
    
\chapter{Potential Novel \textit{Fusarium} Pathogen of Banana Identified }
    \input Chapters/2_PotentialNovelFusarium.tex
    
\chapter{Improving Genomic Tools for Identifying Virulence Factors in   \textit{Fusarium oxysporum formae specialis}}
    \input Chapters/3_ImprovingGenomicTools.tex
    
\chapter{Phenomics: Image-based Analysis of Banana Stresses}
    \input Chapters/4_Phenomics.tex
    
 \chapter{Metabolomics: The Banana-pathogen Metabolome}
    \input Chapters/5_Metabolomics.tex


%%%%%%%%%%%%%%%%%%%%%%%%%%%%%%%%%%%%%%%%%%%%
%Appendices
%%%%%%%%%%%%%%%%%%%%%%%%%%%%%%%%%%%%%%%%%%%%
\appendix                            %% this will do the appendices
\chapter{Scripts used for \textit{Fusarium} Genome Assembly and Analysis}
For all of the scripts written as part of the genome assembly and analysis of the TNAU \textit{Fusarium} isolates, SY-2, S6, S16, and S32. 
%%%%%%%%%%%%%%%%%%%%%%%%%%%%%%%%%%%%%%%%%%%%%%%%
%CODE OUTPUT STYLE
%%%%%%%%%%%%%%%%%%%%%%%%%%%%%%%%%%%%%%%%%%%%%%%%

\lstset{basicstyle=\ttfamily,
  showstringspaces=false,
  commentstyle=\color{red},
  keywordstyle=\color{blue}
}

\lstdefinestyle{customc}{
  belowcaptionskip=1\baselineskip,
  breaklines=true,
  showstringspaces=false,
  basicstyle=\footnotesize\ttfamily,
  keywordstyle=\bfseries\color{green!40!black},
  commentstyle=\itshape\color{purple!40!black},
  identifierstyle=\color{blue},
  stringstyle=\color{orange},
}

%Stops the script running off the page
\lstset{escapechar=@,style=customc}

%%%%%%%%%%%%%%%%%%%%%%%%%%%%%%%%%%%%%%%%%%%%%%%%
%CODE SECTIONS
%%%%%%%%%%%%%%%%%%%%%%%%%%%%%%%%%%%%%%%%%%%%%%%%

\section{GCTrimmer.py}\label{apx:gcTrimmer.py}
Script used to separate contigs in the TNAU assemblies into separate FASTA files based on GC\% threshold. 

\begin{lstlisting}[language=Python, caption=Script to separate contigs based on GC\% threshold]
#Extracts sequences from a FASTA file and order in to two different FASTAs depending on GC content

###############################################
#Set up and import modules.
import re, sys, colorama
from datetime import date
from tqdm import tqdm
from colorama import Fore
from Bio import SeqIO, SeqUtils
from Bio.SeqUtils import GC
from Bio.SeqRecord import SeqRecord
###############################################
#Establish input file (FASTA) and gc Threshold.
infile = sys.argv[1]
gcContentThreshold = sys.argv[2]
###############################################
#Establish colour reset.
colorama.init(autoreset=True)
###############################################
# Establish Count for sequences above and below threshold.
SeqAboveCount = 0
SeqBelowCount = 0

###############################################
#Progress:
#Record date
today = date.today()
startTime = today.strftime("%B %d, %Y")
#Create progress bar
num = len([1 for line in open(infile) if line.startswith(">")])
print("="*80)
print(f'Running gcTrimmer.py.\n{startTime}.\n\nInput File:\t\t{infile}\nNumber of Contigs:\t{num}\nGC threshold:\t\t{gcContentThreshold}%\n')

#Start progress bar
with tqdm(total=num) as pbar:
###############################################
    #Open two .fasta files using the name of the infile.
    with open(f'{infile}_GCcontentBelow{gcContentThreshold}perc.fasta'.format(), "a") as LessThanFile, open(f'{infile}_GCcontentAbove{gcContentThreshold}perc.fasta'.format(), "a") as GreaterThanFile, open(f'gcContentReport.txt'.format(), "a") as report  :

    #Parse and loop through the FASTA input file. Search each scaffold/contig for sequences with length => than the desired GC content and deposit them in the FASTA
    #containing the less than the desired gc content. If the sequence gc content is greater than the gc content threshold, then it is added to the "greater than" FASTA file.
        for sequence in SeqIO.parse(infile, "fasta"):
            print("="*10, file=report, sep="\n")
            gccontent = (sequence.seq.count('G') + sequence.seq.count('C')) / len(sequence) * 100
            pbar.update(1)
            if gccontent < int(gcContentThreshold):
                print(f'{sequence.description} is below the threshold GC content. GC content is {Fore.RED}{gccontent}{Fore.BLACK}.\n', file=report, sep="\n")
                print(sequence.format("fasta"), file=LessThanFile, sep="\n")
                SeqBelowCount = SeqBelowCount + 1
            else:
                print(f'{sequence.description} is above the threshold GC content. GC content is {Fore.GREEN}{gccontent}{Fore.BLACK}.\n', file=report, sep="\n")
                print(sequence.format("fasta"), file=GreaterThanFile, sep="\n")
                SeqAboveCount = SeqAboveCount + 1


#Print the total number of sequences above and below the GC threshold.

print(f'\nThe number of sequences above the GC threshold: {Fore.GREEN}{SeqAboveCount}{Fore.BLACK}.\nThe number of sequences below the GC threshold: {Fore.RED}{SeqBelowCount}{Fore.BLACK}.')
print("="*80)    
\end{lstlisting}

\section{ContaminatFilter.sh}\label{apx:ContamFilter}
\begin{lstlisting}[language=Python, caption=bash script to filter contigs from assembly that are predicted to belong to a non-\textit{Fusarium} genus by Blobtools.]
#!/bin/bash

#Jamie Pike
#Command for filtering conatminant blobtools hits. This groups all of the standard BlobTools commands I used to remove a contaminat contigs analysis. 

python -c "print('=' * 75)"
echo "Blobtools Contaminant Filter"
echo "----------------------------"
echo $(date)
echo "Usage: ContaminantFilter.sh <FASTA file> <blobtools json file> <outfile prefix>"
python -c "print('=' * 75)"

infile=${1?Please provide the assembly input fasta.} #Input Assembly.
json=${2?Please provide a BlobTools json file.} #Input BAM.
prefix=${3?Please provide a prefix for you outputs.} 

########################
#Escape the script if there are any errors. 
set -e 

echo "Creating species table..."
#Use BlobTools view to generate a blobtools table.txt file for filtering.
blobtools view -i ${json} -r species -o ${prefix}

echo "Filetering for Fusarium and no-hit only contigs..."
#Use grep to extact only the Fusarium and no-hit lines from the hit table.
grep -E 'Fusarium|no-hit' ${prefix}*.table.txt > ${prefix}.FusariumHits.table.txt

echo "Generating list file..."
#Create a list of the Fusarium and no-hit only contigs. 
awk '{print $1}' ${prefix}.FusariumHits.table.txt >  ${prefix}.FusariumHits.list.txt

echo "Generating contaminant filtered fasta..."
#Create a list of the Fusarium and no-hit only contigs. 
blobtools seqfilter -i ${infile} -l ${prefix}.FusariumHits.list.txt -o ${prefix}.ContaminantFiltered

echo "Generating contaminant sequences fasta..."
#Create a list of the Fusarium and no-hit only contigs. 
blobtools seqfilter -v -i ${infile} -l ${prefix}.FusariumHits.list.txt -o ${prefix}.ContaminantSequences

echo "ContaminantFilter.sh finished."
echo $(date)
echo "+++++
It is advisable that you check the number of contigs in the ${prefix}.ContaminantFiltered.fasta matches the number of lines in the ${prefix}.FusariumHits.table.txt file."
python -c "print('=' * 75)" 

\end{lstlisting}



\chapter{Scripts used as part of Maei Pipeline and Subsequent Analysis}
For all of the scripts written for the \textit{mimp}-associated effector identification pipeline, as well as scripts used for the additional analysis of the \textit{Fusarium} database used in the identification of effector candidates. 
\input{Appendices/Chap3-CodeSections}

%\bibliographystyle{harvard}
%\bibliography{Bibliogrphy}            %% Start your bibliography here;

                                 %! with sample.bib as your bibliography file. You can
                               %% also use:
                %\begin{thebibliography}
                 %\bibitem{Bibliography/NewTools Bib.bib}
                %\end{thebibliography}
                               %% to generate your bibliography.

%\begin{thesisauthorvita}             %% Write your vita here; it can be
%                                     %% anything in LaTeX2e par-mode.
%\end{thesisauthorvita}               %%




\end{document}                       %% Done.

