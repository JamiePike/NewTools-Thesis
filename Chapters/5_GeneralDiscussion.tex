
\section{Introduction}

Banana is a staple food and cash crop, with a historical significance spanning thousands of years (see section \ref{sec:cultivation}). Like many important agricultural commodities, global banana production is threatened by a host of diseases, with \Acf{fwb} a particular concern \parencite{Ploetz2005}. Recognising the potential impact of \ac{fwb}, this thesis had three main objectives:

\begin{enumerate}
    \item Build tools which aid in the identification of targets for molecular diagnostics and contribute to our understanding of \ac{Focub}-banana interactions as well as the evolutionary history, diversity, and classification of \ac{Focub}. 
    \item Develop molecular diagnostics which can be used to identify specific races of \acl{Fo}. 
    \item Develop metabolic approaches to improve understanding of banana interactions with \ac{Focub} and aid in the development of diagnostics.
\end{enumerate}

\section{Diversity of candidate effectors in \acl{Fo}}

The first two objectives were addressed in Chapter \ref{Chap3}, which introduced our \acf{maei} pipeline. Through the \ac{maei} pipeline we have identified candidate effectors in \ac{Focub}. COVID-19-related restrictions impacted sampling in India and subsequently political issues associated with funding contributions to our lab's collaboration with \acf{tnau} prevented further \ac{Focub} specific work. However, through new collaborations with researchers at the University of Warwick and \ac{niab}, we were able to expand our focus. We trialled the \ac{maei} pipeline on other \ac{Fo} genomes and investigated the distribution of \acfp{cec} among multiple \acf{Fo} \acfp{fsp}.  

\ac{Fo} is a particularly diverse species containing at least 106 clearly described \acp{fsp} \parencite{Edel-Hermann2019}. As \acp{fsp} are classified by host-specificity, \textcite{Dam2016,Dam2017,FoEC2} have reported that isolates from the same host species have a shared repertoire of host-specific effectors. In contrast, isolates with different host species possess distinct effector profiles. Not only does chapter \ref{Chap3} of this thesis support the findings of those three studies, it expands upon them, providing new insights - particularly variation within race (see section \ref{FolaMaei}). Moreover, the \ac{maei} pipeline overcomes the issues we identified in the \ac{mimp}-associated effector identification pipeline developed by \textcite{Dam2016} and attempts to mitigate concerns in computational effector prediction presented by \textcite{Sperschneider2015, LoPresti2015, Todd2022} (see section: \ref{sec:Chap3Intro}). 

As \ac{Fo} continues to spread and new \ac{fsp} and races emerge \parencite{Edel-Hermann2019, Henry2020, Mestdagh2023}, the ability to quickly discern unique diagnostic targets associated with virulence will become essential. The \ac{maei} pipeline can help to address this. Though we have so far been unable to develop race-specific diagnostics for \ac{Focub}, we have begun to develop race-specific diagnostics for \acf{Foa} using several \acp{ce} identified based on their presence in race-specific strains with the \ac{maei} pipeline. These include CEC\_198 (for \ac{Foa} R2) and CEC\_76 (for \ac{Foa} R4). Further, our \acfp{ce} must be experimentally validated, for instance, through characterisation of the gene by deletion, editing, or complementation to confirm function. Characterising their role in virulence is crucial for the development of markers that adhere to the recommendations of \textcite{Magdama2019}, that primer design should focus on a genetic locus related to virulence to limit false positives. 

While the \ac{maei} pipeline holds promise for uncovering \acp{ce} in \ac{Fo}, its reliance on \acp{mimp} as a key \ac{ce} identifying element presents both constraints and advantages. The association between \acp{mimp} and \acp{ce} can enhance confidence in \acp{ce} identified in \ac{Fo} genomes. However, \textit{Fusarium} genomes such as \acl{Fs} and \acl{Fm} contain few \acp{mimp}, resulting in few to no \ac{mimp}-associated \acp{ce}. Additionally, if there are shared effectors between different \textit{Fusarium} species that are not associated with \acp{mimp} they may be missed. This raises the question: to what extent is there an association between \acp{mimp} and \acp{ce} in \textit{Fusarium}, and specifically \ac{Fo}? 

A multi-omic approach may help to combat some of the concerns attributed to the discovery of \acp{ce} in \textit{Fusarium}. Indeed, we employed transcriptomics to confirm the expression of the \acp{ce} identified in \ac{Fola} (see section \ref{sec:Chap3RNASeq}). Separate \acp{ce} were identified using the transcriptomic data that did not appear in the set of \acp{ce} identified using the \ac{maei} pipeline. However, transcriptomic data were collected at one time-point (96\ac{hpi}), timed to coincide with the delivery of \textit{SIX8, SIX9}, and \textit{SIX14} determined by qPCR (data not shown). This approach may have ruled out some genuine \acp{ce} expressed at earlier or later stages of infection that were identified using the  \ac{maei} pipeline \parencite{Toruno2016}. 

Few time-course studies have been conducted on the delivery of \ac{mimp}-associated effectors in \ac{Fo}, and peak expression times may vary. As the costs of DNA and RNA sequencing technology decrease over time \parencite{Chan2005, Hu2021}, opportunities for fungal effector discovery expand. The \ac{maei} pipeline, for instance, could be complemented by transcriptomic analysis conducted over time. Not only will this provide a greater understanding of the temporal delivery of effectors, but also can confirm the expression of candidates.

Should a relationship between effector characteristics and temporal delivery be established, the data collected could enhance the predictive capabilities of fungal effector \ac{ml} prediction tools like EffectorP \parencite{Sperschneider2022}. Indeed, as \textcite{Sperschneider2020} recognise, \ac{ml} tools present an exciting opportunity to study plant-pathogen interactions. A \ac{ml} tool could be developed to annotate \ac{mimp}-associated \acp{ce} without additional identification and filtering steps (e.g. filtering based on strict size threshold), perhaps better adhering to the recommendations of \textcite{Sperschneider2015, LoPresti2015, Todd2022}. Further, fungal effectors could be classified based on predicted structure rather than homology (see \textcite{Seong2023}) as part of the \ac{maei} pipeline. However, it is important to note that \ac{ml} tools require large training datasets, which are not currently available solely for \ac{mimp}-associated \acp{ce} in \ac{Fo}, and we must recognise the limitations associated with \ac{ml} tools. For instance, AlphaFold 2 \parencite{Jumper2021}, developed to predict protein structure, has become a relatively well-known \ac{ml} tool, with recent reviews discussing the "significant impact" \parencite{Yang2023} AlphaFold 2 could have on biology \parencite{Skolnick2021, Marcu2022, Varadi2023}\footnote{This set of citations is not exhaustive.}. However, AlphaFold 2 is not infallible, with \textcite{Bertoline2023} discussing some of its limitations, including difficulty predicting intrinsically disordered proteins/regions and a tendency to over-predict secondary structures in loop regions. As \textcite{Sperschneider2020} stress, the use of \ac{ml} in the field of plant–pathogen interactions research must be accompanied by an awareness of both ML theory and the biological application to achieve robust classifiers that empower knowledge discovery and are of practical use. 

% \Acf{ml} tools are helping to advance the study of fungal effectors. As well as SignalP v5.0b \parencite{Petersen2011}, and EffectorP 2.0 \parencite{Sperschneider2018},  AlphaFold 2 has been used to investigate the structural similarity and diversity of fungal effectors \parencite{Seong2023}. As \textcite{Sarsa2022} recognise, large language models like Chat-GPT \parencite{OpenAI2024} can aid researchers in writing code and using bioinformatic tools.  

Overall, the \ac{maei} pipeline can contribute to our understanding of virulence in \ac{Fo}. It can be used as a starting point for investigating the relationship between \ac{Fo} \acp{fsp} and \ac{mimp}-associated \acp{ce}. However, \acp{ce} still require validation. Beyond this, the \ac{maei} pipeline can be integrated with other -omic approaches, such as transcriptomics, to increase confidence in \acp{ce} and improve our understanding of \acp{ce} in \ac{Fo}. 

\section{Novel approaches for the diagnosis of \acl{fwb}}

To address the third objective of this thesis - develop novel -omics approaches to improve understanding of banana interactions with \ac{Focub} and aid in the development of diagnostics - we applied the first comparative \acf{um} analysis of banana wilting stresses (Chapter \ref{Chap4}). Features of interest that appeared to distinguish three different wilting stresses (\acf{fwb}, \acf{xwb}, and drought stress) from mock-inoculated plants over time were identified. These unique features of interest can be explored as putative biomarkers, and their role in wilt stress response investigated. Further, we propose sampling infected vasculature in any future \ac{um} studies of banana wilting stresses to aid in the identification of a greater number of unique features at higher peak intensities. 

\textcite{Aina2024} provide a detailed review of plant biomarkers as early detection tools in stress management in food crops. However, current diagnosis of \ac{fwb} is primarily centred on molecular diagnostics (see section \ref{sec:chap1-diagnostics}), with some imaging diagnostics developed \parencite{Ye2020a, Ye2020b, Selvaraj2019b, Zhang2022}. While a combined assay to diagnose specific wilting stresses based on unique biomarkers would be beneficial, questions about accessibility, cost-effectiveness, and practicality remain. The technology for molecular marker development is well established, and molecular markers are invaluable for rapid diagnosis in scenarios like import centres (e.g., airports, and regional borders). In contrast, the biomarkers presented in this thesis require further research to establish robust and reliable assays.

Image-based technologies have been developed to distinguish various banana stresses using smartphones \parencite{Aasha2024}. For instance, \textcite{Selvaraj2019b} presents the Tumaini application, which can identify multiple biotic and abiotic stresses in images of symptomatic banana foliage. Given the increasing accessibility of smartphones and applications, relying solely on biomarkers for in-field diagnostics may be less practical. Instead, integrating imaging diagnostics with biomarkers could enhance accuracy: imaging technologies could detect potential stresses, while biomarkers could validate the nature of those stresses. Molecular markers would then be employed when precise race-level pathogen classification is crucial for determining control measures, benefiting agricultural extension services and researchers engaged in screening operations.

Biomarkers can also offer unique insights into responses to abiotic stresses—an area where molecular markers may not provide adequate information. As climate change intensifies exposure to abiotic stresses — such as drought, heat, and salinity — biomarkers can provide critical insights into plant responses, enabling proactive management and adaptation strategies for crop resilience. \textcite{Sambles2017, Sidda2020} recognised that secoiridoid glycosides could be used as a marker of susceptibility or tolerance to ash dieback (caused by \textit{Hymenoscyphus fraxineus}) in ash (\textit{Fraxinus excelsior}), and that these markers could play a useful role in breeding. As \textcite{Nansamba2020, Kumar2020} highlight, banana breeding can be particularly time-consuming. Building upon our foundational research, a \ac{srm} approach may be applied to aid in resistance/tolerance screening in precision banana breeding programmes, identifying markers for \ac{fwb}, \ac{xwb}, and drought resistance/tolerance/susceptibility in new lines quickly. Using \ac{srm}, researchers can employ a targeted mass spectrometry method, allowing for the precise measurement of hundreds of peptides in a single analysis, providing an opportunity for multiplexed biomarker measurements. Indeed, \textcite{Chawade2016} developed a panel of predictive \ac{srm} markers for \textit{Phytophthora infestans} resistance in three potato (\textit{Solanum tuberosum}) tissues (leaves, tubers, and potato leaf secretome). The authors suggest that their marker panel has the predictive potential for three traits, two of which had no commercial DNA markers. \Ac{srm} may also be applied in the banana relative, \textit{Ensete ventricosum}, an orphan crop that can withstand adverse climatic conditions \parencite{Feyisa2022}.

To develop robust diagnostic assays for banana breeding, our putative biomarker candidates, and others, must be validated in real-world field conditions. Moving forward, the insights gained from our comparative \ac{um} analysis of banana wilting stresses open up exciting avenues for further research and practical applications. Additionally, exploring the mechanisms underlying the stress response, and the role of these biomarkers in plant-pathogen interactions, could lead to the development of targeted strategies for disease control and crop improvement.

\section{\Acl{fwb} is potentially caused by a suite of Fusarium pathogens}

Finally, one of the most unexpected, and consequently concerning, findings discussed in this thesis was the diversity of \textit{Fusarium} species capable of causing \ac{fwb} in the Indian sampling region (Chapter \ref{Chap2}). All of the four \textit{Fusarium} isolates collected (S6, S16, S32, and SY-2) in the southern Indian state of Tamil Nadu were reported to be highly virulent on the Cavendish variety 'Grande Naine' (AAA). Our analysis suggests that these four isolates belong to three separate species, \acf{Focub} (S6), \acf{Fs} (S16 and SY-2), and the newly described species, \acf{Fm} (S32) \parencite{Nozawa2023}. This classification was supported by alignment of raw read data (see section \ref{sec:chap2RawReadMapping}), phylogenetic analysis of common \textit{Fusarium} barcodes (\ac{tef} and \ac{rbp2}) (see section \ref{sec:chap2phylogenies}), and profiling of known \ac{Fo} virulence genes (see section \ref{sec:chap2SixGene}). The findings in Chapter \ref{Chap2} were further supported by two recent discoveries of \textit{Fusarium} infecting banana across the globe. Since starting this thesis, a single strain of \ac{Fs} able to infect Cavendish banana has been reported in Guangdong, China \parencite{Cui2021}, while \ac{Fm} sp. nov. infecting Cavendish has been reported solely in Mindanao, the Philippines \parencite{Nozawa2023}. However, it's essential to note that we could not directly access isolates S6, S16, S32, and SY-2 due to Nagoya challenges, thus we have not independently verified their ability to infect Cavendish varieties.

The findings in Chapter \ref{Chap2} suggest that these pathogens have a wider range than initially thought and raise pertinent questions regarding pathogen diversity, evolution, spread, and the future of \ac{fwb} control. First, the virulence of S6, S16, S32, and SY-2 should be further independently confirmed. Next, comprehensive epidemiological studies should be conducted to determine the range of these pathogens across different banana-growing regions, assessing the prevalence and distribution of these \ac{fwb} causal agents. This is essential given the location of first reports for \ac{Fs} pathogenic towards banana (Guangdong, China \parencite{Cui2021}) and \ac{Fm} (Mindanao, the Philippines \parencite{Nozawa2023}). Simultaneously, effective controls (chemical and biological), biosecurity measures, and diagnostic assays must be developed, varietal resistance screening should take place, and awareness programs should be established for banana growers. 

It is important to recognise the continuous evolution and expansion of \textit{Fusarium} as a threat to banana cultivation. Aside from \ac{Focub}, \ac{Fs}, and \ac{Fm}, \textcite{Jones1997, Du2017} report 7 other \textit{Fusarium} species that are linked to distinct banana diseases. Given our findings, the reports of \textcite{Cui2021, Nozawa2023} and the analyses of \textcite{Maryani2019}, we must ask if south-eastern Asia is a "hotspot" for the evolution of \ac{fwb} causal agents. Ongoing surveillance is essential to monitor and identify any emerging \ac{fwb} threats, particularly in India - the world's largest banana producer ($\ge30$ million tonnes in 2021).

The significance of compartmentalised genomes in the \textit{Fusarium} genus has garnered increased attention in scientific literature \parencite{Ma2010, Ma2013, Sperschneider2015b, Frantzeskakis2019, Hoh2022, Kamble2024}. Given the dynamic nature of pathogenicity-associated genes within the \textit{Fusarium} genus, facilitated by compartmentalised genomes, it would be interesting to ascertain what common genomic features are shared among \ac{Focub}, \ac{Fm}, and pathogenic \ac{Fs}. Notably, do putative \ac{Fm} (S32) and \ac{Fs} (S16 and SY-2) isolates possess compartmentalised genomes and if so, what is the degree of commonality across these compartmentalised genomes? \textcite{Cui2021} reported that their \ac{Fs} isolate (FS66), pathogenic towards Cavendish banana, harbours widespread gene transfer on the core chromosomes putatively from the \acl{FOSC}, including 30 genes involved in Fusarium pathogenicity/virulence. However, when the authors aligned their FS66 genome assembly to the four lineage-specific accessory chromosomes \ac{Foly} 4287 reference \parencite{Ma2010}, they found these regions were almost completely absent in FS66 (99.7\%). Future investigations should also compare SY-2 and S16 to the \ac{Fs} isolate pathogenic towards banana sequenced by \textcite{Cui2021} (FS66). Are these isolates (SY-2, S16, FS66) from the same \ac{Fs} strain?

The assemblies we generated for the S6, S16, S32, and SY-2,  isolates collected in Tamil Nadu were too fragmented for thorough synteny analysis or the identification of shared virulence regions due to sub-standard genomic DNA isolation. To avoid the fragmentation of difficult-to-assemble (e.g. \acp{ac}) regions in the \ac{tnau} genomes, long-read sequencing and assembly approaches should be applied. Researchers may also wish to combine transcriptomic profiling and proteomic studies to aid in the discovery of virulence factors and improve confidence in assembly annotations. Future studies may then reveal essential features contributing to virulence towards Cavendish banana in \textit{Fusarium}. 

It is also pertinent to consider the putative \ac{Fs} (S16 and SY-2), \ac{Fm} (S32), and \ac{Focub} (S6) isolates collected in Tamil Nadu in the context of climate change. What are the impacts of climate change on \textit{Fusarium} disease dynamics in banana-growing regions, and have these newly described species emerged as a consequence of changing climate, globalisation, and/or expanded agricultural ranges? Fundamental questions must be asked about the environmental conditions in which the putative \ac{Fs} (S16 and SY-2), \ac{Fm} (S32), and \ac{Focub} (S6) isolates, collected in Tamil Nadu, can survive. Are the temperature, humidity, and host ranges for these isolates the same as \ac{Focub4}? 

Ultimately, Chapter \ref{Chap2} illustrates the diversity of \textit{Fusarium} and poses interesting questions about the dynamic nature of virulence, pathogenicity-associated genes, and genomic regions. Chapter \ref{Chap2} also raises concerns over the future of global banana production and highlights the risks posed by emerging pathogens. Based on these findings, we feel it is imperative that \ac{fwb} continue to be monitored, ideally through WGS of virulent isolates, and that accurate diagnostics and effective controls are studied and deployed (Chapter \ref{Chap2}).

\section{Concluding remarks} 

The argument that plant pathogens present a major threat to global food security is well-rehearsed \parencite{Bebber2014, Fones2020, Nelson2020, Ristaino2021}. Indeed, given globalisation, a changing climate, and the increasing global population, many have stressed that effective controls for crop pathogens will become even more important and that new technologies can contribute to the effective management of plant pathogens, helping to ensure global food security \parencite{Jeger2021, Rizzo2021, Bebber2023, Singh2023}. We have applied two omics-based approaches, genomic and metabolomic, to achieve the main objectives of this thesis. We have demonstrated that \acl{um} presents a new opportunity for understanding \ac{fwb}. We have developed and applied genomic tools to identify novel effector genes in \ac{Focub}, and shown that, not only can this tool be expanded to other \ac{Fo} \acp{fsp} but, that we can use it to contribute to our understanding of the diversity among \ac{Fo} \acp{fsp}. Finally, and perhaps concerningly, we identified potentially newly described species of \textit{Fusarium} causing \ac{fwb} in India through genomic analysis. 

Our methodologies align with the evolving landscape of biological research, often characterised by the integration of diverse omics techniques. The emergence of various omics approaches, including genomics, transcriptomics, proteomics, metabolomics, and phenomics, has revolutionised the study of plants and their interactions with biotic and abiotic stresses. As highlighted by \textcite{Backiyarani2022}, omics techniques have significantly contributed to banana research and improvement efforts in recent years. Moreover, we are witnessing a shift towards the era of "multi-omics", where these techniques and datasets are integrated to uncover novel associations between biological systems at various levels \parencite{Hasin2017}. We have applied both genomic and untargeted metabolomic analysis to study \ac{fwb}. Though we did not investigate links between the \ac{Focub} genome and metabolome, the integration of omics methodologies to study \ac{Focub} could provide valuable insights into the intricate mechanisms underlying plant-pathogen interactions and contribute to future advancements in \ac{fwb} research. 

As we uncover more about \ac{fwb}, exploring the virulence characteristics of \ac{Focub} and the effect of \ac{Focub} on the banana metabolome, we encounter new and unexpected findings that highlight the complexity of \ac{fwb}. These findings suggest that, to understand the complexities of \ac{fwb}, we must employ a multifaceted approach. We must prioritise surveillance, expand genomic resources, investigate virulence mechanisms, and apply new -omics-based techniques, fostering a collaborative, multidisciplinary approach. In this pursuit, I am reminded of Rachel Carson's observation that "in nature, nothing exists alone". It serves as a poignant reminder of the interconnectedness inherent in biological systems. Carson's words highlight the importance of maintaining vigilance and adaptability. By embracing this perspective, we uncover new opportunities for exploration and deepen our understanding of the dynamic relationship between pathogens, hosts, their environments, and us.


 % Omics, as defined by \textcite{Dai2022}, involves "probing and analysing large datasets representing the structure and function of an entire biological system at a specific level". 

% The dispersal of plant pathogens globally is an ever-growing concern \parencite{Bebber2014}; an issue that is further compounded by the influence of climate change on possible plant pathogen ranges, evolution, and host-pathogen interactions \parencite{Singh2023}. Much of the recent \ac{fwb} literature has focused on the threats posed by the spread of \ac{Focub4} \parencite{Ploetz2005, Ordonez2015a, Ploetz2015a, Ploetz2015b, Pegg2019, Viljoen2020}. Currently, reports of \ac{Focub4} in India have only come from the northern states of Uttar Pradesh and Bihar (see section \ref{sec:chap2Intro}). \textcite{Thangavelu2020} described isolates from the \ac{Focub1} VCGs 0125 and 01220 causing \ac{fwb} symptoms on the Cavendish cultivar ’Grand Naine’ (AAA) in Bihar, Uttar Pradesh, Gujarat, and Tamil Nadu. Our results caution that other \textit{Fusarium} species should not be overlooked in \ac{fwb} research and that \ac{Focub4} specific PCR diagnostics are unlikely to capture all disease-causing \textit{Fusarium} field isolates. 

% To build tools that aid in the identification of targets for molecular diagnostics and contribute to our understanding of \ac{Focub}, the \acf{maei} pipeline (Chapter \ref{Chap3}) was developed. Through the \ac{maei} pipeline we have identified candidate effectors in \ac{Focub}, and investigated the distribution of \acfp{cec} among other \ac{Fo} \acp{fsp}. This latter work was necessitated by COVID-19 impacting sampling in India and subsequently political issues associated with funding contributions to our lab's collaboration with \ac{tnau}.  To achieve the second objective of this thesis, the \ac{maei} pipeline is now being used to identify molecular diagnostic targets in \ac{Fo}, with work ongoing in \acf{Foa}. 

% Addressing our third objective of developing novel -omics approaches to improve understanding of banana interactions with \ac{Focub}, we conducted the first \acf{um} analysis of the \ac{Focub4} infected banana metabolome (Chapter \ref{Chap4}). Further, we identified features of interest from three wilt-inducing treatments: \ac{Focub4}-inoculated, \ac{xvm}-inoculated, or drought-stressed, opening new avenues for \ac{Focub}-banana research. 
