
Need for diagnostics - climate change changing pathogen ranges and profiles
    - link to new Fusarium species and need for constantly developing new diagnostics (for Fo.)
    
Viability of biomarkers over PCR, LAMP or imaging etc as a diagnostic? Lateral flows...
    - cost, time, access, ease of use?
    - AI image processing as new field... (smartphones)


New tools....


Omics, is defined by \textcite{Dai2022}, as "probing and analysing large datasets representing the structure and function of an entire biological system at a specific level". Numerous omics approaches, including genomics, transcriptomics, proteomics, metabolomics, and phenomics, have emerged in recent decades and have been instrumental in studying plants and their associated biotic and abiotic stresses. Indeed, \textcite{Backiyarani2022} provide a comprehensive review of omics techniques applied in banana research and their contributions to banana improvement. We are witnessing a transition into the era of "multi-omics", where these techniques and datasets are integrated to uncover novel associations between biological systems at various levels \parencite{Hasin2017}.

The adoption of multi-omics approaches for investigating plant diseases has become increasingly prevalent. 