\section{Introduction}
Alongside the genomic analysis, we developed and employed other -omics techniques to improve our understanding of some important biological parameters of Foc. The application of  a multi-omics approach to investigate plant diseases has become increasingly prevalent in recent years (Crandall \et 2020). Our phenomics analysis includes multispectral and RGB imaging as well as x-ray computed tomography (CT) to investigate pathogen spread through the plant and the hosts responses. 
Multispectral and RGB imaging studies in banana have primarily been establish for disease detection and diagnosis. Johansen et al., (2014) developed an automatic banana identification software to aid in Banana Bunch Top Virus inspection and Liao et al. (2018) employed hyperspectral images and machine learning to diagnose Banana Streak Virus at earlier stages of infection. Ochoa et al. (2016) designed a hyperspectral imaging system for disease scanning on banana plants focusing on Black Sigatoka. Unmanned aerial vehicles (UAVs), RGB imaging, and Artificial Neural Networks have also employed in the monitoring of Yellow Sigatoka (Calou, et al., 2020). Ye, et al. (2020a) demonstrated that UAV-based multispectral imagery can be used to diagnose Fusarium wilt, observing statistically significant differences (p > 0.05) between healthy and diseased plants using six different vegetation indices (VIs). RGB images alongside ML and artificial intelligence (AI) tools have been used to detect and differentiate between different disease in banana, including Foc, with the AI technology now available as a mobile phone application for banana growers (Selvaraj et al., 2019, Selvaraj et al., 2020).
It is clear that multispectral, hyperspectral, and RGB images can be used to diagnose Fusarium wilt, along with many other abiotic and biotic stresses, in banana. What is not clear, however, is what happening from a biological perspective to cause the differences in spectra observed, and whether spectra are different when comparing biotic and abiotic stresses. As Foc is a wilting pathogen it is important to ensure that Foc infection can be differentiated from drought stress and other wilting pathogens, such as Xanthomonas campestris pv. musacearum (Xvm). 
Something about pathogen progression internally and symptom development – then I can write about CT scanning here too! 

\section{Materials and Methods}\label{sec:Chapter4_MM}
\subsection{Plant maintenance }
\textit{In vitro }‘Grand Naine’ banana plants imported from France (VITROPIC, Saint-Mathieu-de-Tréviers, France) were maintained in 1/2L pots in \~400g of compost (Levington Advanced M2 compost, BHGS Ltd, UK), at 25℃ in 12-hour light and 70\% relative humidity. Plants were inoculated 8-12 weeks after arrival. 

\subsection{Fusarium oxysporum f. sp. cubense Tropical Race 4 culture maintenance}
A 50 \(\mu\)L droplet of Foc TR4 25\% glycerol stock (isolate UK0001, provided by Dr Will Kay, Exeter University) was added to 400ml of potato dextrose broth (PDB) and was maintained in a shaking incubator at \~130rpm and 25-28℃ (Max 30℃). After three days, the liquid culture was filtered through two layers of Miracloth, and the filtered suspension was adjusted to 1 × 10\textsuperscript{6} spores ml\textsuperscript{-1}. See Appendix X for media recipes.

subsection{Xanthomonas campestris pv. musacearum culture maintenance}
A 50 \(\mu\)L droplet of Xvm (isolate 5835, University of Warwick collection) 25\% glycerol stock was added to 10ml of yeast, peptone, and glucose broth (YPGB) and maintained in a shaking incubator at \~130rpm and 25-28℃ (Max 30℃). After two days, the culture was centrifuged at 3000 rpm for 10 minutes and the supernatant was discarded. The pellet was resuspended in sterile distilled water and diluted to O.D\textsubscript{600} of 0.5. 

\subsection{Fusarium oxysporum f. sp. cubense Tropical Race 4 inoculation protocol}
Plants were inoculated using the root-drench method from García-Bastidas \et (2019). Briefly, plant roots were wound by cutting with a blade through the soil at a 45° angle. A suspension of 1x10\textsuperscript{6} spores /g of soil was added to the soil by pouring. Negative controls were inoculated using sterile distilled water. Plants were maintained plants in trays covered with clear plastic bags to prevent pathogen spread. 

\subsection{Xanthomonas campestris pv. musacearum inoculation protocol}
A 5ml syringe with a 21-gauge needle was inserted into the pseudostem of the plant approximately 1cm from the base. 2ml of the 108 CFU ml-1 Xvm cell concentration (O.D\textsubscript{600} = 0.5) was slowly introduced into psuedostem. Negative controls were inoculated using sterile distilled water. Plants were maintained plants in trays covered with clear plastic bags to prevent pathogen spread. 


\subsection{Sampling and multispectral image collection}
Samples from four treatments were collected. Plants were inoculated with Foc or Xvm or were exposed to drought stress (no watering from the time of inoculation). A water mock involution treatment group was also established. Plants were images at one-, two-, and three weeks post-inoculation. 
Images were captured using a Sony Alpha 7Rii modified camera body with 10 band lens for a Full-Frame sensor (405-850nm) (Agrowing Ltd., Israel). Plants were imaged individually. Images were captured at 2m above the canopy using a custom imaging system. Images were aligned and analysed using the software Agrowing Basic V.1.2.
\subsection{X-ray computed tomography image collection}
We scanned both a single leaf sample and a section of pseudostem from each sample (2 Control Plants; 4 Xanthomonas Plants; 4 FOC Plants) on our Tescan Unitom XL system. The settings used can be found in the table below. All scans were reconstructed using the standard FDK algorithm.
The X-Ray Computed Tomography (XCT) data used in this report was acquired using the Free-at-Point-of-Access scheme at the National Facility for X-Ray Computed Tomography (NXCT) and carried out at the Centre for Imaging, Metrology, and Additive Technologies (CiMAT) at the University of Warwick under the EPRSC Project Number (EP/T02593X/1).

\section{Results }
\subsection{Multispectral imaging displays…}
\subsection{X-ray computed tomography reveals…}

\section{Discussion and Conclusion}

