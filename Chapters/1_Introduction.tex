\section{Bananamania: Cultivation to Cultural Significance of \textit{Musa}}

Banana (\textit{Musa} spp.) is an agricultural commodity of significant global importance, contributing to food security and economic stability. Like many other crop species, pathogens pose a major threat to global banana production. \ac{fwb} is ranked highly among these pathogens and threatens the future role of bananas as a staple food and cash crop as well as cultural heritage preservation \parencite{Kema2021}. 

\subsection{Anatomy and Classification of Banana}  

Banana (\textit{Musa} spp.) belongs to the Musaceae family in the order Zingiberales (Table ~\ref{tab:TaxonTable}) \parencite{Schoch2020}, a genus of monocotyledonous, herbaceous perennials containing 86 species currently accepted by the World Checklist of Selected Plant Families \parencite{WCSPF2023}. Linnaeus originally described only two species of banana: \textit{M. sapientum }and \textit{M. paradisiaca}. However, the classification of banana using the formalised binomial system proved ineffective due to banana’s complicated genetic history.

In the mid-20th century, \textcite{Cheesman1947} recognised the two species described by Linnaeus to be cultivars of the wild \textit{Musa} species, \textit{M. acuminata }and \textit{M. balbisiana} and regrouped the remaining \textit{Musa} species into 4 ‘sections’. Later, \textcite{Simmonds1955} developed a genome-based, informal nomenclature system for cultivated banana which is still used today. The system classifies varieties into genome groups based on the ploidy contribution of their wild ancestors, where the species contribution is denoted by the letters ‘A’ and ‘B’ for \textit{M. acuminata} and \textit{M. balbisiana}, respectively. For example, “Cavendish banana (\textit{Musa } spp. AAA group) cv. ‘Grand Nain’” indicates the variety’s sub-group (Cavendish), genus (\textit{Musa}), genome group scoring (AAA), and cultivar (Grand Nain).

The anatomy of banana plants is outlined in detail by \textcite{Bakry2009} and \textcite{Robinson2010}. Banana plants produce a fibrous root system from a tuberous rhizome. Extended horizontal growth, typical of most rhizomatous plants, is not observed in banana; the plant instead produces suckers which successively grow outwards from the short rhizome. The leaves are produced from the central meristem of the rhizome, or developing sucker, and laminae continue to widen until they mature. As leaf sheaths develop, they become tightly packed and thicken to form the pseudostem which elongates as more leaves emerge to 1-8m in height (Figure ~\ref{fig:Anatomy of fruiting banana plant}). 

% Please add the following required packages to your document preamble:
% \usepackage{longtable}
% Note: It may be necessary to compile the document several times to get a multi-page table to line up properly
\begingroup
\setlength{\tabcolsep}{15pt} % Default value: 6pt
\renewcommand{\arraystretch}{0.9}
\begin{longtable}[c]{cc}
\caption[Taxonomy of \textit{Musa} genus]{\textbf{Taxonomy of \textit{Musa} genus including 3 example species} (Schoch et al., 2020, WCSP, 2022).}
\label{tab:TaxonTable}\\
\hline
\textbf{Kingdom}     & Plantae       
\endfirsthead
%
\multicolumn{2}{c}%
{{\bfseries Table \thetable\ continued from previous page}} \\
\endhead
%
\textbf{Phylum}      & Tracheophytes \\ 
\textbf{Class}       & Magnoliopsida \\ 
\textbf{Super order} & Lilianae      \\ 
\textbf{Order}       & Zingiberales  \\ 
\textbf{Family}      & Musaceae      \\ 
\textbf{Genus}       & \textit{Musa} \\ 
\textbf{Species} & \textit{\begin{tabular}[c]{@{}c@{}}M. acuminata,\\
M. balbisiana, \\
M. basjoo\end{tabular}} \\ \hline
\end{longtable}
\endgroup

\begin{figure}[hp!]
    \centering
    \includegraphics[width=12cm]{Figures/Diagrammatic-representation-of-a-fruiting-banana-plant-with-suckers-in-Bakry-et-al_W640.jpg}
    \caption[Anatomy of fruiting banana plant]{\textbf{Anatomy of fruiting banana plant} (From \textcite{Bakry2009}).}
    \label{fig:Anatomy of fruiting banana plant}
\end{figure}
\clearpage

\subsection{The History of Banana}

Originating from Southeast Asia, naturally occurring parthenocarpic – development of fruit without fertilisation – individuals of banana have been cultivated for about 10,000 years. Evidence of this can been seen in \textit{Musa banksii} phytoliths discovered in the Kuk swamp in Papua New Guinea \parencite{Denham2011}.  Cultivated bananas were spread from Papua New Guinea, hybridising with other subspecies of \textit{M. acuminata} and \textit{M. balbisiana} in the Philippines and northern New Guinea \parencite{Perrier2009}. Triploid cultivars were produced as a consequence and were dispersed throughout Southeast Asia, Northern Australia, East Africa, and South Asia by the Austronesian peoples; where, although primarily cultivated as a food crop, banana was used in traditional medicine to treat ailments including dysentery, ulcers, leprosy, epilepsy, and insect bites \parencite{Kumar2012}. In these regions, banana has also been used as fodder; in domestic materials, such as plates, children’s toys, and dyes; in building for shelter, rafts, or rope; and has significant social and cultural value, involved in many traditional rituals and customs \parencite{Hapsari2017} .  

Banana cultivation continued to move into the Middle East and Northern Africa from the centre of domestication in Southeast Asia, which can be seen in references to banana in Islamic texts such as Ibn al-'Awwam's 12th-century agricultural work, \textit{Book on Agriculture} \parencite{Clement1866}. Later, during expeditions in Southeast Asia and Africa in the mid-1500s, bananas were encountered by Europeans who transported plants to South America where colonists established banana plantations to supply the USA and Europe \parencite{Guzman-Rivas1960, Salas-Pascual2022}. This legacy continues today, with over 70\% of the EU’s banana supply coming from South American countries \parencite{EuropeanUnion2022}.  

\subsection{Cultivation and Value of Banana}

Global production now takes place year-round in tropical and subtropical regions where plants are commercially grown as a monoculture crop. Although, plants are also cultivated by smallholders and in subsistence production systems \parencite{Viljoen2020}. Commercially, fruits are harvested within about 9-12 months of planting (either\textit{ in vitro} propagated material or suckers) and are grown as a ratoon crop – the plant is cut back to the rhizome once the fruit has been harvested and left to regrow – for several years, by which time yields will have reduced and a new plantation must be generated. 

Today, banana is frequently reported as one of the world’s top agricultural commodities, the annual global production of bananas (including plantain) steadily increased from 98.4 million tonnes in 2000 to an estimated 155.2 million tonnes in 2018  and global exports increased from 15,540 tonnes in 2011 to 20,465 tonnes in 2021 \parencite{FAO2022}. However, only around 15\% of bananas produced worldwide are traded in international markets, the remaining 85\% are consumed locally, highlighting the value of bananas as a subsistence crop. Particularly in developing countries where fruits can account for 25\% of daily calorie intake and act as a source of income \parencite{FAO2019}. The world’s largest banana growers are India and China which produced over 30 million tonnes and 11 million tonnes in 2021, respectively \parencite{BananaLink2023} (Figure ~\ref{fig:bananaProdMap}).  Currently, up to 40\% of all global banana production (domestic and export markets) is reliant on the Cavendish bananas \parencite{Warman2018}, a triploid, clonally propagated subgroup deriving from \textit{M. acuminata}.

\begin{figure}[h!]
    \centering
    \includegraphics[width=15cm]{Figures/BananaProdMap.pdf}
    \caption[Global banana production, 2021]{\textbf{Global banana production 2021.} No data available for countries in grey  (Data source: BananaLink, 2023).}
    \label{fig:bananaProdMap}
\end{figure}



\section{Bananageddon: An Introduction to \acl{Focub}}

In recent decades, there have been reports of a potential ‘Bananageddon!’, whereby global banana production is decimated by the fungal pathogen \acf{Focub} \acf{tr4}, the causative agent of \acf{fwb} (syn. Panama disease). The global banana community’s concern is demonstrated through the establishment of the \acs{tr4} Task Force, part of the \ac{FAO} \ac{wbf}; the Peruvian government’s declaration of a National Emergency upon the discovery of \ac{Focub4} in 2021 (FAO, 2021); and in the review by \textcite{Westerhoven2022}, where the authors stress \ac{Focub4}'s threat to African food security.   


\subsection{A Brief History of Fusarium Wilt}

This robust response to \ac{Focub4} is not unfounded. During the first half of the 20th century, \ac{Focub1} \ac{r1} devastated global production, destroying  ~40,000 hectares of ‘Gros Michel’ plantations \parencite{Agrios2005, Kema2021}, and resulted in estimated economic losses of \$400 million (USD) between 1940 and 1960 \parencite{Ploetz2005}. \ac{Focub1} eventually led to the disappearance of ‘Gros Michel’ as an export banana in the 1960s \parencite{Molina2007}, when the \ac{Focub1} resistant Cavendish bananas subsequently replaced ‘Gros Michel’ \parencite{Ordonez2015}. 

The threat posed by \ac{Focub} still remains, however. In the 1960s Cavendish banana plants in Taiwan were observed displaying symptoms of \ac{Focub} infection \parencite{Agrios2005}. Identified in 1994 as \acl{Focub4} \parencite{Ploetz1994} the strain affecting Cavendish bananas has continued to spread. \Ac{Focub4} is now found in Africa, Asia, and Europe \parencite{Ploetz2015a, Thangavelu2019} and is gaining a foothold in South America, with recent incursions in Columbia in 2019 \parencite{Garcia-Bastidas2019}, Peru in 2021 \parencite{Acuna2022}, and Venezuela in 2023 \parencite{Herrera2023}.

\subsection{Disease Development of \acl{Focub}}

As a soil-borne pathogen, \ac{Focub} first colonises the roots. Asexual spores found in the soil germinate in response to host root exudates and infection hyphae produced from resting spores (chlamydospores) penetrate the root epidermis at the tip. \ac{Focub} hyphae progress through the root cortex to the rhizome and migrate through the xylem in the outer leaf sheaths of the pseudostem, occluding vessels and interfering with nutrient uptake and upward water transport; often before visible symptoms are observed \parencite{Li2017, Warman2018}. Two asexual spore types, microconidia and macroconidia, are produced within the xylem and help \ac{Focub} overcome some of the host’s barriers, such as naturally occurring end walls or perforation plates of the xylem strands, which somewhat restrict pathogen movement through the host \parencite{Dita2018}. Xylem vessels begin to collapse, and external symptoms are observed. Tissues become chlorotic and the plant withers as it transpires more than it can translocate. Once the plant dies, the fungus begins to invade other plant tissues and sporulate, particularly on the plant surface, producing chlamydospores (Figure: \ref{fig:MyLifeCycle}). Chlamydospores can persist in the soil for decades and are resistant to adverse conditions (Pegg \et 2019).  

Alternative hosts, including many grass and weed species, also act as a reservoir of inoculum \parencite{Hennessy2005}.  \ac{Focub} is dispersed through the movement of contaminated plant and soil material, water, wind, and people. Typical symptoms of \ac{Focub} infection include pseudostem splitting and discolouration, yellowing of lower leaves often forming a ‘skirt’ around the plant, and, occasionally, a streaking pattern is observed on the foliage (Figure ~\ref{fig:FusariumWiltSymptoms}). 

\newpage{}
\begin{figure}[p!]
    \includegraphics[width=16cm]{Figures/MyLifeCylceNarrow.pdf}
    \caption[Fusarium wilt of banana disease cycle.]{\textbf{Fusarium wilt of banana disease cycle.} Chlamydospores germinate in the soil in response to plant root exudates. Infection hyphae from chlamydospores penetrate the root tip epidermal cells and progress through the host root cortex to the xylem. Microconidia and macroconidia are produced in the xylem and help to facilitate pathogen movement through the host. As \acl{Focub} migrates through the xylem it occludes vessels and interferes with nutrient uptake and upward water transport. External wilting symptoms begin to develop; tissues become chlorotic and the plant withers as it transpires more than it can translocate. Eventually, the plant dies, and asexual spores are formed on the dead tissue. Chlamydospores persist in the soil or \acl{Focub}  survives endophytically in an alternative host species. Fungal hyphae are shown in blue. Images of \acl{Focub}  spores adapted from Fourie \et  (2011) and symptomatic banana plant images sourced from Maymon \et  (2020). Figure created with BioRender.com.}
    \label{fig:MyLifeCycle}
\end{figure}
\clearpage


\begin{figure}[pt!]
    \centering
    \includegraphics[width=15cm]{Figures/SymptomsofFoc.pdf}
    \caption[Typical symptoms of Fusarium  wilt of banana.]{\textbf{Typical symptoms of \textit{Fusarium} wilt of banana.} (a) Splitting of the psuedostem, (b) discolouration of the psuedostem, (c) yellowing of lower leaves, and (d), a streaking pattern is observed on the foliage.}
    \label{fig:FusariumWiltSymptoms}
\end{figure}

\subsection{Classification of \acl{Focub}}

\ac{Focub} belongs to the \ac{FOSC}. The species complex contains both plant and animal pathogens, as well as saprophytes and endophytes \parencite{Leslie2006}. \acs{Fo} has high functional and genetic diversity, shown plainly in its broad plant host range. The species complex can infect around 150 species of both monocotyledonous and dicotyledonous plants, including many horticulturally important crops such as banana, tomato, and lettuce \parencite{Edel-Hermann2019}.  Individual pathogenic \ac{Fo} isolates can infect only one or a few plant species. Plant pathogenic \ac{Fo} are therefore divided into ‘special forms’, or \ac{fsp}, each of which is adapted to specific hosts \parencite{Snyder1940}. For instance, strains which cause Fusarium wilt in tomato are classified as \ac{Fol}. To date, 106 \ac{fsp} have been clearly described and with at least an additional 37 putative \ac{fsp} \parencite{Edel-Hermann2019}.   

Some \ac{fsp} can be further subdivided into pathogenic races, determined by pathogenicity towards specific host cultivars. Three races of \ac{Focub} affecting banana have been identified. \acf{r1}  affects banana cultivars such as ‘Gros Michel’ (AAA), ‘Maqueno’, ‘Silk’, ‘Pome’ (AAB), and ‘Pisang Awak’ (ABB); \acf{r2} affects ‘Bluggoe’ and other cooking bananas (ABB); Race 4 affects the subgroup Cavendish (AAA) as well as hosts of \ac{r1} and \ac{r2} \parencite{Ploetz2015a} and has been further divided into \ac{str4} and \acl{tr4}. \Ac{tr4} is distinguished from \ac{str4} as it affects susceptible cultivars without disease-predisposing conditions \parencite{Ploetz2015b}.  

Identification of isolates based on race is often subjective. During the 1980s an additional method of \ac{Focub} identification based on \ac{vcg} was developed \parencite{Correll1991}. \acp{vcg} are comprised of isolates that possess the same alleles at all \ac{vic} loci, and so can form stable heterokaryons with each other \parencite{Correll1991}. So far, 24 \acp{vcg} of \ac{Focub} have been identified (Table ~\ref{tab:VCGTab}) \parencite{Czislowski2018}. Some \acp{vcg} of \ac{Focub} are taxonomically closer to other \ac{fsp} of \ac{Fo} than other \ac{Focub} \acp{vcg}. This, coupled with distinct genetic lineages identified through phylogenetic analysis, has resulted in the recognition of a polyphyletic – taxonomic group including members from different ancestral lineages – evolutionary history for \ac{Focub} \parencite{Koenig1997, Ploetz2007}. 

\textcite{Maryani2019} revised the taxonomy of \ac{Focub}, generating the Fusarium of Banana Species Complex and proposing several new species. However, \textcite{Torres2021} have suggested that the revision is premature and that the data do not fully support the revision. There continues to be active debate surrounding \ac{Focub} classification and a clear, robust, universally accepted taxonomy has not yet been established. Race, although somewhat informal, remains the most widely accepted method of \ac{Focub} classification – whereby isolates are identified based on virulence towards specific host cultivars.  

\afterpage{

%Insert the VCG table
% Please add the following required packages to your document preamble:
% \usepackage{longtable}
% Note: It may be necessary to compile the document several times to get a multi-page table to line up properly
\begingroup
\setlength{\tabcolsep}{10pt} % Default value: 6pt
\renewcommand{\arraystretch}{0.67}
\begin{longtable}[tc!]{ccc}
\caption[\acf{vcg} of \ac{Focub}]{{\textbf{\acf{vcg} of \ac{Focub}} and race (Czislowski \et 2018).} Cross-compatible VCGs can form VCG complexes.}
\label{tab:VCGTab}\\
\hline
\textbf{VCG} & \textbf{VCG   complex} & \textbf{Race}    \\ \hline
\endfirsthead
%
\multicolumn{3}{c}%
{{\bfseries Table \thetable\ continued from previous page}} \\
\endhead
%
0120  & 0120–01215           & STR4             \\
0121  &                      & R4               \\
0122  &                      & R4               \\
0123  &                      & R1               \\
0124  & 0124–0125–0128–01220 & R1,   R2         \\
0125  & 0124–0125–0128–01220 & R1, R2           \\
0126  &                      & R1               \\
0127  & No longer valid      & No longer valid  \\
0128  & 0124–0125–0128–01220 & R1,   R2         \\
0129  & 0129–01211           & STR4             \\
01210 &                      & R1               \\
01211 & 0129–01211           & STR4             \\
01212 &                      & Not   Identified \\
01213 & 01213–01216          & TR4              \\
01214 &                      & R2               \\
01215 & 0120–01215           & STR4             \\
01216 & 01213–01216          & TR4              \\
01217 &                      & R1               \\
01218 &                      & R1               \\
01219 &                      & Not identified   \\
01220 & 0124–0125–0128–01220 & R1               \\
01221 &                      & Not Identified   \\
01222 &                      & Not   Identified \\
01223 &                      & Not Identified   \\
01224 &                      & Not   Identified \\ \hline
\end{longtable}
\endgroup

\clearpage
}

\subsection{The \acl{Fo} genome} 

An emerging explanation for shared host specificity between different \ac{Fo} lineages relates to the compartmentalisation of the \acl{Fo} genome. A theory which has become increasingly prevalent in \ac{Fo} genomics following the work of \textcite{Ma2010}. This study explains that the \textit{Fusarium} genome can be organised into two parts containing either \acp{cc} or \acp{ac} (syn. pathogenicity/lineage-specific/supernumerary chromosomes). The 11 \acp{cc} within the \ac{Fo} genome encode functions necessary for primary metabolism and reproduction \parencite{Dam2017}. The remaining \acp{ac}, which vary in number among \ac{fsp}, are not required for primary metabolism and reproduction, contain secondary metabolite biosynthetic gene clusters, encode virulence genes such as effectors, and possess a high number of \acp{te} \parencite{Ma2010, Schmidt2013, Witte2021}. Further, \textcite{Ma2010} suggests these \acp{ac} can be horizontally transferred and afford host-specific pathogenicity to the recipient. A theory which has since been demonstrated in \ac{Fol} \parencite{Vlaardingerbroek2016, Li2020}, \ac{Fo} f. sp. \textit{radicis-cucumerinum} \parencite{Dam2017} and \ac{Fo} f. sp.  \textit{melonis} \parencite{Li2020}. 
 
Not only do \acp{ac} go some way to explaining diversity between \ac{fsp}, but also among strains within specific \ac{fsp}. For instance, a recent preprint by \textcite{Westerhoven2023} analyses the dynamics of \acp{ac} in \ac{Focub} and explores the role segmental duplications have in \ac{ac} evolution and diversity within \ac{Focub}. \textcite{Westerhoven2023} sequenced and assembled 69 \ac{Focub} strains (including \ac{r1}, \ac{r2}, and \ac{tr4})  isolated from various banana-growing regions. They generated chromosome-level genome assemblies and identified \acp{ac} in these strains.  The authors reported a significant amount of variability of the \acp{ac} between the \ac{fwb} strains, for instance, \acp{ac} ranged in size from 6.7Mb to 16.5Mb and exhibited extensive diversity in gene content, repeat content, and GC content when compared to the \ac{Focub} core chromosomes. \Acp{ac} are most commonly shared among strains of \ac{Focub1} from the same lineages, and few \acp{ac} are shared between \ac{r1} lineages. \Acp{ac} in \ac{Focub4} strains, however, showed low nucleotide diversity and the authors suggest that \ac{Focub4} evolved from a single recent clonal origin. 

This manuscript highlights the prevalence of segmental duplication in \acp{ac} and sub-telomeric regions of \ac{Focub}. The authors demonstrate that these duplications are associated with the recent evolution and diversification of genes, including virulence genes, and play a crucial role in the expansion and adaptation of \acp{ac} in the \ac{FOSC} genome. However, it is worth noting that this manuscript has not currently published the \ac{Focub} assemblies generated, nor does it provided an appropriate data set detailing the assembly quality of the isolates sequenced for the study, making it challenging to validate the authors conclusions. 

\subsection{The Plant Innate Immune System}

As a plant pathogen, \ac{Fo} faces considerable evolutionary pressure to develop diverse mechanisms that can evade host detection and facilitate successful infection. Due to the relatively recent evolutionary history of \acp{ac}, wide diversity, ability to function separately from \ac{Fo}'s \ac{cc}, and a high number of \acp{te} suggests that this specific section of the genome serves to fulfil the pathogen's adaptive needs in its interaction with hosts and evade plant defence mechanisms.

Plants have coordinated defence mechanisms which enable them to detect and respond to invading pathogens effectively. The plant’s innate immune system is comprised of two interconnected tiers which perceive and respond to pathogen infection \parencite{Jones2006, Han2019, DeFalco2021}. The first tier recognises microbial- or \ac{pamp} which are slow-evolving, conserved molecules (such as the fungal cell wall component, chitin) using transmembrane \ac{prr} on the plant cell surface. Once \ac{pamp} have been recognised, \ac{pti} is induced, including numerous cell signalling events such as altered ion fluxes, the production of apoplastic reactive oxygen species, callose deposition, and transcriptional reprogramming. 

To evade, suppress or interfere with \ac{pti} responses and establish infection, fungal pathogens secrete proteins termed ‘effectors’ into the host cell which results in \ac{ets}. To combat \ac{ets}, plants possess a suite of \ac{rprot} which perceive effectors (termed '\ac{AVR} proteins') and induce further immune responses, known collectively as \ac{eti} \parencite{Jones2006} (Figure:~\ref{fig:PlantImmuneSystem}). \Ac{rprot} are conserved intracellular receptors of the \ac{nlr} class. The structure, function, and effector perception (direct or indirect) of \ac{rprot} is variable and often specific (see: \textcite{Chen2022, Wang2022}). Similarly, effectors are typically highly variable and specific to individual pathogens \parencite{LoPresti2015}. \textcite{Gutierrez2023} have prepared a comprehensive review of virulence factors in the \textit{Fusarium} genus. 

\begin{figure}[h]
    \centering
    \includegraphics[width=14cm]{Figures/DoddsArticleModel.pdf}
    \caption[Overview of the plant immune system.]{\textbf{Overview of the plant immune system.} Pattern recognition receptors (PRRs) perceive microbial- pathogen-associated molecular patterns (MAMPs/PAMPs) and induce a PAMP-triggered immunity (PTI) response from the host. In order to overcome host cell defences, pathogens secrete small molecules known as effectors into the host cell, inducing effector-triggered susceptibility (ETS). Plants have adapted a family of resistance proteins (R proteins), which can perceive pathogen effectors (avirulence proteins) and stimulate an effector-triggered immunity (ETI) response from the plant host (From Dodds and Rathjen, 2010).}
    \label{fig:PlantImmuneSystem}
\end{figure}


\subsection{The \acl{Fo} Effectorome}

In the \ac{FOSC}, the only family of effectors currently recognised are the \acp{sixg}. In total, 14 SIX proteins (SIX1 – SIX14) have been identified through mass spectrometry of infected tomato xylem sap and whole genome sequencing \parencite{Houterman2007}.  All 14 \acp{sixg} are found on the \acp{ac} in \ac{Fol} \parencite{Schmidt2013}, and studies have confirmed that \textit{SIX1, SIX2, SIX3, SIX5}, and \textit{SIX6} are essential for conferring virulence in tomato, with  \textit{SIX4} (\textit{Avr1}), \textit{SIX3} (\textit{Avr2}), and \textit{SIX1} (\textit{Avr3}) all recognised by corresponding \ac{rprot} (called ‘I-’ for Immunity) in tomato \parencite{Rep2004, Lievens2009, Takken2010, Gawehns2014, Ma2015}.  

Since their discovery in \ac{Fol}, homologs of \acp{sixg} have been identified in other \ac{fsp}, including \textit{cepae, conglutinans, fragariae, melonis, pisi, radicis-cucumerinum,} and \textit{vasinfectum} \parencite{Czislowski2018}. A further \textit{SIX} gene, (\textit{SIX15}) has been described by \textcite{Simbaqueba2018} in \acl{Fo} f. sp. \textit{physalis}, although \textit{SIX15} is not widely reported. 

In \ac{Focub}, homologs of \textit{SIX1, SIX2, SIX4, SIX6, SIX7, SIX8, SIX9}, and \textit{SIX13} have been identified, and \textit{SIX1} and \textit{SIX8} have been recognised in \ac{Focub} \ac{tr4} as essential for virulence through knockout mutants \parencite{ Widinugraheni2018, An2019}. \textcite{Czislowski2018} reported differences in the \ac{sixg} repertoire of \ac{Focub} \ac{r1} and \ac{r4} lineages and hypothesised that the differences in \ac{sixg} profiles contribute to the variability in the \ac{Focub} host cultivar range.  However, as suggested, \acp{sixg} are unlikely to be the only set of effectors involved in host specificity, and state "it should be expected that novel and undiscovered effectors remain to be identified in the lineages of \ac{Focub}" (Czislowski \et 2018 pp. 1168).  

\textcite{Ma2010} showed that the pathogenicity chromosomes in \ac{Fol}, have a high number of retro-elements, including the non-autonomous class II \ac{te}, \ac{mites}. A specific family of \ac{mites}, known as \acp{mimp}, are found upstream of all known \acp{sixg} in \ac{Fol} \parencite{Schmidt2013}. Characterised by their \ac{tir} sequence and short length (180–220 \acs{bp}), \textcite{Schmidt2013} surmised that \acp{mimp} could be used for the prediction of effectors in \ac{Fo}, particularly \acp{sixg}, a theory which was later demonstrated by \textcite{Dam2016, Dam2017, Armitage2018, FoEC2}.  

The hypothesis that effector profiles contribute to host specificity is reported for other \ac{fsp} \parencite{Achari2021, Batson2021} and is argued on the host species level for the \ac{FOSC} \parencite{Dam2016}. With advances in genome sequencing and analysis, it is possible to identify new candidate effectors using common characteristics shared between known effectors in \ac{Fol} and other \ac{fsp}, as well as their location in the \ac{Fo} genome and evolutionary relationship.

\subsection{Control of \acl{Focub}}

There are few effective controls for \ac{Focub} in the wider environment. Chlamydospores can persist in the soil for decades and \ac{Focub} can invade non-host plants and survive as an endophyte, with grass and weed species acting as sources of inoculum \parencite{Pegg2019}. Biological control agents and fungicides have been investigated as a means of \ac{Focub} control. However, most of these are \textit{in vitro} and greenhouse assays which, although successful, are not practical or effective in-field \parencite{Dita2018}.  

As most commercial bananas are triploid and parthenocarpic, conventional breeding for resistance is exceedingly challenging \parencite{Dale2017}. One somaclonal variant of Cavendish, ‘Formosana’ is marketed as being resistant to \ac{tr4}, however, yields are slightly reduced due to its longer growth cycle and there have been reports of  ‘Formosana’ showing \ac{tr4} susceptibility in later field trials \parencite{Lee2011, Dale2017}. A further somaclonal variant, ‘Guijiao 9’, is also marketed as showing lower disease severity compared to conventional commercial Cavendish varieties such as ‘Williams’ \parencite{Sun2019}, but the variety has only recently been released to the market and field trials outside of the region, in which this variety was developed, have not yet been conducted.  

Advances in genetic modification technologies have facilitated the development of two \ac{tr4}-resistant transgenic Cavendish lines \parencite{Dale2017}. One of these lines, ‘QCAV-4’ is currently undergoing regulatory approval in Australia \parencite{Lu2023}. The transgenic 'QCAV-4' variety was generated by transforming \textit{M. acuminata }Cavendish cv. 'Grand Naine' (AAA subgroup) embryogenic cell suspensions (prepared from immature male flowers) using the centrifugation assisted \textit{Agrobacterium tumefaciens}-mediated method with \ac{rga2}. \ac{rga2} is a putative \ac{nlr}-type \ac{rgene} from a seedling of \textit{M. acuminata} ssp. \textit{malaccensis} with resistance to \ac{Focub4}. However, the market for genetically modified crops within Europe is limited, although debate and legislation are currently under review and development. 

Gene-edited banana may be a more marketable alternative, particularly for Europe, and editing for disease resistance has been demonstrated in banana. \textcite{Tripathi2021} used CRISPR-Cas9 to make mutations to the downy mildew resistance 6 (DMR6) orthologue in banana, showing enhanced resistance to the bacterial pathogen \ac{xvm}, this approach may be applied to target a gene which could provide resistance to \ac{Focub}.  

\section{Bananomics: Tools and Technologies to Identify \textit{Fusarium} Wilt of Bananna}

Prevention of \ac{Focub} introduction has so far proven the most effective method of \ac{Focub} control on a large scale \parencite{Ploetz2015b}, with management primarily focused on hygiene practices and prevention. The development of accessible, fast, and accurate diagnostic tools is imperative. Molecular and image-based diagnostics have both been developed to identify \ac{Focub}. However, many of the current image-based diagnostics are unable to distinguish \ac{Focub} from other banana wilt pathogens, and the molecular assays struggle to accurately identify pathogenic races.  

\subsection{Molecular Diagnostic Approaches}
\Ac{pcr}-based methods of identification are routinely employed in the detection of \ac{Focub}, alongside \ac{vcg} screening and race typing. Assays have been developed by \textcite{Lin2009, Dita2010, Li2013a, Li2013b}. A study by \textcite{Magdama2019} reviewed the efficacy of these \ac{pcr}-based detection methods for the identification of \ac{Focub4} against 302 \textit{Fusarium} isolates, including 292 \ac{Fo} isolates. The authors demonstrated that the primers developed by Lin \et (2009) and Li \et (2013a) for \ac{str4} and \ac{tr4} identification displayed false positives for \ac{r1} and \ac{r2} as well as five non-pathogenic \ac{Fo} endophytes. Similarly, primers developed by Dita \et (2010) showed false positives for \ac{Fo} endophytes. 

\textcite{Magdama2019} concluded that the most effective primers for \ac{tr4} identification are those developed by \textcite {Li2013b} and that, due to the diversity of the \ac{FOSC}, a wider range of \ac{Fo} isolates should be assessed when developing assays, and that primer design should focus on a genetic locus related to virulence to limit false positives.  Indeed, the gene targeted for diagnostics by \textcite {Li2013b} was implicated in virulence as a result of a T-DNA insertion conferring reduced virulence. However, to date there has been  no further characterisation of the gene by deletion, editing, or complementation to confirm its function.  

\textcite{Ordonez2019} used genome data and Diversity Array Technology sequencing to identify new genomic markers for \ac{Focub4} identification to develop a \ac{lamp} assay for in-field \ac{Focub4} diagnostics. The \ac{lamp} primers were assessed for specificity against 67 isolates, including 22 \ac{tr4} isolates, 28 alternative \ac{Focub} \acp{vcg}, 16 other \ac{Fo} \ac{fsp} isolates, and one \textit{Ralstonia solanacearum} isolate. Although thorough, \textcite{Ordonez2019}  did not survey as many isolates as \textcite{Magdama2019}. Further research is required to determine if the sequences identified adhere to the recommendations of \textcite{Magdama2019}. 

\subsubsection{Effectors as a Target for \acl{Focub} Diagnostics}

Ideally, in order to develop molecular diagnostics which successfully distinguish between various \ac{Focub} races, \ac{Fo} \ac{fsp}, and endophytes, a key genetic locus with an important role in pathogen virulence, such as effectors, should be targeted. Since the review by \textcite{Magdama2019}, diagnostic assays using \ac{pcr}-based methods targeting \ac{sixg} in \ac{Focub} have been developed by \textcite{Carvalhais2019}. These assays include simplex and duplex \acp{pcr} with additional restriction digestion steps. Primers specifically amplify regions of \textit{SIX6} in \ac{r1}, \textit{SIX1} in \ac{tr4}, \textit{SIX8} in \ac{str4}, \textit{SIX9/SIX10} in \ac{Focub} \ac{vcg} 0121 (R4), and \textit{SIX13} in \ac{Focub} \ac{vcg} 0122 (R4). The assays were validated using 250 \textit{Fusarium} isolates, including 16 different plant-associated \textit{Fusarium} species and 21 \acp{vcg}  commonly associated with \ac{Focub}. Amplicons were not generated for 5 of the 65 \ac{r1} isolates and the assay for \ac{Focub} \ac{vcg} 0121 generated false positives for three out of the 32 \ac{tr4} isolates. Furthermore, no known \ac{r2} isolates were used when validating the assays. Although effective in identifying isolates within R4, these assays arguably lack specificity for effective, accurate diagnostics of \ac{r1}, and specific groups within R4, and may provide false positives should an \ac{r2} isolate be sampled. This study indicates that \acp{sixg} provide a potential target for \ac{Focub} diagnostics, and access to a greater number of \ac{Focub} genomes may allow for more accurate primer development.  

\textcite{Chang2020} identified putative effectors in \ac{Focub} using a \ac{mimp}-based approach. They identified candidate effectors in \ac{Focub} by searching local genome databases of \ac{Focub1} and \textit{Foc} R4 using 13 \ac{mimp} sequences from \textcite{Dam2016}. They identified 20 candidate effectors; 3 unique to \ac{Focub1}, 6 unique to \textit{Foc} R4. To demonstrate the accuracy of their identification method, they generated a knockout mutant and its complement of one candidate effector, gene \textit{Foc 1324}. Virulence tests on banana plants demonstrated that \textit{Foc 1324} was a virulence factor required for the pathogenicity of \textit{Foc} R4. Confirmation of function is still required for all candidate effectors in this study. Furthermore, candidate effectors were only identified in two genomes, one \ac{r1} genome and one R4 genome, \acp{vcg} for these genomes are not provided. Candidate effectors which \textcite{Chang2020} report are unique to \ac{r1} and R4 require further validation using more genomes representing different \ac{Focub} races.

Interestingly, the 13 mimp sequences identified by \textcite{Dam2016} in \ac{Focub} were found across three isolates, a considerably lower number of \acp{mimp} compared to other \ac{fsp} in their study. Subsequent investigation revealed that the method of \ac{mimp} identification employed by \textcite{Dam2016} was case sensitive (only searched for capitalised \ac{mimp} \acp{tir}). Two of the \ac{Focub} genomes (B2 and N2) used had been repeat masked, therefore many potential \acp{mimp} in \ac{Focub} are likely to have been missed. A greater number of \ac{mimp}-associated effectors could have been discovered by \textcite{Chang2020} should they have had a query set of \ac{mimp} sequences more representative of those in \ac{Focub}.

\subsection{Image-based Approaches for Disease Diagnosis in Banana}
 
 Non-invasive technologies, such as \ac{rs}, can help to monitor plant pathogens and, employed alongside molecular diagnostics, can be used for disease diagnosis \parencite{West2010}. As \ac{rs} technologies have advanced and increased in popularity, a wealth of multispectral, hyperspectral, thermal, RGB, and radar imaging data have been generated. These data have been employed in mapping and quantifying pathogens, providing a perspective on pathogen movement at multiple scales, helped to generate epidemiological models, and control disease spread \parencite{Zhang2019}.

Satellites, such as Landsat and Sentinel, which make data freely accessible, combined with aircraft, \acp{uav}, and tractors fitted with \ac{rs} technologies, have collectively and increasingly been employed in plant disease management. Satellites have become popular for mapping disease spread at a large spatial scale. Accurate detection of disease from satellites requires not only a high enough resolution, but also favourable weather conditions and can be influenced by the satellite revisit period \parencite{Zhang2020}. Other technologies, such as \acp{uav} and tractors, have enabled field and sub-field scale detection of diseases in a variety of crops, and have become a key component of precision agriculture. Again, these technologies are affected by resolution and weather conditions, however, unlike satellites, the timing of application and distances from the crop (affecting resolution) can be easily manipulated.  

\Ac{rs} technologies are used to observe changes in the optical properties of leaves or crop canopies. Disease symptoms are typically identified in the visible (VIS = 400\textendash700nm), near-infrared (NIR = 700\textendash1,200nm), and shortwave infrared (SWIR = 1,200\textendash2,400nm) spectral bands, using either passive sensors (i.e. sensors measuring the reflectance of solar radiation) or active sensors (i.e. systems equipped with an own source of radiation which their sensors record) \parencite{Mahlein2016}.  

In healthy plants, due to the absorption of blue, yellow, and red bands by photoactive pigments (e.g. anthocyanins, carotenoids, chlorophylls), leaves typically exhibit low reflectance as VIS wavelengths. Similarly, water and leaf chemical components (e.g. proteins, lignin, cellulose) absorb SWIR resulting in low reflectance at the SWIR range \parencite{West2010}. Conversely, healthy plants exhibit high reflectance at the NIR range, as cell arrangement in leaf mesophyll tissue causes multiple scattering of NIR spectral bands \parencite{Ollinger2011}. In diseased plants, divergence from these optical properties can be observed. As a consequence of photochemical pigment degradation/destruction, necrotic or chlorotic lesions on appear on leaves, which, alongside pathogen propagules, cause increased reflectance in the VIS range. Similarly, increased reflectance in the SWIR range can be observed where pathogenesis affects water content, senescence and reduced growth, while defoliation drives decreased reflectance in the NIR range.   

\subsubsection{Remote Sensing for Image-based Identification of Disease in Banana}

Various studies have been published employing \ac{rs} for disease diagnosis in banana. \textcite{Johansen2014} developed an automatic banana identification software to aid in Banana Bunch Top Virus inspection and \textcite{Liao2018} employed hyperspectral images and machine learning to diagnose Banana Streak Virus at earlier stages of infection. \textcite{Ochoa2016} designed a hyperspectral imaging system for disease scanning on banana plants focusing on Black Sigatoka. \acp{uav}, RGB imaging, and Artificial Neural Networks have also been employed in the monitoring of Yellow Sigatoka \parencite{Calou2020}.  

Recently, \ac{rs} disease diagnosis efforts in banana have focused on the identification of \ac{Focub} infection \parencite{Ye2020a, Ye2020b, Selvaraj2019}.  Ye et al (2020a,b) demonstrated that \ac{uav}-based multispectral imagery can be used to diagnose \ac{fwb}. The authors observed statistically significant differences (p > 0.05) between healthy and diseased plants using six different \acp{vi}. Using logistic regression models to describe the relationship between the \acp{vi} in healthy or diseased plants, the authors reported that the red-edge chlorophyll index had the best performance for identifying \ac{fwb}.  

It is clear that spectral reflectance can be used to diagnose \ac{fwb} in banana. What is not clear, however, is what is happening from a biological perspective to cause the differences in spectra observed, and whether the changes in spectra can be distinguished from other biotic and abiotic stresses. For instance, as \ac{Focub} is a vascular pathogen, it is important to ensure that \ac{Focub} infection can be differentiated from drought stress and other vasucular wilt pathogens like \acf{xvm}. 

\textcite{Marin2020} assessed the efficacy of \ac{rs} to distinguish water-stressed tomato plants from those infected with \ac{Fo} Fo5 strain isolated from \textit{Passiflora edulis} (passionfruit). The study showed that diseased plants could be distinguished from those subjected to hydric stress at 448–523nm, 624–696nm, 740–960nm, 973–976nm, and 992–995nm. However, these results were derived from multivariant analysis, whereby reflectance was correlated with the net assimilation rate of CO\textsubscript{2}, intercellular CO\textsubscript{2} concentration, stomatal conductance, transpiration rate, quantitative yield of \ac{ps2}, and continuous fluorescence. There is little indication that a single method of physiological stress measurement and its relationship to reflectance could be used to distinguish water stress from disease symptoms. One has to question how viable such multivariant analysis is as a means of diagnosis in the field. Is it practical or cost-efficient to measure these photosynthetic parameters in-field and correlate this with spectral reflectance to determine if plants are infected, or would alternative diagnostics be more effective and efficient?   

\subsection{Exploring Disease Biomarkers: Untargeted Metabolomics in Plant Systems}

Metabolomics, another vital "omics" field used in biological sciences, is the comprehensive analysis of the spectrum of metabolites within a biological system \parencite{Klassen2017}. Frequently used to track disease progression and pinpoint potential novel metabolites associated with disease and disease susceptibility, metabolomics has become a useful tool in disease diagnostics. As metabolites mirror downstream expression of the genome, transcriptome, and proteome, they offer an intimate snapshot of an organism's phenotype at a specific stage of infection and scrutinising metabolic variations in infected, resistant, and healthy individuals can unearth disease, susceptibility, or resistance biomarkers. Furthermore, biomarkers may be used for the early identification of diseases, potentially underpinning a pivotal disease management strategy. 

Metabolomics studies are typically classified into two main categories: targeted and untargeted analyses. Targeted metabolomics (TM) approaches identify a specifically selected set of compounds for analysis. This approach is usually employed when precise quantitative analysis is necessary, often requiring complex extraction protocols. When employing an \ac{um} approach, researchers aim to identify a wide range of features without predefined targets. This facilitates the discovery of new and diverse compounds, including previously unknown compounds and metabolites. \ac{um} typically employs simpler extraction and detection procedures compared to targeted studies. However, it generates highly intricate data, which comes with an increased challenge of false discoveries. Consequently, \ac{um} demands substantial effort in data analysis and interpretation.

Now emerging as a useful tool in plant pathology, \ac{um} has been employed in the study of infection as well as plant resistance and susceptibility \parencite{Allwood2021}. Notably, \textcite{Garcia2018} harnessed \ac{lcms} to pinpoint biomarkers indicative of \textit{Phythopthora infestans} infection in tomato plants, even in asymptomatic cases. Similarly, \textcite{Sambles2017} and \textcite{Sidda2020} used \ac{um} to identify certain secoiridoid glycosides as discriminatory metabolites signifying susceptibility to ash die-back in both UK and Danish ash trees (\textit{Fraxinus excelsior}).

\textcite{Kasote2020} employed a compartmentalised TM and\ac{um} strategy to differentiate watermelon (\textit{Citrullus vulgaris}) plants exhibiting symptoms of \textit{Fusarium} wilt, caused by \ac{Fon}), from their asymptomatic counterparts, while also distinguishing them from healthy plants of distinct varieties. Using this approach, the authors were able to identify biomarkers associated with the progression of Fusarium wilt across various watermelon varieties. They showed that the metabolic profiles of \textit{Fusarium} wilt-infected plants exhibit distinct variations depending on their genotype, as well as differences between leaf and stem tissues compared to the root. Phytohormones such as jasmonic acid-isoleucine (JA-Ile) and methyl jasmonate (MeJA) accumulated in resistant varieties, whereas indole-3-acetic acid (IAA) was identified in all resistant lines 16 days after inoculation. The authors suggest that IAA can be used as a potential biomarker of \ac{Fon} infection in watermelon. However, as IAA, as well as the other phytohormones identified,  are commonly associated with immune responses, one has to question how effective they will be as a \ac{Fon} specific biomarker. The varying levels of amino acids (Arg, Asp, Cit, His, Val, and Lys) and the phenolic acid, Phthalic acid (PHA) also offer valuable insights into the interaction between \ac{Fon} and watermelon plants. However, it is imperative to conduct further comprehensive investigations to determine the precise roles of these distinctive metabolites in the development of \textit{Fusarium} wilt and their contribution and specificty to watermelon plant resistance against \ac{Fon}.

Most research into the metabolic composition of bananas primarily focuses on the fruit and its relationship with diet. Only a limited number of studies have delved into the metabolic profiles of bananas affected by \ac{Focub}. In a study conducted by \textcite{Li2013c}, the virulence of various \ac{Focub} isolates was assessed, and LC/MS/MS analysis was used to quantify the presence of two mycotoxins commonly associated with \ac{Focub}, namely beauvericin and fusaric acid, in different parts of banana plants such as roots, fruits, pseudostems, and foliage. Their findings revealed a strong correlation between virulence and the accumulation of these toxins. Additionally, they investigated the occurrence of these toxins in field-grown plants displaying symptoms of \ac{Focub} infection and found that, while the toxins were present in the fruit, their levels were too low to pose any significant risk to human or animal health.

To the best of our knowledge, there have been no studies to date that have employed \ac{um} to comprehensively analyse the metabolic profile of \ac{Focub}-infected banana plants. One objective of this thesis is to compare the data generated from metabolomics studies with imaging data, to shed light on the biological processes contributing to the observed differences during imaging. This approach also aimed to identify specific markers associated with each stress condition. 

\newpage
\section{Project Aims and Objectives}

The overall aim of this research is to develop multi-omic tools for the detection of \acl{Focub} \ac{tr4} in banana and contribute towards the understanding of \acl{Focub} classification, identification, infection and banana responses. 

\begin{enumerate}
    \item Build tools which aid in the identification of targets for molecular diagnostics and contribute to our understanding of \ac{Focub}-banana interactions as well as the evolutionary history and classification of \ac{Focub}. 
    \item Develop molecular diagnostics which can be used to identify specific races of \acl{Fo}. \footnote{Due to challenges sourcing isolates and sequence information from collaborators in Tamil Nadu Agricultural University, India (TNAU), our effector identification tool cannot currently be validated using \ac{Focub}. However, we are testing this tool using other \acl{Fo} \ac{fsp} with collaborators at NIAB and the University of Warwick.}
    \item Employ phenomics to investigate banana responses to \ac{Focub} infection and contribute to the development of image-based diagnostic techniques.  
    \item Develop novel -omics approaches to improve understanding of banana interactions with \ac{Focub}.  
\end{enumerate}

\newpage
\section{Thesis Structure}

\subsection{Chapter 2: Genomics: Potential Novel \textit{Fusarium} Pathogen of Banana Identified} 

This chapter reviews the  genome assembly and subsequent analysis performed on \textit{Fusarium} isolates collected by collaborators at Tamil Nadu Agricultural University in India. Phylogeny of isolates and geographic distribution are discussed and compared to publicly available genomes, and a potential novel species is presented. 

\subsection{Chapter 3: Effectoromics: Improving Genomic Tools for Identifying Virulence Factors in \textit{Fusarium oxysporum}} 

This chapter charts the development of the \textit{mimp}-Associated Effector Identification (Maei) tool which identifies putative effectors in \acl{Fo}, and the implications effector profiles have on the classification and genome compartmentalisation of the FOSC. It includes new genome sequences from collaborators at the University of Warwick and NIAB. Effector presence, location, predicted structures, homology, expression, and potential function are explored. 
These candidate effectors are used in the development of molecular diagnostics, and the potential of the Maei tool in the development of \acl{Fo} \ac{fsp} race-specific diagnostics is explored. 

\subsection{Chapter 4: Phenomics: Image-based Analysis of Banana Stresses}

Banana disease imaging was conducted. This chapter discusses and explores the technologies currently available for image-based disease detection in banana, how spectral data can distinguish stresses, and why different spectral patterns are observed. 
It also explores X-ray computerised tomography data, recording the internal development of symptoms and how this relates to multispectral data.

\subsection{Chapter 5: Metabolomics: The Banana-pathogen Metabolome}
Untargeted Metabolomics analysis was performed on infected banana to help develop our understanding of the \acl{Focub} infection process and indicate how this may be related to patterns observed in image-based analysis. We also identify specific markers of infection from a variety of biotic and abiotic stresses. 

\subsection{Chapter 6: General Discussion}
This chapter discusses the main results from the thesis and draws overall conclusions. It also considers areas for improvement and future work. 


 

 



 



 



 