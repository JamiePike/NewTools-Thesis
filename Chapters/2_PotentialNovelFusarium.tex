%%%%%%%%%%%%%%%%%%%%%%%%%%%%%%%%%%%%%%%%%%%%%
%INTRODUCTION
%%%%%%%%%%%%%%%%%%%%%%%%%%%%%%%%%%%%%%%%%%%%%
% Strcture
%----------
% - Pathogens spread 
% - Distribution of Foc is prime example 
% - TR4 is a strong reason to monitor the distribution and genetic makeup of Fwb pathogens.
% - Lack of data from world's largest producer and climate change is a threat. 
% - We're sequencing isolates...
%%%%%%%%%%%%%%%%%%%%%%%%%%%%%%%%%%%%%%%%%%%%%

\section{Introduction}
\label{sec:chap2Intro}

The history of \acf{Focub} exemplifies the profound impact pathogen spread and diversification can have on global crop production and economies. Although first reported in Australia in 1876 by \textcite{Bancroft1876}, current evidence suggests that the center of origin for \ac{Focub} is Southeast Asia, where it co-evolved with its \textit{Musa} host \parencite{Maryani2019}. Global banana cultivation inadvertently let to the introduction of \ac{Focub} to new regions, leading to severe damage to banana crops \parencite{Kema2021}. 

Notable early dispersion of \acs{Focub} \acf{r1} occurred in South America. The first reports in the region came from Panama and Costa Rica in the 1890s, where \ac{Focub1} acquired the common name, Panama disease \parencite{Ashby1913}. From there, \ac{Focub1}  spread across 'Gros Michel' plantations in Jamaica (1903), Suriname (1906), Trinidad and Tobago (1907), Cuba (1908), Guatemala (1910), Colombia (1929), and Venezuela (1930). By the mid-20th century, the banana export industry had been decimated by the \ac{fwb} epidemic. \Ac{Focub1} is now widely distributed in banana-producing countries, where it still affects many local varieties \parencite{Dita2018}.

Current global production of banana is dominated by the Cavendish variety, which takes its name from William Spencer Cavendish, the sixth Duke of Devonshire. Cavendish plants were originally brought to the duke's residence, Chatsworth House, Derbyshire, UK (1829) after their discovery in Southern China in 1826 by British colonial botanists. Shoots from Chatsworth's Cavendish were distributed to British colonies across the globe and resistance to \ac{Focub1} was eventually identified. By the 1950s the banana export industry had substituted the 'Gros Michel' variety for the Cavendish subgroup \parencite{Ploetz2005, Dita2018}. However, in the following decades, Cavendish banana plants in Taiwan displayed symptoms of \ac{Focub} infection \parencite{Agrios2005}. By the 1990s, the same strain of \ac{Focub} was devastating Cavendish plantations in Western Indonesia and the Malaysian peninsular \parencite{Kema2021}. 

Identified in 1994 as \ac{Focub} \acf{tr4} \parencite{Ploetz1994}, the race affecting Cavendish bananas has continued to spread. \Ac{Focub4} arrived in China from Taiwan in 1996, later expanding to Laos, Myanmar, and Vietnam. A first report was also recorded for Australia in 1997 \parencite{Ploetz2015a}. In 2005, \ac{Focub4} was recorded in the Philippines, rapidly spreading in Mindanao \parencite{Molina2009}. In the past 15 years, first reports of \ac{Focub4} have been published frequently, with occurrences in Pakistan (2012), Lebanon (2013), Oman (2013), Jordan (2013), Mozambique (2013), India (2015), Israel (2016), Colombia (2019), Peru (2021), and Venezuela (2023) (Figure \ref{fig:FocDis}) \parencite{Butler2013, Ploetz2015a, Ordonez2015b, Zheng2018, Thangavelu2019, Garcia-Bastida2020, Maymon2020, Kema2021, Acuna2022,  Herrera2023}.  The rapid expansion of \ac{Focub} poses a significant threat to global banana production, particularly in Asia, which serves as the centre of diversity for \ac{Focub}. Accurate monitoring of pathogen distribution and genetic diversity across the region is essential.  

\begin{figure}[h!]
  \includegraphics[width=14.5cm]{Figures/FocDis.pdf}
  \caption[Global Distribution of \textit{Fusarium oxysporum} f. sp. \textit{cubense}]{\textbf{Global Distribution of \textit{Fusarium oxysporum} f. sp. \textit{cubense}}. Presence or absence shown by country. Countries which have not reported \ac{Focub} are shown in grey. R1: Race 1, R2: Race 2, TR4: Tropical race 4.}
  \label{fig:FocDis}
\end{figure}

\vbox{
Currently, reports of \ac{Focub4} in India, the top banana-producing country globally, are limited. As the Cavendish cultivar 'Grand Naine' (AAA) accounts for approximately 70\% of India's banana production, \ac{Focub4} is of major concern \parencite{Damodaran2019}.  In 2017, \ac{r1} and R2 (VCG 0124/5 complex) were the dominant races found in India, with reports in the southern states of Andhra Pradesh, Karanata, Kerala, and Tamil Nadu, as well as, Gujarat in the east, and Assam, Nagaland, Uttar Pradesh, and West Bengal in the north-west \parencite{Mostert2017, Thangavelu2020}. Concerningly, in 2019, \textcite{Thangavelu2020} reported some \ac{Focub1} isolates (VCGs 0125 and 01220) caused \ac{fwb} symptoms on the Cavendish cultivar 'Grand Naine' in Bihar, Uttar Pradesh, Gujarat, and Tamil Nadu. }

\Ac{Focub4} was first reported in Barari in the state of Bihar in 2015, and has continued to spread; since reported in Mansahi, Kursela, Falka, Korha, and Pothia in the Katihar district and in Dhamdhaha and Rupoli in the Purnia district \parencite{Thangavelu2019}. \textcite{Viljoen2020} warned that \ac{Focub4} was likely to spread from Bihar to Madhya Pradesh, Maharashtra, and Uttar Pradesh, as vehicles and labourers frequently move between the states. \ac{Focub4} has now been reported in Kushi Nagar and Ambedkar Nagar in the state of Uttar Pradesh (Figure \ref{fig:FocDisIndia}) \parencite{Damodaran2019, Thangavelu2019}, but limited data and no biological materials have been provided to identify the strain(s) present \parencite{Kema2021}. As \ac{Focub} spreads through India, necessitating containment measures, the demand for rapid, precise diagnostics and monitoring of pathogen genetic diversity becomes increasingly evident. Particularly given reports of \ac{Focub1} strains causing disease in Cavendish varieties \parencite{Thangavelu2020}.

\begin{figure}[hb!]
\centering
  \includegraphics[width=13.5cm]{Figures/FocDis_India.pdf}
  \caption[Distribution of \textit{Fusarium oxysporum} f. sp. \textit{cubense} in India]{\textbf{ Distribution of \textit{Fusarium oxysporum} f. sp. \textit{cubense} in India}. This map shows the presence or absence by states and union territories. Areas which have not reported \ac{Focub} are shown in grey. R1: Race 1, R2: Race 2, TR4: Tropical race 4.}
  \label{fig:FocDisIndia}
\end{figure}

Collaborators at \ac{tnau}, India, assessed \ac{Focub} diversity in Tamil Nadu. Their primary goal was to explore the genetic and molecular underpinnings of banana resistance against \ac{Focub} using advanced next-generation genomics. A survey was conducted in various banana-growing areas of Tamil Nadu to record symptoms of wilt in different banana cultivars. Samples displaying symptoms of \ac{fwb}, including the \ac{Focub4} susceptible Cavendish banana variety 'Grande Naine', were collected from the field, and the pathogen was isolated and cultured in the laboratory. Among the collected samples, isolates S6, S16, S32, and SY-2, were chosen for further analysis. DNA extraction and PCR amplification were performed to confirm the identity of isolates. The isolated strains, S6, S16, and S32, exhibited severe disease symptoms when inoculated into tissue-cultured 'Grand Naine' banana plants and were classified as highly virulent. PCR amplification using \ac{r4} \textit{SIX9}-specific primers from \textcite{Carvalhais2019} confirmed that \ac{tnau} isolates were \ac{Focub} \ac{r4}, capable of infecting Cavendish banana\footnote{Results were reported by collaborators at \ac{tnau}, but data were not shown.}.

We generated genome assemblies for the \textit{Fusarium} isolates  (S6, S16, S32, and SY-2) retrieved from Cavendish banana plants and analysed genetic traits. Taxonomic identity and genetic diversity were assessed and suggest that some of the isolates collected (S16, S32, and SY-2) are members of previously unreported species in this region. Of the isolates sequenced, S6 is similar to \ac{Focub} \ac{r1} isolates but displays a \ac{Focub4} phenotype, while S16, S32, and SY-2 belong to the \acl{Ff} species complex. These classifications were supported by phylogenetic analysis, read mapping, and \ac{sixg} profiling. It is important to note, however, that bacterial contamination was detected in S6 and S32 isolates, warranting future studies to confirm this potentially novel taxon.

%%%%%%%%%%%%%%%%%%%%%%%%%%%%%%%%%%%%%%%%%%%%%
%METHODS
%%%%%%%%%%%%%%%%%%%%%%%%%%%%%%%%%%%%%%%%%%%%%
\newpage
\section{Materials and Methods}


\subsection{Extraction of genomic DNA from \textit{Fusarium} isolates carried out at \acf{tnau}}

Collaborators at \ac{tnau} maintained cultures of \textit{Fusarium} isolates (collected between 2017 and 2021 in Tamil Nadu, India) as \acf{pda} slants before transferring them to \acf{pda} plates and incubating at 28$^{\pm 2\circ}$C for four days. After 48 hours, the inoculum was introduced into 250 ml Erlenmeyer flasks containing 150 ml of \ac{pdb} and incubated at room temperature (28${\pm 2^\circ}$C) for seven days\footnote{Unfortunately, we were unable to obtain shaker model and rpm from collaborators at \acf{tnau}.}. The resulting mycelium was harvested via filtration through sterile filter paper, immediately frozen at -80$^\circ$C, and stored until DNA extraction. 

Collaborators harvested approximately 1 g of frozen mycelium and ground finely using liquid \ch{N2}. They then mixed with 5 ml of 2\% CTAB extraction buffer (containing 10 mM tris base, pH 8.0, 20 mM EDTA, pH 8.0, 1.4 M NaCl, 2\% CTAB, 0.1\% mercaptoethanol, and 0.2\% PVP) before being incubated at 65$^\circ$C for one hour. An equal volume of phenol-chloroform-isoamyl alcohol (25:24:1) mixture was added to the suspension, thoroughly mixed by vortexing, and then centrifuged at 12,000rpm at 4$^\circ$C for 5 minutes. The upper aqueous layer containing DNA (300\(\mu\)L) was carefully pipetted out, combined with 0.5 volumes of 5M NaCl and an equal volume of ice-cold isopropanol, and incubated at -20$^\circ$C for DNA precipitation for 12 hours. The precipitate was collected by centrifugation at 13,000rpm at 4$^\circ$C for 10 minutes, washed with 0.1M ammonium acetate in 70\% ethanol, and air-dried. The air-dried pellet was suspended in TE buffer (10 mM Tris, 1 mM EDTA, pH 8.0), and they determined DNA concentration spectrophotometrically.

\subsection{Library kit and protocol used for sequencing \textit{Fusarium} isolates at \ac{tnau}}

High-quality genomic DNA was prepared using TruSeq DNA Library Preparation Kit (Illumina Inc. San Diego, California, Cat FC-121-2003) for genome sequencing by collaborators at \ac{tnau}. Genomic DNA was fragmented and obtained fragments were end-repaired, A-tailed, size selected and ligated with Illumina sequencing barcode adapters. The qualified DNA libraries were pooled, and paired-end whole genome sequencing (PE 2 x 150 bp) was carried out using Illumina NovaSeq 6000 platform (M/s. Oneomics Private Limited, Trichy, India). 

\subsection{Analysis of Raw Read data from TNAU isolates}

The raw Illumina paired-end genomic reads for the \textit{Fusarium} isolates SY-2, S6, S16, and S32 were supplied by collaborators at \ac{tnau}. Isolates S6, S16 and S32 were reported to be highly virulent on Cavendish banana by collaborators (personal communication). Following FastQC (v0.11.8)
\parencite{Andrews2010} analysis, the raw reads were mapped to \ac{Focub4} UK0001 (\href{https://www.ncbi.nlm.nih.gov/datasets/genome/GCA_007994515.1/}{GCA\_007994515.1}) \parencite{Warmington2019} reference genome using Bowtie2 (v2.4.5) \parencite{Langmead2012} (for full command-line arguments and shells scripts see the \href{https://github.com/JamiePike/NewTools-Project/blob/master/docs/Assembly/AssemblyNotes.md}{GitHub Repository} referenced in section \ref{sec:chap2dataavail}). Due to poor mapping rates for some isolates, 1000 raw reads were examined from each \ac{tnau} isolate in an attempt to diagnose the issue; they were used as queries in  \ac{ncbi} \href{https://blast.ncbi.nlm.nih.gov/Blast.cgi?PROGRAM=blastn&BLAST_SPEC=GeoBlast&PAGE_TYPE=BlastSearch}{BLASTN searches} against the nr/nt database \parencite{Nih2014}. Raw reads from each isolate were also mapped to high-quality assemblies of the \ac{Focub1} isolate 160527 (\href{https://www.ncbi.nlm.nih.gov/datasets/genome/GCA_005930515.1/}{GCA\_005930515.1}) \parencite{Asai2019} and the \acl{Fs} isolate FS66 (\href{https://www.ncbi.nlm.nih.gov/datasets/genome/GCA_017165645.1/}{GCA\_017165645.1}), reported to cause leaf blight in Cavendish banana \parencite{Cui2021}. Isolates S6, S16 and S32 were also mapped to a reference quality \textit{Stenotrophomonas maltophilia} genome assembly, isolate NCTC10258 (\href{https://www.ncbi.nlm.nih.gov/datasets/genome/GCF_900475405.1/}{GCA\_900475405.1}). \textit{S. maltophilia} is a bacterial species frequently isolated from the environment and is not known to cause disease in banana and is a likely contaminant \parencite{said2021stenotrophomonas}.

\subsection{Genome Assembly and Annotation}

\subsubsection{\textit{De novo Assembly}}
A \textit{de novo} assembly was generated for each of the \ac{tnau} isolates using SPAdes (v3.14.1) \parencite{Prjibelski2020} following FastQC (v0.11.8) analysis \parencite{Andrews2010}. SPAdes is routinely used for \textit{Fusarium} genome assembly
(see: \textcite{Armitage2018, Hudson2020, Tanaka2022}). A custom Python script was developed to remove sequences with <25\% GC content from the assembly (see section \ref{sec:chap2dataavail}  \href{https://github.com/JamiePike/NewTools-Project/blob/master/bin/gcTrimmer.py}{GitHub}). As raw read data for isolates S6 and S32 contained high levels of contamination, \textit{de novo} assemblies were generated using only reads mapping to reference genomes \ac{Focub1} isolate 160527 (\href{https://www.ncbi.nlm.nih.gov/datasets/genome/GCA_005930515.1/}{GCA\_005930515.1}) \parencite{Asai2019} and \acl{Fs} isolate FS66 (\href{https://www.ncbi.nlm.nih.gov/datasets/genome/GCA_017165645.1/}{GCA\_017165645.1}) \parencite{Cui2021}, respectively. Mapped reads were determined using Bowtie2 (v2.4.5) and were extracted into separate FASTQ files using SAMtools (v1.6, using htslib v1.6) \parencite{Danecek2021}. 

\subsubsection{Contamination Analysis and Quality Assessments}
BlobTools (v1.1.1) \parencite{Laetsch2017} was used for the taxonomic partitioning of the assemblies. The \ac{tnau} assemblies were searched against the \ac{ncbi} nt/nr database using BLASTN (v2.9.0+) and the paired-end raw reads were aligned to each of the assemblies using Bowtie2 (v2.4.5). Taxonomic hits were ranked at the species level using default BlobTools (v1.1.1) settings. To extract contigs that were assigned to a non-\textit{Fusarium} genus by BlobTools (v1.1.1), the output JSON file was filtered using blobtools view, with the -r species flag. Contigs assigned to a \textit{Fusarium} species or "no-hit" were then extracted and saved in a separate text file. The BlobTools (v1.1.1) seqfilter command with the default settings was then used to extract contigs which were assigned to \textit{Fusarium} or had "no-hit".

The quality and completeness of the assembled, contaminant-filtered genomes was estimated using \ac{busco} (v5.4.6) with the hypocreales\_odb10 data set \parencite{Manni2021}. All of the raw reads were mapped back to the assemblies using Bowtie2 (v2.4.5) and a Qualimap (v2.2.2) assessment was conducted \parencite{Garcia-Alcalde2012}. Quast (v5.0.2) \parencite{Gurevich2013}  and gfastats (v1.3.6) \parencite{Formenti2022} were used to generate assembly quality statistics. 

\subsubsection{Identification of Repetitive Elements}

Repetitive elements were identified in each contaminant-filtered assembly using RepeatModeler (v2.0.4) \parencite{Flynn2020} including the -LTRStruct option. Once models had been generated, assemblies were masked iteratively with RepeatMasker (v4.1.5) \parencite{Smit2010} using the output from each round as input for the following masking step. First small, simple repeats were masked using the -noint and -xsmall options; then assemblies were masked using the default RepeatMasker (v4.1.5)  database (Dfam v3.3) \parencite{Storer2021} using the -species Fusarium option; followed by masking using the models generated by RepeatModeler (v2.0.4).

\subsubsection{Genome Annotation}
Following masking, the \ac{tnau} genomes were annotated using the MAKER pipeline (v3.01.04) \parencite{Holt2011}, including AUGUSTUS (v3.3) \parencite{Stanke2006} for \textit{ab initio} gene prediction with the “Fusarium” species option. As no RNA sequencing data were available for these isolates, homology evidence from a reference set of 387,728 proteins, from the \ac{ncbi} RefSeq nr database \parencite{Agarwala2016}, generated using the search term "Fusarium AND srcdb\_refseq[PROP]", was used. 

\subsection{Phylogenetic Analysis of \ac{tnau} isolates}\label{chap2:phylogeny}

The common \textit{Fusarium} genetic barcodes \acf{tef} and \acf{rbp2} were used to generate phylogenies \parencite{Edel-Hermann2019}. A \ac{tef} and \ac{rbp2} sequence databases were compiled using available reference sequences from the \ac{ncbi} database (see \href{https://github.com/JamiePike/NewTools-Project/tree/master/data/phylogenies}{https://github.com/JamiePike/NewTools-Project/tree/m\-aster/data/phylogenies}). Homologues of \ac{tef} and \ac{rbp2} were identified in each of our \ac{Fo} assemblies (See section:~\ref{chap3:fusariumdb}) using BLASTN (v 2.9.0+)(1e\textsuperscript{-6} cut-off). The locations of top hits were recorded, and the sequence within this region was manually examined and extracted using SAMtools (v1.15.1). The \ac{tef} and \ac{rbp2} regions from each genome were combined into a multiFASTA file for each barcode. MAFFT (v7.505) \parencite{Katoh2019} was used to construct a multiple sequence alignment, adjusting the direction according to the first sequence to ensure correct alignment and any overhanging regions were trimmed manually. IQ-TREE2 (v2.2.0.3) \parencite{Nguyen2015} was used to infer a maximum-likelihood phylogeny using the ultrafast bootstrap setting for 1000 bootstrap replicates and was visualised using iTOL \parencite{Letunic2021}. Additionally, the extracted \ac{tef} and \ac{rbp2} regions for the \ac{tnau} isolates were searched against the \ac{ncbi} nr/nt database using the  \ac{ncbi} \href{https://blast.ncbi.nlm.nih.gov/Blast.cgi?PROGRAM=blastn&BLAST_SPEC=GeoBlast&PAGE_TYPE=BlastSearch}{BLASTN suite} \parencite{Nih2014}, and the \href{https://fusarium.mycobank.org/page/Fusarium_table}{MycoBank database} \parencite{Robert2013}. 

\subsection{Identification of pathogen-specific features.}

\subsubsection{\textit{Secreted In Xylem} gene identification and phylogenetic analysis}
\label{chap1:tnauSIXgenePhylo}
To identify known \acs{Fo} effectors in the \ac{tnau} assemblies and compare the \ac{sixg} profile to other \textit{Fusaria}, reference sequences for SIX1-SIX14 used by \textcite{Czislowski2018} were downloaded from GenBank (Appendix A\ref{apx:sixgenerefs}) and homologues of each \textit{SIX} gene was identified using TBLASTN (v2.9.0+) (1e\textsuperscript{6} cut-off) in the \ac{tnau} assemblies, as well as publicly available assemblies of \ac{Focub} (n=14), \ac{Fs} (n=2), the \ac{Fo} endophyte Fo47, \ac{Fo} formae speciales \textit{cepae, conglutinans, coriandrii, lini, lycopersici,  rapae} and \textit{vasinfectum}, and \textit{F. graminearum}, which were downloaded from \href{https://www.ncbi.nlm.nih.gov/data-hub/genome/}{GenBank} following a genome search. (Table: \ref{tab:GenomeDB}). A binary data matrix indicating presence (“1”) or absence (“0”) was generated using  the TBLASTN hit data for visualisation. 

Homologues of each \ac{sixg} identified were extracted for phylogenetic analysis.  The locations of hits were recorded, and the sequence within this region was manually examined and extracted using SAMtools (v1.15.1). Sequences from each genome were added to a multiFASTA file for each \ac{sixg} and aligned with MAFFT (v7.505) \parencite{Katoh2019}, adjusting the direction according to the first sequence to ensure correct alignment. Overhanging regions were trimmed manually. IQ-TREE2 (v2.2.0.3) \parencite{Nguyen2015} was used to infer a maximum-likelihood phylogeny using the ultrafast bootstrap setting for 1000 bootstrap replicates and resulting trees were visualised using the R (v4.3.1) \parencite{R} package, ggtree (v3.10.0) \parencite{ggtree}.
\subsubsection{Identification of putative effectors}

To identify \acfp{ce} in the \ac{tnau} isolates, the predicted protein set for each isolate was filtered by size (>30aa and <450aa) and submitted to SignalP (v5.0b) \parencite{Petersen2011}. Sequences that were predicted to contain a signal peptide were passed to EffectorP (v2.0.1) \parencite{Sperschneider2018} for effector prediction. The \ac{ce} sets from each genome were then combined and clustered using CD-HIT (v4.8.1) \parencite{Fu2012} at 80\% to identify shared \acp{ce}. This approach was chosen as it adheres to the arguments for effector prediction made by \textcite{Sperschneider2015, Todd2022} (see section \ref{sec:chap2Intro}). 

%\subsubsection{Reference genome alignment and raw read mapping}

%Pathogenicity chromosomes are common among plant pathogenic \ac{Fo} \parencite{Ma2010, Fokkens2020}. As the \ac{tnau} isolate assemblies were too fragmented to characterise  pathogenic regions, \acp{mimp}, \acp{sixg}, and \acp{ce} identified using the Maei pipeline (Chapter \ref{Chap3}) were recorded in the high-quality \ac{Focub4} UK0001 and \ac{Fs} FS66 assemblies, and \ac{tnau} isolate raw reads mapped on to these assemblies using Bowtie2 (v2.4.5). Nucmer (-max match, deltafilter –g) from MUMmer (v4.0.0rc1) \parencite{Marcais2018} was used to align the ac{Focub4} UK0001 and \ac{Fs} FS66 assemblies to identify shared putative pathogenic regions between the assemblies. Circos (v0.69-8) \parencite{Krzywinski2009} was used to visualise virulence factor location and genome alignments. 

\subsection{Data and software availability}
\label{sec:chap2dataavail}

The full computational pipelines, command-line arguments as well as bash, R, and Python scripts used for all analysis outlined are available in the \href{https://github.com/JamiePike/NewTools-Project}{NewTools-Project} GitHub Repository at https://github.com/JamiePike/NewTools-Project, with supporting documentation provided in Markdown format for \href{https://github.com/JamiePike/NewTools-Project/blob/master/docs/Assembly/AssemblyNotes.md}{genome assemblies}, \href{https://github.com/JamiePike/NewTools-Project/blob/master/docs/Annotations/RepeatMaskingNotes.md}{repeat element identification and masking}, \href{https://github.com/JamiePike/NewTools-Project/blob/master/docs/Annotations/Annotations.md}{annotations}, \href{https://github.com/JamiePike/NewTools-Project/blob/master/docs/Effectors/PredicitionofCandidateEffectors.md}{\ac{ce} identification}, and \href{https://github.com/JamiePike/NewTools-Project/blob/master/docs/Phylogeny/Phylogenies.md}{phylogenetic analysis}.

%%%%%%%%%%%%%%%%%%%%%%%%%%%%%%%%%%%%%%%%%%%%%
%RESULTS
%%%%%%%%%%%%%%%%%%%%%%%%%%%%%%%%%%%%%%%%%%%%%
\newpage
\section{Results}

\subsection{Analysis of Raw Read data from TNAU isolates}
\label{sec:chap2RawReadMapping}

Mapping rates of the raw reads from isolates S6, S16, S32, and SY-2 to the \ac{Focub4} UK0001 assembly (\href{https://www.ncbi.nlm.nih.gov/datasets/genome/GCA_007994515.1/}{GCA\_007994515.1}) were 8.72\%, 53.81\%, 15.69\%, and a 54.11\%, respectively (Table ~\ref{tab:RawReadMapping}). Raw reads from each isolate were also mapped to the \ac{Focub1} 160527 assembly (\href{https://www.ncbi.nlm.nih.gov/datasets/genome/GCA_005930515.1/}{GCA\_005930515.1}), with alignment rates for all \textless 55\%. Due to the low alignment rates, unmapped reads were extracted and a random subset of 1000 reads per isolate was searched using \ac{ncbi} \href{https://blast.ncbi.nlm.nih.gov/Blast.cgi?PROGRAM=blastn&BLAST_SPEC=GeoBlast&PAGE_TYPE=BlastSearch}{web BLASTN suite} \parencite{Nih2014}, which revealed that many of the unmapped reads from these isolates were similar ($\geq90\% $ identity) to sequences from species within \ac{FFSC}. Further, isolates S6 and S32 displayed signs of possible bacterial contamination, with raw reads displaying $ \geq90\% $ identity to sequences from \textit{Stenotrophomonas} species, particularly \textit{S. maltophilia}. 

All raw reads from isolates S6, S16, S32, and SY-2 were mapped to a \acf{Fs} reference assembly, isolate FS66 (\href{https://www.ncbi.nlm.nih.gov/datasets/genome/GCA_017165645.1/}{GCA\_017165645.1}). \ac{Fs} is a member of the \acf{FFSC} and FS66 has recently been reported to cause leaf blight symptoms in banana in in Guangdong, China \parencite{Cui2021}. \textcite{Cui2021} observed yellow to brown lesions spreading from the leaf blade edge inward, but no discolouration of vessels in the roots or pseudo-stems of the most severely affected trees, indicating that these isolates likely cause leaf blight rather than \ac{fwb}. Virulence testing further confirmed that FS66 could induce blight on leaves and pseudo-stems, but not wilt, suggesting that FS66 pathogenicity is limited to leaf tissue. However, the symptoms reported by our collaborators at \ac{tnau} for isolates SY-2 and S16 are more consistent with typical \textit{Fusarium} wilt, rather than leaf blight. Unfortunately, we have not been able to access the \ac{tnau} isolates to confirm whether their symptoms align more closely with a wilt or a blight.

Raw reads from isolates S6 and S32 had a 5.24\% and 22.49\% mapping rate to the \ac{Fs} reference, respectively, whereas 68.65\% of the raw reads from isolate S16 and 93.96\% of the raw reads from the SY-2 isolate mapped to the \ac{Fs} reference assembly (Table ~\ref{tab:RawReadMapping}). Isolates S6, S16, and S32 were also mapped to a reference-quality \textit{S. maltophilia} genome assembly (\href{https://www.ncbi.nlm.nih.gov/datasets/genome/GCF_900475405.1/}{GCA\_900475405.1}, isolate NCTC10258). Approximately 50\% of raw reads from isolates S6 and S32 mapped to the \textit{S. maltophilia} reference, whereas raw reads from isolate S16 only had a 0.01\% mapping rate to the \textit{S. maltophilia} reference. Further, the raw S6 and S32 reads show a similar GC\% to the \textit{S. maltophilia} reference genome (S6=63\%, S32=61\%, NCTC10258=66.5\%).

\bigskip
% Please add the following required packages to your document preamble:
% \usepackage{multirow}
% \usepackage{lscape}
% \usepackage{longtable}
% Note: It may be necessary to compile the document several times to get a multi-page table to line up properly

\begin{landscape}
\begingroup
\setlength{\tabcolsep}{6pt} % Default value: 6pt
\renewcommand{\arraystretch}{0.93}
\begin{longtable}[c]{ccccccccc}
\caption[Overall alignment rate of all raw reads from each TNAU isolates to fungal and bacterial reference species]{\textbf{Overall alignment rate of all raw reads from each TNAU isolates to fungal and bacterial reference species.} Overall alignment rate determined by Bowtie2 (version 2.4.5). Reference assemblies were downloaded from GenBank.}
\label{tab:RawReadMapping}\\
\hline
\multirow{\textbf{\begin{tabular}[c]{@{}c@{}}Reference \\ Species\end{tabular}}} &
  \multicolumn{4}{c}{\textbf{\begin{tabular}[c]{@{}c@{}}Isolate Bowtie2 Raw Read \\ Overall Alignment Rate\end{tabular}}} &
  \multirow{\textbf{\begin{tabular}[c]{@{}c@{}}Reference \\ Strain\end{tabular}}} &
  \multirow{\textbf{\begin{tabular}[c]{@{}c@{}}Reference \\ GenBank \\ Accession\end{tabular}}} &
  \multirow{\textbf{\begin{tabular}[c]{@{}c@{}}No. of \\ Contigs\end{tabular}}} &
  \multirow{\textbf{\begin{tabular}[c]{@{}c@{}}Contig \\ N50 \\ (Mb)\end{tabular}}} \\ \cline{2-5}
 &
  \textbf{SY-2} &
  \textbf{S6} &
  \textbf{S16} &
  \textbf{S32} &
   &
   &
   &
   \\ \hline
\endfirsthead
%
\multicolumn{9}{c}%
{{\bfseries Table \thetable\ continued from previous page}} \\
\endhead
%
\multicolumn{1}{c}{\textit{\begin{tabular}[c]{@{}c@{}}Fusarium oxypsorum \\ f. sp. cubense TR4\end{tabular}}} &
  54.11\% &
  8.72\% &
  53.81\% &
  15.69\% &
  UK0001 &
  GCA\_007994515.1 &
  15 &
  4.49 \\
\multicolumn{1}{c}{\textit{\begin{tabular}[c]{@{}c@{}}Fusarium oxysporum \\ f. sp. cubense R1\end{tabular}}} &
  54.02\% &
  H &
  H &
  H &
  160527 &
  GCA\_005930515.1 &
  12 &
  4.88 \\
\multicolumn{1}{c}{\textit{F. sacchari}} &
  93.96\% &
  5.24\% &
  68.65\% &
  22.49\% &
  FS66 &
  GCA\_017165645.1 &
  48 &
  1.97 \\
\multicolumn{1}{c}{\textit{\begin{tabular}[c]{@{}c@{}}Stenotrophomonas \\ maltophilia\end{tabular}}} &
  *\footnote{SY-2 was not assessed for \textit{Stenotrophomonas maltophilia} contamination as it was sent as part of a previous sequencing run. } &
  49.32\% &
  0.01\% &
  53.93\% &
  NCTC10258 &
  GCA\_900475405.1 &
  1 &
  4.5 \\
  \hline
\end{longtable}
\endgroup
\end{landscape}

\subsection{Contamination Analysis and \textit{de novo} Genome Assembly}

As <55\% of raw reads from the \ac{tnau} isolates aligned to the \ac{Focub1} and \ac{Focub4} reference assemblies, the \ac{tnau} genomes were assembled \textit{de novo}. All raw reads were used for the S16 and SY-2 genome assemblies, and due to the level of \textit{S. maltophilia} contamination (Table \ref{tab:RawReadMapping}), S6 and S32 \textit{de novo} genome assemblies were generated using only reads aligning to the reference genomes \ac{Focub1} isolate 160527 (\href{https://www.ncbi.nlm.nih.gov/datasets/genome/GCA_005930515.1/}{GCA\_005930515.1}) \parencite{Asai2019} and \ac{Fs} isolate FS66 (\href{https://www.ncbi.nlm.nih.gov/datasets/genome/GCA_017165645.1/}{GCA\_017165645.1}) \parencite{Cui2021}, respectively. Following assembly, low-GC contigs  (\textless 25\%) were removed from the SY-2 and S16 assemblies (this was not performed for S6 and S32, as assembly GC content followed a normal distribution, see appendix A\ref{fig:tnuaGCplots}), reducing the total number of contigs from 1,548 to 441 in the SY-2 assembly (2.01 Mb) and 1,666 to 768 in the S16 assembly (1.85 Mb).  

A BlobTools (v1.1.1) analysis was used to identify any potential contaminant contigs after removing the low GC contigs. Although \textit{Fusarium} species accounted for the majority of the \ac{ncbi} nt database hits for the SY-2 genome assembly, with 365 contigs assigned to \ac{Ff}, some potential contamination from \textit{Musa} species was identified, with 14 contigs assigned to \textit{Musa balbisiana} and one assigned to \textit{Musa textilis} (Figure \ref{fig:SY-2:BlobTools}). 

This potential \textit{Musa} contamination is unexpected given that gDNA was extracted from mycelial cultures. While the source of this contamination remains unclear, it's important to consider several possibilities: contamination could have occurred during sample preparation in a shared laboratory space. The DNA extractions were performed by collaborators at \ac{tnau}, where the laboratory environment handles banana tissues. Alternatively, the sequences identified as \textit{Musa} might be misidentified fungal sequences, possibly from \textit{Fusarium}. Another possibility is that both the \textit{Fusarium} and \textit{Musa} samples could share contamination from a common source, such as bacteria.

The S6, S16, and S32 genome assemblies contained hits from only \textit{Fusarium} species or had no species allocated (no-hit) and no other genera were identified. For S16 and S32, the majority of contigs were assigned to \acf{Ff} (Figure \ref{fig:S16:BlobTools}, \ref{fig:S32:BlobTools}), whereas the  S6 genome assembly contained contigs assigned to \ac{Ff} and \ac{Fo}, although 2,301 contigs had no hits (Figure \ref{fig:S6:BlobTools}).  

%%%%%%%%%%%%%%%%%%%%%%%%%%
%%%%% SY-2 BlobTools %%%%%
%%%%%%%%%%%%%%%%%%%%%%%%%%
\begin{figure}[hp!]
    \centering
    \begin{subfigure}[]{0.99\textwidth}
        \centering
        \includegraphics[width=\textwidth]{Appendices/SY-2_GCtimmed.blobtools.blobDB.species.plot.SY-2_GCtimmed.blobtools.blobDB.json.bestsum.species.p8.span.100.blobplot.bam0.png}
        \caption{}
        \label{fig:BlobPlot-SY-2}
    \end{subfigure}
    \begin{subfigure}[]{0.9\textwidth}
        \centering
        \includegraphics[width=\textwidth]{Appendices/SY-2_GCtimmed.blobtools.blobDB.species.plot.SY-2_GCtimmed.blobtools.blobDB.json.bestsum.species.p8.span.100.blobplot.read_cov.bam0.png}
        \caption{}
        \label{fig:BlobPlot_readcov-SY-2}
    \end{subfigure}
    \caption[BlobTools visualisations of the SY-2 assembly]{\textbf{BlobTools visualisations of the SY-2 assembly.}
        \textbf{\subref{fig:BlobPlot-SY-2})} BlobPlot of SY-2. Sequences in the assembly are depicted as circles, with diameter proportional to sequence length and coloured by taxonomic annotation based on BLASTN (v2.9.0+) of NCBI nt database.
        \textbf{\subref{fig:BlobPlot_readcov-SY-2})} Read coverage plot of the SY-2 assembly. Mapped reads are shown by taxonomic group at the rank of species.}
        \label{fig:SY-2:BlobTools}
\end{figure}

%%%%%%%%%%%%%%%%%%%%%%%%%%
%%%%% S16 BlobTools %%%%%%
%%%%%%%%%%%%%%%%%%%%%%%%%%
\begin{figure}[hp!]
    \centering
    \begin{subfigure}[]{0.99\textwidth}
        \centering
        \includegraphics[width=\textwidth]{Figures/TNAU_S16.species.blobplot.pdf}
        \caption{}
        \label{fig:BlobPlot-S16}
    \end{subfigure}
    \begin{subfigure}[]{0.9\textwidth}
        \centering
        \includegraphics[width=\textwidth]{Figures/TNAU_S16.blobtools.blobDB.json.bestsum.species.p8.span.100.blobplot.read_cov.bam0.pdf}
        \caption{}
        \label{fig:BlobPlot_readcov-S16}
    \end{subfigure}
    \caption[BlobTools visualisations of the S16 assembly]{\textbf{BlobTools visualisations of the S16 assembly.}
        \subref{fig:BlobPlot-S16})\textbf{ BlobPlot of S16. Sequences in }the assembly are depicted as circles, with diameter proportional to sequence length and coloured by taxonomic annotation based on BLASTN (v2.9.0+) of NCBI nt database.
        \textbf{\subref{fig:BlobPlot_readcov-S16}}) Read coverage plot of the S16 assembly. Mapped reads are shown by taxonomic group at the rank of species.}
        \label{fig:S16:BlobTools}
\end{figure}
%%%%%%%%%%%%%%%%%%%%%%%%%%
%%%%% S32 BlobTools %%%%%%
%%%%%%%%%%%%%%%%%%%%%%%%%%
\begin{figure}[hp!]
    \centering
    \begin{subfigure}[]{0.99\textwidth}
        \centering
        \includegraphics[width=\textwidth]{Appendices/TNAU_S32-FS66ReadsAssembly.blobtools.blobDB.json.bestsum.species.p8.span.100.blobplot.bam0.png}
        \caption{}
        \label{fig:BlobPlot-S32}
    \end{subfigure}
    \begin{subfigure}[]{0.9\textwidth}
        \centering
        \includegraphics[width=\textwidth]{Appendices/TNAU_S32-FS66ReadsAssembly.blobtools.blobDB.json.bestsum.species.p8.span.100.blobplot.read_cov.bam0.png}
        \caption{}
        \label{fig:BlobPlot_readcov-S32}
    \end{subfigure}
    \caption[BlobTools visualisations of the S32 assembly]{\textbf{BlobTools visualisations of the S32 assembly.}
        \textbf{\subref{fig:BlobPlot-S32}) }BlobPlot of S32. Sequences in the assembly are depicted as circles, with diameter proportional to sequence length and coloured by taxonomic annotation based on BLASTN (v2.9.0+) of NCBI nt database.
        \textbf{\subref{fig:BlobPlot_readcov-S32})} Read coverage plot of the S16 assembly. Mapped reads are shown by taxonomic group at the rank of species.}
        \label{fig:S32:BlobTools}
\end{figure}
%%%%%%%%%%%%%%%%%%%%%%%%%%
%%%%%% S6 BlobTools %%%%%%
%%%%%%%%%%%%%%%%%%%%%%%%%%
\begin{figure}[hp!]
    \centering
    \begin{subfigure}[]{0.99\textwidth}
        \centering
        \includegraphics[width=\textwidth]{Appendices/S6_Foc1Assembly.blobtools.blobDB.json.bestsum.species.p8.span.100.blobplot.bam0.png}
        \caption{}
        \label{fig:BlobPlot-S6}
    \end{subfigure}
    \begin{subfigure}[]{0.9\textwidth}
        \centering
        \includegraphics[width=\textwidth]{Appendices/S6_Foc1Assembly.blobtools.blobDB.json.bestsum.species.p8.span.100.blobplot.read_cov.bam0.png}
        \caption{}
        \label{fig:BlobPlot_readcov-S6}
    \end{subfigure}
    \caption[BlobTools visualisations of the S6 assembly]{\textbf{BlobTools visualisations of the S6 assembly.}
        \textbf{\subref{fig:BlobPlot-S6}) }BlobPlot of S6. Sequences in the assembly are depicted as circles, with diameter proportional to sequence length and coloured by taxonomic annotation based on BLASTN (v2.9.0+) of NCBI nt database.
        \textbf{\subref{fig:BlobPlot_readcov-S6})} Read coverage plot of the S6 assembly. Mapped reads are shown by taxonomic group at the rank of species.}
        \label{fig:S6:BlobTools}
\end{figure}

Following BlobTools (v1.1.1) analysis, contaminated contigs (assigned to species other than \textit{Fusarium}) were removed. The  S6, S16, S32, and SY-2 contaminant-filtered genome assemblies contained 6,048, 768, 2,443, and 408 contigs, respectively (Table~\ref{tab:TNAUAssemblyStats}). Coverage of the contaminant filtered assemblies varied from 16x (S6) to 148x (S16), and percentage GC content ranged from 47.53\% (S16) to 48.85\% (S32) (Table~\ref{tab:TNAUAssemblyStats}). All genome assemblies were between 40Mb and 50Mb in length and contained 15,719 to 17,891 predicted protein coding genes. Genome completeness was assessed using \ac{busco} (v5.4.6) with the hypocreales\_odb10 dataset. SY-2, S6, S16, and S32 genome assemblies contained 99.60\%, 97.50\%, 97.40\%, and 97.40\% intact, single-copy orthologs, respectively. 
\bigskip

% Please add the following required packages to your document preamble:
% \usepackage{multirow}
% \usepackage{longtable}
% Note: It may be necessary to compile the document several times to get a multi-page table to line up properly
%\begin{landscape}
\begingroup
\setlength{\tabcolsep}{3pt} % Default value: 6pt
\renewcommand{\arraystretch}{0.7}
\setlength\LTcapwidth{\textwidth} % default: 4in (rather less than \textwidth...)
\setlength\LTleft{0pt}            % default: \parindent
\setlength\LTright{0pt}           % default: \fill
\begin{longtable}[c]{ccccc}
\captionsetup{width=\linewidth} 
\caption[Summary statistics of TNAU genome assemblies.]{\textbf{Summary statistics of TNAU genome assemblies. }\textit{De novo} assemblies generated using SPAdes (version 3.14.1) with all raw reads supplied by Tamil Nadu Agricultural University.}
\label{tab:TNAUAssemblyStats}\\
\hline
\multirow{2}{*}{\textbf{\begin{tabular}[c]{@{}l@{}}Assembly\\Statistic\end{tabular}}} & \multicolumn{4}{c}{\textbf{TNAU Isolate Assembly}} \\ \cline{2-5} 
                        & \textbf{S6} & \textbf{S16} & \textbf{S32} & \textbf{SY-2} \\ \hline
\endfirsthead
%
\multicolumn{5}{c}%
{{\bfseries Table \thetable\ continued from previous page}} \\
\endhead
%
No. PE reads                       & 26,213,258     & 22,473,220     & 34,526,189      & 11,032,845 \\
Number of contigs                  & 6,048          & 768            & 2,443           & 408    \\
Largest contig (Mb)                & 0.43           & 0.88           & 0.77            & 0.87   \\
Total length (Mb)                  & 47.23          & 44.86          & 40.92           & 44.22  \\
GC (\%)                            & 47.87          & 47.53          & 48.85           & 47.98  \\
N50 (bp)                           & 92,555         & 234,991        & 78,523          & 200,307 \\
L50                                & 148            & 60             & 109             & 63     \\
Mapped Reads (\%)                  & 9.9            & 99.53          & 24.01           & 99.6   \\
Mean Coverage                      & 16x            & 148x           & 60x             & 73x    \\ 
BUSCO (\%)                         & 97.5           & 97.4           & 97.4            & 99.6   \\
Bases soft masked(Mb)              & 2.115 (4.48\%) & 1.853 (4.13\%) & 0.612 (1.50\%)  & 1.430 (3.23\%)      \\
Predicted protein coding genes     & 17,891         & 15,727         & 15,824          & 15,719     \\
\hline  
\end{longtable}
\endgroup
%\end{landscape}



\subsubsection{Contamination in the S6 and S32 raw read data}
\label{sec:BlobToolsOfS6S32-allreads}

Since a high proportion of the genomic reads from isolates S6 and S32 originated from the bacterium \textit{S. maltophilia}, BlobTools (v1.1.1) was used to evaluate the taxonomic partitioning of \textit{de novo} assembly. These genome assemblies generated from both unfiltered and unmapped genomic raw reads contained 100,147 and 1,048 contigs, recorded  97.70\% and 97.70\% \ac{busco} intact single-copy orthologs, were 97.64 Mb and 49.62 Mb in length and had a GC content of 46.75\% and 49.80\% for S6 and S32, respectively. The S6 genome assembly generated using all raw reads was much larger than is typical for a \ac{Fo} assembly and was highly fragmented. Both genome assemblies contained a large number of contigs that were either assigned to no genera, or genera other than \textit{Fusarium}. For instance, 93,702 contigs from the S6 genome assembly had no hits and over 2,000 contigs were assigned to species other than \textit{Fusarium}, though 814 and 647 of the contigs were assigned to \ac{Ff} and \ac{Fo}, respectively (Appendix A\ref{fig:S6:BlobToolsAllreads}). The majority of contigs from the S32 \textit{de novo} genome assembly had the greatest sequence similarity to \textit{Fusarium} species, particularly \ac{Ff} (n=241), although 350 contigs had no-hits and 200 contigs were assigned to \textit{Stenotrophomonas} species, as was observed in the \acs{blast}N search of unmapped raw reads (Appendix A\ref{fig:S32:BlobToolsAllreads}). 

The combined results from the mapping and \textit{de novo} assembly approaches provided crucial insights into the nature of the \ac{tnau} isolates. The low mapping rates to \ac{Focub4} and \ac{Focub1} references suggested that these isolates were not closely related to the \ac{Fo}, typically associated with \ac{fwb}, and suggests the isolates from \ac{tnau} are more closely aligned with species within the \ac{FFSC}, such as \ac{Fs}. This was further supported by the de novo assembly results, where \ac{Ff} accounted for most contigs across multiple isolates. However, identifying potential bacterial contamination in isolates S6 and S32, and unexpected \textit{Musa} sequences in the SY-2 assembly, raised additional complexities, highlighting the importance of stringent contamination controls.

\subsection{\Acl{tef} and \acl{rbp2} reveal novel clade of Fusarium pathogenic towards banana in Indian sampling region}
\label{sec:chap2phylogenies}

 Sequences containing the common \textit{Fusarium} barcodes, \acf{tef}  and \acf{rbp2}, were extracted from the \ac{tnau} isolates and used to infer a maximum-likelihood phylogeny. S6 falls within one of the \ac{Focub1} clades in the \acs{tef} and \ac{rbp2} phylogeny (Figure \ref{fig:TEF1aPhylo}, Appendix A\ref{fig:rbp2Phylo}). \Ac{tef} and \ac{rbp2} sequences from the S16 genome assembly sit within the same clade as the SY-2 and reference \ac{Fs} \ac{tef} and \ac{rbp2} sequences which, taken together with the raw read mapping data, suggests these isolates may be strains of \ac{Fs} pathogenic towards banana (Figure \ref{fig:TEF1aPhylo}, Appendix A\ref{fig:rbp2Phylo}). Based on the \ac{tef} and \ac{rbp2} phylogenies, S32 appears to be a sister lineage of \ac{Fs}. S32 groups with the newly described species recovered from symptomatic Cavendish banana in the Philippines, \acf{Fm}, proposed by \textcite{Nozawa2023}  (Figure \ref{fig:TEF1aPhylo}, Appendix A\ref{fig:rbp2Phylo}). These extracted \ac{tef} sequences were also searched against the Fusariod-ID MSLT \parencite{Robert2013} and \ac{ncbi} BLAST databases for similar sequences. A search of the \ac{ncbi} database revealed that the \ac{tef} sequence extracted from the S16 genome assembly best scoring hits were for \ac{Fs} (Table \ref{tab:Tef1-NCBIdb}). Further, the Fusariod MSLT database best scoring hits for the \ac{tef} sequence extracted from the S16 genome assembly were for sequences from the \ac{FFSC}, in which \ac{Fs} can be found (Appendix A\ref{tab:Tef1-MLSTdb}). Searches for the S6 isolate extracted \ac{tef} sequence suggest S6 belongs to the \ac{FOSC}. There were matches for \ac{Focub} \ac{tef} sequences for the S6 \ac{tef} sequence, although these were not in the top 3 results from searches of both databases. No matches were found for the S32 extracted \ac{tef} sequences in the Fusarioid-ID MSLT database, and hits against the \ac{ncbi} GenBank database were for \ac{tef} sequences from \ac{Ff}. 

% Please add the following required packages to your document preamble:
% \usepackage{multirow}
% \usepackage{graphicx}
% \usepackage{lscape}
\begin{landscape}
\begin{table}[]
\centering
\captionsetup{width=\linewidth} 
\caption{[\Ac{tnau}\acf{tef} \acf{ncbi} and Fusariod-ID MSLT database searches.]\textbf{Best hits of extracted \acf{tef} sequences from the \acf{tnau} isolate \textit{de novo} assemblies against the Fusariod-ID MSLT database and \ac{ncbi}}}
\label{tab:Tef1-NCBIdb}
\resizebox{\columnwidth}{!}{%
\begin{tabular}{cccc}
\multicolumn{1}{l}{\multirow{2}{*}{\textbf{TNAU Isolate Assembly}}} & \multicolumn{3}{c}{\textbf{NCBI database}}                                        \\ \cline{2-4} 
\multicolumn{1}{l}{}                                       & Hit 1                    & Hit 2                    & Hit 3                       \\ \hline
\textbf{S6 All reads}                                      & \textit{F. oxysporum} isolate 170 & \textit{F. oxysporum} f. sp. \textit{koae} & \textit{F. oxysporum} f. sp. \textit{dianthi} \\
\textbf{S6 \textit{F. oxysporum}  Reads}                            & \textit{F. oxysporum} isolate 170 & \textit{F. oxysporum} f. sp. \textit{koae} & \textit{F. oxysporum} f. sp. \textit{dianthi} \\
\textbf{S16}                                               & \textit{F. sacchari} CBS:147.25   & \textit{F. sacchari} NRLL 66326  & \textit{F. globosum} CBS:428.97      \\
\textbf{S32 All reads}                                     & \textit{F. fujikuroi} I1.3        & \textit{F. fujikuroi} IMI 58289   & \textit{F. fujikuroi} Augusto2       \\
\textbf{S32 \textit{F. oxysporum}  Reads}                           & \textit{F. fujikuroi} I1.3        & \textit{F. fujikuroi} IMI 58289   & \textit{F. fujikuroi} Augusto2       \\
\textbf{S32 \textit{F. sacchari} reads}                             & \textit{F. fujikuroi} I1.3        & \textit{F. fujikuroi} IMI 58289   & \textit{F. fujikuroi} Augusto2      
\end{tabular}%
}
\end{table}
\end{landscape}

\begin{figure}[htp!]
    \centering
    \includegraphics[width=14cm]{Figures/TEF1-aPhylo3.pdf}
    \caption[\Acl{tef} phylogeny of \acl{tnau} \textit{Fusarium} isolates.]{\textbf{\Acl{tef} phylogeny of \acl{tnau} \textit{Fusarium} isolates.} \Ac{tnau} isolates S16, S32 and SY-2 sit within the \acf{FFSC} clade. The \ac{tef} sequences from \ac{tnau} are shown in red text. The \acf{Fs} clade is shown in pink. \Acf{Focub4} \ac{tef} sequences are in blue and \acf{str4} in brown. \Acf{Focub1} \ac{tef} sequences are shown in green. The \ac{tef} sequences from \acf{Focub} genome assemblies with race not recorded are shown in yellow. Percentages represent values from 1000 bootstrap replicates. The tree is rooted through \textit{Fusarium graminearum} PH-1 \ac{tef}.}
    \label{fig:TEF1aPhylo}
\end{figure}
\bigskip

\subsection{Identification of pathogen-specific features}

\subsubsection{\textit{SIX} gene distribution suggest \ac{tnau} isolates are not \acl{Focub4}}
\label{sec:chap2SixGene}

\Acfp{sixg} are the only family of effectors currently confirmed in \ac{Fo} \parencite{Armitage2018, Czislowski2018}. \Ac{sixg} homologues (\textit{SIX1-SIX14}) were identified in the \ac{tnau} genome assemblies, alongside publicly available genome assemblies of \ac{Focub} (n=14), \ac{Fs} (n=2), \ac{Fo} \acp{fsp} (n=7), an \ac{Fo} endophyte, and a \acl{Fg} genome assembly. No \acp{sixg} were identified in the S32 genome assembly. A \textit{SIX2} homologue was identified in S16, SY-2  and  \ac{Fs} genome assemblies (Figure \ref{fig:SixTNAU}), and no other \acp{sixg} were identified. The S6 genome assembly clustered with other \ac{Focub1} isolates that have \textit{SIX1, SIX4, SIX6}, and \textit{SIX9} homologues identified, although, \textit{SIX13} is not present in the S6 genome assembly. Interestingly, no \acp{sixg} were identified in the publicly available \ac{Focub1} genome assembly, Foc1 60 \parencite{Yun2019}, which clustered with S32, as well as the \ac{Fo} endophyte and \textit{F. graminearum} genome assemblies. 

\begin{figure}[htp!]
  \centering
  \includegraphics[width=15cm]{Figures/SIX_Heatmap.pdf}
  \caption[Binary distribution of \textit{SIX} genes across Fusarium \ac{tnau} genome assemblies]{\textbf{Binary distribution of \textit{SIX} genes across Fusarium \ac{tnau} genome assemblies}. \aclp{sixg} identified using TBLASTN (cut off 1\-e\textsuperscript{6}) using \textit{SIX} genes from \acl{Foly} as a reference. \textit{SIX4} not found in the \acl{Foly} 4287 assembly, as previously reported \parencite{Czislowski2018}}
  \label{fig:SixTNAU}
\end{figure}

As homologues of \textit{SIX1, SIX2, SIX4, SIX6}, and \textit{SIX9} were identified in the \ac{tnau} genome assemblies, these sequences were extracted for phylogenetic analysis.  A \textit{SIX9} homologue was identified in the S6 genome assembly, copies of which were identified in all \ac{Focub} genome assemblies (Figure \ref{fig:FusSIX9}). A further copy of \textit{SIX9} was found in each of the \ac{Focub4} genome assemblies and assemblies for VPRI44081, VPRI44082, and VPRI44083 (race not known). In the \textit{SIX1} phylogeny, the S6 \textit{SIX1} falls within the same clades as the \textit{SIX1} from \ac{Focub1} N2 and 160527 genome assemblies, and the \ac{Focub1} assemblies which have no race data available: VPRI44079, VPRI44082, VPRI44083, VPRI44084 (Figure \ref{fig:FusSIX1}), but the number of copies of \textit{SIX1} identified varies between isolate genome assembly. The same can also be observed for \textit{SIX4} and \textit{SIX6} (Figure \ref{fig:FusSIXMultiPhylo}). \textit{SIX2} was not identified in the S6 assembly or the \ac{Focub1} assemblies. However, it was identified in the S16 and SY-2, \ac{Fs} and other \ac{Focub} genome assemblies. The SY-2 and S16 \textit{SIX2} sequences fall within the \ac{Fs} clade in the \textit{SIX2} phylogeny (Figure \ref{fig:FusSIX2}).

The \ac{sixg} distribution and phylogenies reinforce the findings from raw read mapping, \textit{de novo} assembly, and genetic barcode phylogenies, providing multiple lines of evidence that the \ac{tnau} isolates are not \ac{Focub4}. Unlike \ac{Focub4}, the \ac{tnau} isolates lack the characteristic \ac{sixg} profile shared among \ac{Focub} isolates of the same race, and phylogenic analysis of \acp{sixg} demonstrate that the \ac{sixg} sequences in the \ac{tnau} isolates, particularly \textit{SIX1}, \textit{SIX4}, and \textit{SIX6} in S6, align more closely with those of \ac{Focub1} isolates, while \textit{SIX2} in S16 and SY-2 clusters with \ac{Fs} isolates. This supports their identification as a different \textit{Fusarium} lineage. These integrated approaches improve species identification accuracy and offer insights into potential differences in virulence, aiding in future diagnostic efforts.

\begin{figure}[htp!]
  \centering
  \includegraphics[width=\textwidth]{Figures/FusSIX9.phylo.png}
  \caption[Maximum likelihood phylogeny from alignment of \textit{SIX9} sequences from \textit{Fusarium} assemblies]{\textbf{Maximum likelihood phylogeny from alignment of \textit{SIX9} sequences from \textit{Fusarium} assemblies reveals \acl{Focub4} \textit{SIX9} homologue in S6}. \acl{sixg} identified by TBLASTN (cut off 1e\textsuperscript{-6}) using \textit{SIX} genes from \acl{Foly} as a reference. \textit{SIX9} sequences from \ac{tnau} are shown in red. White boxes indicate bootstrap values from 1000 bootstrap replicates. IQ-TREE2 (v2.2.0.3) best-fit model of substitution; TPM3+I. The tree is rooted through \textit{F. oxysporum f. sp. lycopersici} \textit{SIX9}.}
  \label{fig:FusSIX9}
\end{figure}

\begin{figure}[htp!]
  \centering
  \includegraphics[width=\textwidth]{Figures/FusSIX1.phylo.png}
  \caption[Maximum likelihood phylogeny from alignment of \textit{SIX1} sequences from \textit{Fusarium} assemblies]{\textbf{Maximum likelihood phylogeny from alignment of \textit{SIX1} sequences from \textit{Fusarium} assemblies}. \textit{SIX1} identified by TBLASTN (cut off 1e\textsuperscript{-6}) using \textit{SIX} genes from \acl{Foly} as a reference. White boxes indicate bootstrap values from 1000 bootstrap replicates. IQ-TREE2 (v2.2.0.3) best-fit model of substitution; TNe+R2 The tree is rooted through \textit{F. oxysporum f. sp. lycopersici} \textit{SIX1}.}
  \label{fig:FusSIX1}
\end{figure}

\begin{figure}[hp!]
\centering
    \begin{subfigure}[]{0.9\textwidth}
        \centering
        \includegraphics[width=\textwidth]{Figures/FusSIX4.phylo.png}
        \caption{}
        \label{fig:FusSIX4.phylo}
    \end{subfigure}
        \begin{subfigure}[]{0.9\textwidth}
        \centering
        \includegraphics[width=\textwidth]{Figures/FusSIX6.phylo.png}
        \caption{}
        \label{fig:FusSIX6.phylo}
    \end{subfigure}
    \caption[Maximum likelihood phylogeny from alignment of \textit{SIX4} and \textit{SIX6} gene sequences from \textit{Fusarium} assemblies.]{\textbf{Maximum likelihood phylogeny from alignment of \textit{SIX4} and \textit{SIX6} gene sequences from \textit{Fusarium} assemblies}.
    \acl{sixg}, \textit{SIX4} and \textit{SIX6}, identified by TBLASTN (cut off 1e\textsuperscript{-6}) using \textit{SIX} genes from \acl{Foly} as a reference. White boxes indicate bootstrap values from 1000 bootstrap replicates. The IQ-TREE2 (v2.2.0.3) best-fit model of substitution for \textit{SIX4} \subref{fig:FusSIX4.phylo}); K2P. The \textit{SIX4} tree is rooted through \textit{F. oxysporum f. sp. conglutinans} \textit{SIX4}. he IQ-TREE2 (v2.2.0.3) best-fit model of substitution for \textit{SIX6} \subref{fig:FusSIX6.phylo}); K2P. The  \textit{SIX6}  tree is rooted through \textit{F. oxysporum f. sp. lycopersici} \textit{SIX6}.}
    \label{fig:FusSIXMultiPhylo}
\end{figure}

\begin{figure}[htp!]
  \centering
  \includegraphics[width=\textwidth]{Figures/FusSIX2.phylo.png}
  \caption[Maximum likelihood phylogeny from alignment of \textit{SIX2} sequences from \textit{Fusarium} assemblies]{Maximum likelihood phylogeny from alignment of \textbf{\textit{SIX2} sequences from \textit{Fusarium} assemblies}. \textit{SIX2} genes identified by TBLASTN (cut off 1e\textsuperscript{-6}) using \textit{SIX} genes from \acl{Foly} as a reference. White boxes indicate bootstrap values from 1000 bootstrap replicates. The IQ-TREE2 (v2.2.0.3) best-fit model of substitution;  K2P. The tree is rooted through \textit{F. oxysporum} f. sp. \textit{lycopersici} \textit{SIX2}.}
  \label{fig:FusSIX2}
\end{figure}

\subsubsection{\Acl{ce} distribution in Tamil Nadu Agricultural University \textit{Fusarium} genome assemblies.}
\label{tnauCEs}

Small secreted proteins ($\geq30$aa to $\leq450$aa)  from the predicted proteomes of the \ac{tnau} genome assemblies were analysed using EffectorP (v2.0.1) to identify potential effectors. The candidate effectors identified here are not intended as primary diagnostic targets, instead their analysis provides an additional layer of evidence supporting the classification of the \ac{tnau} isolates, demonstrating similarities between them. Among the genome assemblies, S6 displayed the largest \acl{ce}ome, with 333 \acp{ce}, followed by genome assembly S32 with 314 \acp{ce}, while both SY-2 and S16 had 289 \acp{ce} each. Of these candidates, 132 were conserved between all genome assemblies when clustered at 80\% identity (Figure \ref{fig:TNAUVenn}). S6 also had the largest number of unique candidates when clustering at this percentage identity threshold, totalling 138. S32 had the next largest number of unique candidates, at 68. 70 of the candidates were shared between the S16 and SY-2 genome assemblies, but not the other genome assemblies. A total of 69 candidates were shared between S16, S32, and SY-2 but not in S6, and 39 candidates were shared between S6 and S32 but were not identified in SY-2 or S16 at the  80\% identity threshold. 

\begin{figure}[hp!]
  \centering
  \includegraphics[width=\textwidth]{Figures/sharedCandEffsVenn.png}
  \caption[Distribution of \acp{ce} among Tamil Nadu Agricultural University \textit{Fusarium} genome assemblies.]{\textbf{Distribution of \acp{ce} among Tamil Nadu Agricultural University \textit{Fusarium} genome assemblies.} Numbers represent the total \acp{ce}. SY-2=red oval, S6=grey oval, S16=blue oval, S32=yellow oval.}
  \label{fig:TNAUVenn}
\end{figure}


%%%%%%%%%%%%%%%%%%%%%%%%%%%%%%%%%%%%%%%%%%%%%
%DISCUSSION
%%%%%%%%%%%%%%%%%%%%%%%%%%%%%%%%%%%%%%%%%%%%%
\clearpage
\section{Discussion and Conclusions}

Collaborators at \ac{tnau} investigated \ac{Focub} diversity in Tamil Nadu. Samples were collected from symptomatic banana in banana-growing regions, including the Cavendish variety, 'Grande Naine'. Of the isolates collected, SY-2, S6, S16, and S32 were chosen for further study due to their highly virulent phenotype on Cavendish banana. Collaborators report (personal communication) that \ac{pcr} analysis confirmed SY-2, S6, S16, and S32 identity as \ac{Focub4} isolates\footnote{Using \textit{SIX9} primers from \textcite{Carvalhais2019}. It is important to note that collaborators at \ac{tnau} did not show data supporting their conclusions.}. However, our study does not support their conclusions. Genomic analysis suggests that banana-pathogenic isolate S32 belongs to a clade within genus \textit{Fusarium} species that may represent an as-yet undescribed species, that S6 belongs to a clade usually associated with \ac{Focub1}, but has acquired a \ac{Focub4} phenotype. Isolates SY-2 and S16 appear to belong to \ac{Fs}, a species recently reported to be pathogenic towards banana \parencite{Cui2021}. Interestingly, SY-2 and S16 are reported to display symptoms consistent with \ac{fwb} by collaborators at \ac{tnau}, whereas the \ac{Fs} isolate pathogenic towards Cavendish reported by \textcite{Cui2021} (FS66) displays foliar blight symptoms. \textcite{Cui2021} report that \ac{Fs} isolate FS66 has widespread gene transfer from core chromosomes identified in \ac{FOSC}, as well as 30 genes involved in Fusarium pathogenicity/virulence; including \textit{FoSlt2} which encodes a mitogen-activated protein kinase required for virulence in \ac{Focub} \parencite{ding2015mitogen}.  As we have not been able to access the \ac{tnau} \textit{Fusarium} isolates, we have been unable to compare symptoms to FS66, the \ac{Fs} isolate pathogenic towards Cavendish reported by \textcite{Cui2021}. 

Further complicating our study, the raw read data for isolates S32 and S6 contained high levels of contamination, which posed significant challenges. Unfortunately, access to cleaner raw reads from collaborators at \ac{tnau} before the project was completed was not possible. While the degree of bacterial contamination in isolates S6 and S32 made \textit{de novo} assembly challenging, every attempt was made to improve the quality of the data that we did have access to. Despite this, the read depth for isolate S6 in the \textit{de novo} assembly was only 16x, which is suboptimal for high-confidence genomic analysis. Though the inclusion of this data was not ideal, given the constraints of the project timeline and the challenges in obtaining uncontaminated data, it was included to demonstrate the efforts made to extract meaningful insights. The decision to report on these findings reflects the reality of the difficulties faced during the project and provides transparency regarding the quality and limitations of the data.

Initially, the raw read data supplied by collaborators at \ac{tnau} were aligned to high-quality \ac{Focub4} (\href{https://www.ncbi.nlm.nih.gov/datasets/genome/GCA_007994515.1/}{GCA\_007994515.1}) and \ac{Focub1} (\href{https://www.ncbi.nlm.nih.gov/datasets/genome/GCA_005930515.1/}{GCA\_005930515.1})  assemblies for a reference-guided assembly approach. However, alignment rates and web BLASTN searches of the unmapped reads against the nr/nt database contradicted TNAU's initial classifications, and revealed bacterial contamination in the S6 and S32 datasets. Later supported by BlobTools (v1.1.1) analysis of \textit{de novo} assembly constructed using all raw reads (See section: \ref{sec:BlobToolsOfS6S32-allreads}). Consequently, raw reads were mapped to assemblies for which a large number of highly similar ($ \geq90\% $ identity) BLASTN hits were identified. Genome assemblies for SY-2, S16, and S32 showed closer alignment to the \textit{F. sacchari} reference compared to the \ac{Focub} references, and S6 and S32 contained contamination from \textit{S. maltophilia}.

As these results suggested that SY-2, S16, and S32 are not \ac{Focub4} isolates, it was no longer clear which species should be used as a reference for a reference-guided assembly. Further, the \ac{tnau} isolates displayed a highly virulent phenotype and a reference-guided assembly could lose any large-scale rearrangements in the genome which may contribute to this phenotype. However, contamination in the S6 and S32 data meant that a \textit{de novo} genome assembly for these isolates was likely to result in misassembled contigs that are chimeric (part target species, part non-target species). These contigs can be challenging to identify and may result in contigs that should remain in the assembly being filtered out, and contigs that do not belong to the target species being kept in the assembly, even when using BlobTools (v1.1.1.) to separate target and non-target contigs \parencite{Cornet2022}, so must be removed from the raw read data. 

Consequently, genomes for SY-2 and S16 were assembled \textit{de novo} and S6 and S32 were assembled \textit{de novo} using only the reads that mapped to reference genomes with the highest mapping value. For S6, therefore, a \textit{de novo} genome assembly was generated using the raw reads which mapped to \ac{Focub1} 160527 (\href{https://www.ncbi.nlm.nih.gov/datasets/genome/GCA_005930515.1/}{GCA\_005930515.1}) reference and for S32 a \textit{de novo} assembly was generated using the raw reads which mapped to the \ac{Fs} FS66 (\href{https://www.ncbi.nlm.nih.gov/datasets/genome/GCA_017165645.1/}{GCA\_017165645.1}) reference. BlobTools (v1.1.1) was used to confirm that the levels of contamination in the assemblies had been reduced, pinpoint any potential contaminant contigs to remove, and support isolate classification. Predicted protein coding genes, percentage GC content, size, and \ac{busco} scores for the contaminant filtered assemblies were comparable to other \textit{Fusarium} short-read assemblies (see: \textcite{DitaHerai2013, Chiara2015, Srivastava2018}).

As the identity of \ac{tnau} isolates was uncertain, the genetic markers \acf{tef} and \acf{rbp2} extracted from the assemblies and queried against Fusariod-ID MSLT and \ac{ncbi} BLAST databases. For the S16 extracted \ac{tef}, the top hits in the Fusariod MSLT database and \ac{ncbi} database were associated with the \ac{FFSC} and \ac{Fs}, respectively (Table \ref{tab:Tef1-NCBIdb}, Appendic A\ref{tab:Tef1-MLSTdb}). In the case of S6, the extracted \ac{tef} sequences indicated a potential association with the \ac{FOSC}, although specific matches for \ac{Focub} \ac{tef} sequences were not among the top results from both databases. Notably, no matches were found for the S32 \ac{tef} sequences in the Fusarioid-ID MSLT database, while hits in the \ac{ncbi} GenBank database pointed towards \ac{Ff} sequences, further suggesting that S32 is a member of the \ac{FFSC} pathogenic towards banana. Expanding the marker count has the potential to fortify the robustness of our phylogenetic inferences. 

Phylogenies generated using the same genetic markers (\ac{tef} and \ac{rbp2}) supported these initial conclusions. S6 clustered within a \ac{Focub1} clade in both \acs{tef} and \ac{rbp2} phylogenies (Figure \ref{fig:TEF1aPhylo}, Appendix A\ref{fig:rbp2Phylo}) and, S16 appeared to align closely with SY-2 and reference \ac{Fs} species in the phylogenetic analysis. Notably, S32's placement suggested a relation to the recently proposed species, \acf{Fm}, identified in symptomatic Cavendish bananas in the Philippines \parencite{Nozawa2023}. \textcite{Nozawa2023} suggest that \ac{Fm} is a sister linage of the \ac{Fs}, which our data supports. The authors also suggest this newly proposed species has acquired the \ac{sixg} \textit{SIX6}, though they have not confirmed the \textit{SIX6} homologue identified is involved in virulence. Though S32 falls within the same clade as the novel species proposed by \textcite{Nozawa2023}, we did not identify a \textit{SIX6} homologue in the S32 assembly, which suggests that alternative virulence genes are associated with S32 pathogenicity in banana. The lack of \textit{SIX6} in S32 may also be an artefact of the raw read filtering, whereby only reads that mapped to the \ac{Fs} reference (FS66) were included. The FS66 genome assembly does not include a \textit{SIX6} homologue. To confirm the absence of \textit{SIX6} in S32, raw reads could be searched against the \textit{SIX6} reference sequence. 

The apparent absence of any \acp{sixg} in the S32 assembly distinguishes it from the other \ac{tnau} genome assemblies which also appear to be in the \ac{FFSC}, as SY-2 and S16 both contain a \textit{SIX2} homologue. Further, the \textit{SIX2} homologue identified in the SY-2 and S16 assemblies supports their classification as \ac{Fs} isolates pathogenic towards banana, as they fall within the same clade as the \ac{Fs} isolates included, FS66 and NRRL 66326. FS66 is reported as pathogenic towards banana \parencite{Cui2021}, and the FS66, SY-2, and S16 \textit{SIX2} homologues have 100\% identity (Figure \ref{fig:FusSIX2}). However, the role of the \textit{SIX2} homologue in \ac{Fs} has not yet been characterised, nor has expression been confirmed. The presence of \textit{SIX2} in S16 and SY-2 genomes may contribute towards their virulence in Cavendish banana varieties, as \textit{SIX2} is only found in \ac{r4} genomes of \ac{Focub} \parencite{Czislowski2018}. However, \textcite{An2019} suggest that \textit{SIX2} has a reduced role in virulence as inoculation of \textit{Musa acuminata} L. ‘Brazilian’ (AAA group) plantlets with \textit{SIX2} knock-out mutants showed similar disease progress as plants inoculated with wild-type \ac{Focub4}. Also, confirmation of \textit{SIX2} expression during infection in SY-2 or S16 is still required. 

Variation in \acp{sixg} in S6 compared to other \ac{Focub} genome assemblies can also be observed. The \acp{sixg} identified in S6 consistently fell within the same clade as \ac{Focub1} genomes, though copy number varied (Figures \ref{fig:FusSIX9}, \ref{fig:FusSIX1}, and \ref{fig:FusSIXMultiPhylo}). It is important to consider the assembly approach, however. Despite using raw reads that mapped to the \ac{Focub1} assembly from \textcite{Asai2019} (\href{https://www.ncbi.nlm.nih.gov/datasets/genome/GCA_005930515.1/}{GCA\_005930515.1}) or had no-hit, the S6 genome was assembled \textit{de novo}. This approach aimed to ensure an unbiased depiction of the genetic structure, including the \acp{sixg} sequence, in the S6 (and S32) assembly. Interestingly, collaborators at \ac{tnau} suggest that S6 is highly virulent towards Cavendish banana varieties. It is of note, that a \textit{SIX13} homologue was identified in the other \ac{Focub1} genomes used in this study, but was not found in the S6 assembly.  This may be due to the assembly quality of S6. High contiguity is often challenging to achieve in repeat-rich regions of short-read assemblies \parencite{Treangen2012, Peona2021}, which is where the many \textit{Fusarium} effectors can be found \parencite{Ma2010, Schmidt2013, Armitage2018}. It is possible that no \textit{SIX13} was identified due to genome assembly fragmentation. Alternatively, the absence of \textit{SIX13} may contribute towards S6 virulence on Cavendish banana. 

Collaborators at \ac{tnau} classified isolates S6, S16, S32, and SY-2 as \ac{Focub4} using Koch's postulates to confirm virulence towards Cavendish  'Grande Naine' and \ac{pcr} amplification of \textit{SIX9} (primers developed by \textcite{Carvalhais2019}) for \ac{Focub} \ac{r4} confirmation (personal communication). \textcite{Carvalhais2019} presented multiple primer sets designed to identify variation within \ac{Focub4}. They developed primers that specifically amplify regions of \textit{SIX1} in \ac{tr4}, \textit{SIX8} in \ac{str4}, \textit{SIX9/SIX10} in \ac{Focub} \ac{vcg} 0121 (\ac{r4}), and \textit{SIX13} in \ac{Focub} \ac{vcg} 0122 (R4). Therefore, the use solely of the \textit{SIX9} primer set by collaborators at \ac{tnau} was not appropriate for full \ac{tr4} classification. Further, our \ac{sixg} analysis (see \ref{sec:chap2SixGene}) contradicts the classification and results proposed by collaborators at \ac{tnau} (personal communication), as \textit{SIX9} was only identified in the S6 assembly, and the \ac{tr4}-specific \textit{SIX9} homologue was not identified (Figure \ref{fig:FusSIX9}). It is important to note that gel images or comprehensive results were not provided by \ac{tnau} to confirm their \ac{pcr} results. Further, we did not identify \textit{SIX8}, \textit{SIX9, SIX10}, or \textit{SIX13}, for which  \textcite{Carvalhais2019} have developed specific primer sets, in any of the \ac{tnau} assemblies. 

There are challenges associated with molecular diagnostics tools beyond those encountered by collaborators at \ac{tnau}. A major concern in \textit{Fo} pathogen molecular diagnostics is the high genetic diversity within the \ac{FOSC}, which includes both pathogenic and non-pathogenic strains, as highlighted by \textcite{demers2015highly}. This diversity can compromise the specificity of molecular markers, as shown by \textcite{Magdama2019}, who found that some primer sets intended to identify \ac{Focub4} produced false positives or negatives for \ac{Focub1}, \ac{r2}, and non-pathogenic isolates. Such cross-reactivity poses significant challenges, particularly given recent global outbreaks of \ac{fwb} caused by \ac{Focub4}, where precise detection is vital for effective management, including quarantine and crop destruction. \textcite{Magdama2019} suggest targeting virulence-associated genetic loci; however, due to high evolutionary pressure on fungal effectors \parencite{Todd2022}, the long-term reliability of these targets is uncertain. As pathogens evolve, maintaining effective diagnostics requires constant adaptation. A potentially more robust approach might be to target conserved pathogenic genes with sequence variations to distinguish between pathogens and non-pathogens if these exist in \ac{Focub}. 

Access to molecular diagnostics also presents challenges for growers. They must rely on agricultural extension services, which need both the expertise and resources to conduct assays. This reliance can be problematic, especially in remote or under-resourced areas where timely and accurate diagnostics are critical for effective disease management. Improving the availability and affordability of diagnostic services, and ensuring that extension services are well-supported, is essential for enhancing disease control measures.

% As well as \acp{sixg}, other \acfp{ce} were identified and clustered based on sequence identity. Though this chapter does not explore these \acp{ce} in-depth, similarities can be identified which help to support our classification of the \ac{tnau} isolates. The \ac{ce} profiles of SY-2 and S16 reinforce the proposition that these two isolates are closely related if not isolates of the same \ac{Fs} strain. Both SY-2 and S16 had 289 \acp{ce} each, 282 of which are shared when clustering at 80\% identity, and only one candidate was unique to each genome assembly. For S6, however, 138 of its 333 \acp{ce} were not shared between any of the other \ac{tnau} genome assemblies. Comparison of these predicted effectors against the effector repertoires of \ac{Focub} and \ac{Fs} would help to further corroborate our classification of the \ac{tnau} isolates and may provide insights into the \textit{Fusarium} host specificity. This is partially explored in Chapter \ref{Chap3}. It is important to highlight, however, that it may be challenging to draw clear conclusions from downstream analyses of the predicted proteomes of the \ac{tnau} genome assemblies. Annotations were generated without transcriptome data from these isolates, and assemblies may contain fragments of contamination despite the care was taken to filter contaminants from the raw data. Furthermore, assemblies are not highly contiguous ($ \geq408$ contigs), so gene models may become truncated or missed, particularly genes typically found in repeat-rich regions, such as effectors \parencite{Schmidt2013}.  

As well as \acp{sixg}, other \acfp{ce} were identified and clustered based on sequence identity. Though this chapter does not explore this \acp{ce} in-depth, similarities that help support our classification of the \ac{tnau} isolates identified. The \ac{ce} profiles of SY-2 and S16 reinforce the proposition that these two isolates are closely related if not isolates of the same \ac{Fs} strain. SY-2 and S16 had 289 \acp{ce} each, 282 of which are shared when clustered at 80\% identity, and only one candidate was unique to each genome assembly. For S6, however, 138 of its 333 \acp{ce} were not shared with any other \ac{tnau} genome assemblies.

The \ac{ce} Venn diagram was generated to highlight the observed differences in the \ac{tnau} isolates, given the variance in their predicted \ac{ce} repertoires. Ideally, these comparisons would extend to a broader range of \textit{Fusarium} species for a more comprehensive analysis. However, this could not be completed due to time constraints. Comparison of the \ac{tnau} predicted effectors against the effector repertoires of \ac{Focub} and \ac{Fs} would help to further corroborate our classification of the \ac{tnau} isolates and may provide insights into \textit{Fusarium} host specificity. It is important to highlight, however, that it may be challenging to draw clear conclusions from downstream analyses of the predicted proteomes of the \ac{tnau} genome assemblies produced as part of this study. Annotations were generated without transcriptome data from these isolates, and assemblies may contain fragments of contamination despite the care taken to filter contaminants from the raw data. Furthermore, assemblies are not highly contiguous ($\geq408$ contigs), so gene models may become truncated or missed, particularly genes typically found in repeat-rich regions, such as effectors \parencite{Schmidt2013}. Future work should include the generation of new-long read sequences for these \ac{tnau} isolates, as well as comparisons to validate the \ac{ce} profiles observed and to explore their potential relevance for species differentiation and diagnostics. 

\textcite{Thangavelu2020} identified \ac{Focub1} isolates within the 0125 and 01220 \ac{vcg} complex from Tamil Nadu that are pathogenic to Cavendish bananas. The S6 isolate may also belong to this \ac{vcg} complex, but laboratory assays are necessary for confirmation. Classifying both the 0125 and 01220 \ac{vcg} isolates and S6 as \ac{Focub1} pathogenic to Cavendish bananas is problematic, as \ac{Focub} race is determined by host specificity. However, classifying isolates as \ac{Focub4} does not fully reflect the diversity of \textit{Fusaria} in our South India sampling region; thus, S6 is described as a \ac{Focub1} isolate with a \ac{Focub4} phenotype. 

Given that many \acp{fsp} are polyphyletic \parencite{Czislowski2018, Henry2021, jerushalmi2022members}, and that horizontal transfer of pathogenicity chromosomes may account for this diversity \parencite{Fourie2009, Schmidt2013, Chiara2015, Epstein2022}, \textcite{Torres2021} argue that race classification should be considered a "phenotypic rather than taxonomic designation". I support this perspective and propose extending it to recognise \ac{fsp} classifications as primarily phenotypic designation. However, this approach still leaves a gap in the \ac{Fo} classification system. To better capture both genetic diversity and phenotypic similarity, a new classification approach that integrates genetic and phenotypic data should be developed.

The spread of \ac{Focub} throughout banana-growing regions, particularly in India, is a major concern. This chapter addressed some of the first objectives of this thesis, it contributes to our understanding of the diversity, and classification of \ac{fwb}, and highlights the complex evolutionary history of \textit{Fusarium} by characterising the genomic diversity and pathogenic potential of \textit{Fusarium} isolates from Tamil Nadu. This chapter revealed significant discrepancies between traditional diagnostics used by collaborators at \ac{tnau} for isolate classification and our genomic analysis. By comparing raw genomic read data, genetic barcodes, \ac{sixg} profiles, and \acl{ce} profiles, we have identified multiple lines of evidence indicating that the \ac{tnau} isolates are distinct from \ac{Focub4}. All isolates assessed in this study are reported to be highly virulent on Cavendish banana by collaborators at \ac{tnau}; S6 appears to be a \ac{Focub1} isolate displaying a \ac{Focub4} phenotype, while SY-2, S16, and S32 are part of the \ac{FFSC}. Although much emphasis has been placed on the threat of \ac{Focub4} to Cavendish production in this region \parencite{Viljoen2020, Damodaran2019}, our results suggest that the \ac{Focub4} \ac{vcg} 01213/16 is not the only threat and underscores the need for continuous monitoring, accurate diagnostics, and effective control measures. The data presented in this chapter challenge the current diagnostic frameworks and highlight the need for more comprehensive approaches. Future research should focus on developing multi-locusm=, robust diagnostic tools to better account for the genetic diversity and evolutionary dynamics observed in these pathogens.