%%%%%%%%%%%%%%%%%%%%%%%%%%%%%%%%%%%%%%%%%%%%%
%INTRODUCTION
%%%%%%%%%%%%%%%%%%%%%%%%%%%%%%%%%%%%%%%%%%%%%

\section{\textbf{DRAFT:}An Introduction to plant pathogenic Fusarium species}
No text here yet. 



\subsection{\textit{Fusarium} of banana}
\subsection{Isolates collected by Tamil Nadu Agricultural University}

- Set story with TNAU isolates and Fusarium in India then add justification for this work!

This chapter presents the assembly and analysis of the genomic characteristics of \textit{Fusarium} isolates (SY-2, S6, S16, and S32) obtained from highly virulent on Cavendish banana plants by collaborators at Tamil Nadu Agricultural University (TNAU). taxonomic identity and genetic diversity are assessed and suggest that S6 is a \Focub race 1 isolate, while SY-2, S16 and S32 belong to the\textit{ F. fujikuroi} species complex. Phylogenetic analysis and \textit{SIX} gene profiling supported these classifications. Further investigations are necessary to ascertain whether SY-2, S16 and S32 represent a novel species within the \textit{F. fujikuroi} complex. Additionally, bacterial contamination with \textit{Stenotrophomonas} species was detected in S6 and S32 isolates, warranting future studies to explore potential implications on banana pathogenicity.

%%%%%%%%%%%%%%%%%%%%%%%%%%%%%%%%%%%%%%%%%%%%%
%METHODS
%%%%%%%%%%%%%%%%%%%%%%%%%%%%%%%%%%%%%%%%%%%%%

\section{Materials and Methods}
\subsection{Phenotyping and genotyping data supplied by TNAU.}

Collaborators a Tamil Nadu Agricultural University aimed to assess the diversity of the \Foc in Tamil Nadu and to understand the genetic and molecular basis of resistance against FoC in banana through next-generation genomics. A survey was conducted in various banana-growing areas of Tamil Nadu to record symptoms of wilt in different banana cultivars. 

Samples displaying symptoms of \textit{Fusarium} wilt, including the \Foc TR4 susceptible Cavendish banana variety 'Grande Nain' were collected from the field, and the pathogen was isolated and cultured in the laboratory. The isolates SY-2, S6, S16, and S32, were among the samples collected and analyzed. DNA extraction and PCR amplification were performed to confirm the isolates' identity, including SY-2, S6, S16, and S32. These isolates caused severe disease symptoms in tissue-cultured 'Grand Naine' banana plants and were classified as highly virulent. PCR amplification using race 4-specific primers confirmed that SY-2, S6, S16, and S32 were \Foc Tropical Race 4 isolates, capable of infecting Cavendish banana (Work conducted at TNUA is summarised in Appendix ). Consequently, isolates SY-2, S6, S16, and S32 were sent for sequencing using the \textcolor{red}{WHICH PLATFORM, WHAT STATS CAN WE EXTRACT HERE?}.\footnote{\textcolor{red}{Due to communication challenges with collaborators at TNAU, we have not been able to establish the extraction protocol used to produce the DNA used for sequencing.}}

\subsection{Analysis of Raw Read data from TNAU isolates}

The raw Illumina paired-end reads for the \textit{Fusarium} isolates SY-2, S6, S16, and S32, which are reported to be highly virulent on Cavendish banana, were supplied by collaborators at Tamil Nadu Agricultural University, India (TNAU). Following FastQC (Version 0.11.8) (Andrews, 2010) analysis, raw reads were mapped to \Focub TR4 isolate UK0001 (Warmington, \et 2019) reference quality genome using BOWTIE2 (version 2.4.5) (Langmead and Salzberg, 2012). Due to poor mapping rates for some isolates, 1000 raw reads were extracted from each TNAU isolate and searched using NCBI web BLAST (Nih.gov, 2014). 
Raw reads were then mapped to \Focub R1 isolate 160527(Asai, \et 2019), as well as \textit{Fusairum sacchari} isolate FS66 (Cui \et 2021). Isolates S6, S16 and S32 were also mapped to a reference quality \textit{Stenotrophomonas maltophilia} isolate NCTC10258 genome (GCA\_900475405.1) 

\subsection{\textit{De novo} Genome Assembly and Contamination Analysis}

A  \textit{de novo}assembly was generated for the TNAU isolates using SPAdes (v3.14.1). SPAdes is routinely employed when generating \textit{Fusarium} genomes (see: Armitage, \et 2018; Hudson \et 2020, Tanaka, \et 2022). A custom Python script was developed to remove sequences with <25\% GC content from the assembly (Appendix ~\ref{apx:gcTrimmer.py}). For assemblies with high levels of contamination (S32, S6), only reads mapping to reference genomes were used to generate \textit{de novo} assemblies.

Blobtools (v1.1.1) (Laetsch and Blaxter, 2017) was used for the taxonomic partitioning of the assemblies. The TNAU assemblies were searched against the NCBI nt database using BLASTN and the paired-end raw reads were aligned to each of the assemblies using BOWTIE2 (version 2.4.5). Taxonomic hits were ranked at the species level and default BolbTools settings were used. To extract contigs which have been assigned to a non-\textit{Fusarium} genus by BlobTools, the blob tools output json file was filtered using the blobtools view -r species flag. Contigs assigned to a \textit{Fusarium} species or "no-hit" were then extracted and saved in a separate text file. the blobtools seqfilter command with the default settings then used to extract contigs which had been assigned to \textit{Fusarium} or had "no-hit" (script available in Appendix ~\ref{apx:ContamFilter}).

The quality and completeness of the assembled genomes were estimated using Benchmarking Universal Single-Copy Orthologs (BUSCO) with the data set hypocreales\_odb10 (Manni \et 2021). The raw reads were mapped back to the assemblies and a Qualimap assessment was conducted (García-Alcalde \et 2012). Quast (v5.0.2) (Gurevich \et 2013) was used to estimate assembly quality. 


\subsection{Annotation Approach}

\subsection{Phylogenetic Analysis of TNAU isolates}\label{chap2:phylogeny}

The common Fusarium genetic barcodes \textit{Translation elognation factor-1\(\alpha\)} (Tef-1\(\alpha\)) and RPBII  were used to generate phylogenies (Edel-Hermann and Lecomte, 2019). Briefly, a Tef-1\(\alpha\) and RPBII sequence database were compiled using available reference sequences from the NCBI database. Homologs of each barcode were identified in each assembly in our database (See section:~\ref{chap3:fusariumdb}) using BLASTN (1e-\textsuperscript{6} cut-off). The locations of hits with greater than 70\% identity and 90\% coverage were recorded, and the sequence within this region was extracted using Samtools (Version 1.15.1). The Tef-1\(\alpha\) and RPBII regions from each genome were concatenated into a single FASTA file for each barcode. The online web server MAFFT (Katoh \et 2019) was used to construct a multiple sequence alignment with the setting “Adjust direction according to the first sequence” to ensure correct alignment and any overhanging regions were trimmed manually. IQ-TREE (Version 2.2.0.3) (Nguyen \et 2015) was used to infer a maximum-likelihood phylogeny using the ultrafast bootstrap setting for 1000 bootstrap replicates and was visualised using iTOL (Letunic and Bork, 2021). 

\subsection{Identification of pathogen-specific regions.}


\subsection{Ortholog analysis?}

\subsection{SIX gene searches – remove coverage and ident thresholds.}
SIX genes from Fol isolate Fol007 were downloaded from the GenBank were used as a query in a BLASTX search (1e-6 cut-off, 50\% identity and 70\% coverage threshold) against the genomes to check the presence/absence of SIX genes in all reading frames. A binary data matrix indicating presence (“1”) or absence (“0”) was generated using the BLASTX hit data and a heatmap is generated using the R package, Pheatmap.

\subsection{\textit{Mimp} identification approach.}

\subsection{EffectorP analysis of proteome through SignalP and EffectorP}

%%%%%%%%%%%%%%%%%%%%%%%%%%%%%%%%%%%%%%%%%%%%%
%RESULTS
%%%%%%%%%%%%%%%%%%%%%%%%%%%%%%%%%%%%%%%%%%%%%

\section{Results}
\subsection{Phenotyping and genotyping data supplied by TNUA.}

\subsection{Analysis of Raw Read data from TNAU isolates}

Raw read mapping of the SY-2, S6, S16, and S32 to the \Focub TR4 UK0001 assembly, produced a 54.11\%, 8.72\%, 53.81\%, and a 15.69\% alignment for all raw reads, respectively (~\ref{tab:RawReadMapping}). 

Due to the low alignment rates, unmapped reads were extracted and a random subset of 1000 reads per isolate were searched using NCBI web BLAST (Nih.gov, 2014). For isolates S6 and S32, the majority of hits with >90\% coverage and identity were for \textit{Stenotrophomonas} species, particularly \textit{Stenotrophomonas maltophilia}. Further, the raw S6 and S32 reads show a similar GC\% to the \textit{S. maltophilia }reference genome assembly (GCA\_900475405.1) (S6=63\%, S32=61\%, \textit{S. maltophilia} reference=66.5\%). For isolate S16, the majority of hits for unmapped reads with >90\% coverage and identity against the NCBI database were for \textit{Fusarium fujikuroi}. Isolates S6, S16 and S32 were also mapped to a reference quality \textit{Stenotrophomonas maltophilia} genome due to to the high number of blast hits. 

Approximately 50\% of raw reads from isolates S6 and S32 mapped to the \textit{S. maltophilia} reference (Table ~\ref{tab:RawReadMapping}). Raw reads from isolate S16 only had a 0.01\% mapping rate to the \textit{S. maltophilia} reference. Raw reads from isolates S6 and S32 had a 5.24\% and 22.49\% mapping rate to the \textit{F. sacchari} reference, respectively, whereas 68.65\% of the raw reads from isolate S16  and 93.96\% of the raw reads from the SY-2 isolate mapped to the \textit{F. sacchari} reference.  

% Please add the following required packages to your document preamble:
% \usepackage{multirow}
% \usepackage{lscape}
% \usepackage{longtable}
% Note: It may be necessary to compile the document several times to get a multi-page table to line up properly

\begin{landscape}
\begingroup
\setlength{\tabcolsep}{6pt} % Default value: 6pt
\renewcommand{\arraystretch}{0.93}
\begin{longtable}[c]{ccccccccc}
\caption[Overall alignment rate of all raw reads from each TNAU isolates to fungal and bacterial reference species]{\textbf{Overall alignment rate of all raw reads from each TNAU isolates to fungal and bacterial reference species.} Overall alignment rate determined by Bowtie2 (version 2.4.5). Reference assemblies were downloaded from GenBank.}
\label{tab:RawReadMapping}\\
\hline
\multirow{\textbf{\begin{tabular}[c]{@{}c@{}}Reference \\ Species\end{tabular}}} &
  \multicolumn{4}{c}{\textbf{\begin{tabular}[c]{@{}c@{}}Isolate Bowtie2 Raw Read \\ Overall Alignment Rate\end{tabular}}} &
  \multirow{\textbf{\begin{tabular}[c]{@{}c@{}}Reference \\ Strain\end{tabular}}} &
  \multirow{\textbf{\begin{tabular}[c]{@{}c@{}}Reference \\ GenBank \\ Accession\end{tabular}}} &
  \multirow{\textbf{\begin{tabular}[c]{@{}c@{}}No. of \\ Contigs\end{tabular}}} &
  \multirow{\textbf{\begin{tabular}[c]{@{}c@{}}Contig \\ N50 \\ (Mb)\end{tabular}}} \\ \cline{2-5}
 &
  \textbf{SY-2} &
  \textbf{S6} &
  \textbf{S16} &
  \textbf{S32} &
   &
   &
   &
   \\ \hline
\endfirsthead
%
\multicolumn{9}{c}%
{{\bfseries Table \thetable\ continued from previous page}} \\
\endhead
%
\multicolumn{1}{c}{\textit{\begin{tabular}[c]{@{}c@{}}Fusarium oxypsorum \\ f. sp. cubense TR4\end{tabular}}} &
  54.11\% &
  8.72\% &
  53.81\% &
  15.69\% &
  UK0001 &
  GCA\_007994515.1 &
  15 &
  4.49 \\
\multicolumn{1}{c}{\textit{\begin{tabular}[c]{@{}c@{}}Fusarium oxysporum \\ f. sp. cubense R1\end{tabular}}} &
  54.02\% &
  H &
  H &
  H &
  160527 &
  GCA\_005930515.1 &
  12 &
  4.88 \\
\multicolumn{1}{c}{\textit{F. sacchari}} &
  93.96\% &
  5.24\% &
  68.65\% &
  22.49\% &
  FS66 &
  GCA\_017165645.1 &
  48 &
  1.97 \\
\multicolumn{1}{c}{\textit{\begin{tabular}[c]{@{}c@{}}Stenotrophomonas \\ maltophilia\end{tabular}}} &
  *\footnote{SY-2 was not assessed for \textit{Stenotrophomonas maltophilia} contamination as it was sent as part of a previous sequencing run. } &
  49.32\% &
  0.01\% &
  53.93\% &
  NCTC10258 &
  GCA\_900475405.1 &
  1 &
  4.5 \\
  \hline
\end{longtable}
\endgroup
\end{landscape}

\subsection{\textit{De novo} Genome Assembly and Contamination Analysis}

As raw reads alignment rate for \Foc Race 1 (Asai, \et 2019) and \Foc TR4 (Warmington, \et 2019) were <55\% for all TNAU isolates, a \textit{de novo} approach using SPAdes (v3.14.1) was used for genome assembly. Following assembly, contigs with \textless 25\% GC content were removed from the assembly SY-2 and S16 assemblies.

Using SPAdes, the SY-2 and S16 assemblies recorded 1,591 and 768 contigs and a BUSCO completeness score of C:99.6\% and 97.4\% (hypocreales\_odb10 dataset), respectively. The raw reads from each isolate were mapped back to the assemblies and a Qualimap assessment was conducted; 98.88\% of the raw reads from the SY-2 isolate mapped back to the \textit{de novo} SY-2 assembly and coverage was estimated to be 70.38x. In the S16 isolate, 99.53\% of the reads were mapped and mean coverage was estimated to be 148x (Table~\ref{tab:TNAUAssemblyStats}).

Although the S6 and S32 isolates had high mapping rates to \textit{S. maltophilia} reference, all raw reads were used to generate a \textit{de novo} assembly to assess for taxonomic partitioning using BlobTools (v1.1.1) (Laetsch and Blaxter, 2017). The S6 and S32 assemblies generated using all raw reads contained 100147 and 1048 contigs, had a BUSCO completeness score of 97.7\% and 97.7\%, were 97.64Mb and 49.62Mb in length and had a GC\% of 46.75 and 49.80, respectively. The S6 assembly is much larger than is typical for a \textit{Fusarium oxysporum} assembly and is highly fragmented.  

% Please add the following required packages to your document preamble:
% \usepackage{multirow}
% \usepackage{longtable}
% Note: It may be necessary to compile the document several times to get a multi-page table to line up properly
%\begin{landscape}
\begingroup
\setlength{\tabcolsep}{3pt} % Default value: 6pt
\renewcommand{\arraystretch}{0.7}
\setlength\LTcapwidth{\textwidth} % default: 4in (rather less than \textwidth...)
\setlength\LTleft{0pt}            % default: \parindent
\setlength\LTright{0pt}           % default: \fill
\begin{longtable}[c]{ccccc}
\captionsetup{width=\linewidth} 
\caption[Summary statistics of TNAU genome assemblies.]{\textbf{Summary statistics of TNAU genome assemblies. }\textit{De novo} assemblies generated using SPAdes (version 3.14.1) with all raw reads supplied by Tamil Nadu Agricultural University.}
\label{tab:TNAUAssemblyStats}\\
\hline
\multirow{2}{*}{\textbf{\begin{tabular}[c]{@{}l@{}}Assembly\\Statistic\end{tabular}}} & \multicolumn{4}{c}{\textbf{TNAU Isolate Assembly}} \\ \cline{2-5} 
                        & \textbf{S6} & \textbf{S16} & \textbf{S32} & \textbf{SY-2} \\ \hline
\endfirsthead
%
\multicolumn{5}{c}%
{{\bfseries Table \thetable\ continued from previous page}} \\
\endhead
%
No. PE reads                       & 26,213,258     & 22,473,220     & 34,526,189      & 11,032,845 \\
Number of contigs                  & 6,048          & 768            & 2,443           & 408    \\
Largest contig (Mb)                & 0.43           & 0.88           & 0.77            & 0.87   \\
Total length (Mb)                  & 47.23          & 44.86          & 40.92           & 44.22  \\
GC (\%)                            & 47.87          & 47.53          & 48.85           & 47.98  \\
N50 (bp)                           & 92,555         & 234,991        & 78,523          & 200,307 \\
L50                                & 148            & 60             & 109             & 63     \\
Mapped Reads (\%)                  & 9.9            & 99.53          & 24.01           & 99.6   \\
Mean Coverage                      & 16x            & 148x           & 60x             & 73x    \\ 
BUSCO (\%)                         & 97.5           & 97.4           & 97.4            & 99.6   \\
Bases soft masked(Mb)              & 2.115 (4.48\%) & 1.853 (4.13\%) & 0.612 (1.50\%)  & 1.430 (3.23\%)      \\
Predicted protein coding genes     & 17,891         & 15,727         & 15,824          & 15,719     \\
\hline  
\end{longtable}
\endgroup
%\end{landscape}


Once the \textit{de novo} assembly was generated for each isolate, a Blobtools analysis was conducted. \textit{Fusarium fujikuroi} accounted for the majority of the NCBI TaxID hits for each contig for the S16 isolate and SY-2 assemblies, with \textcolor{red}{[HOW MANY?]} contigs assigned to this species, respectively (Figures). 

\textcolor{red}{[Need image for S16 and SY-2]}

The de novo assemblies generated using all raw reads for the S6 and S32 isolates contained a large number of contigs which had either no hits or were assigned to other genera. For instance, the majority of contigs for the de novo S6 assembly had no hits (Figure 2). Some of the contigs had Fusarium fujikuroi and Fusarium oxysporum assigned. The remaining contigs had greatest sequence similarity to bacterial species. The majority of contigs from the S32 de novo assembly had greatest sequence similarity to Fusarium species, particularly F. fujikuroi, although some contigs were assigned to Stenotrophomonas species, as was observed in the BLAST search of unmapped raw reads (Figure 3).  

\textcolor{red}{[Justification for discard S6 assembly and only using reads – David’s suggestion.]}

\subsection{Annotations Approach}

\subsection{Results from \textit{Tef-1\(\alpha\)} and RBPII Phylogenies and database searches showing novel clade.}

The common Fusarium genetic barcode  (Edel-Hermann and Lecomte, C., 2019) was used for phylogenetic analysis of the TNAU isolates alongside other Fusarium species. The SY-2 isolate \Tef sequence extracted for the phylogeny shared in June 2022 was also included in this phylogeny for reference. For the S16 isolate, the \Tef  sequence was extracted from the GC-trimmed de novo assembly. For the S6 and S32 isolates, the \Tef  sequence was extracted from the de novo assemblies generated using all raw reads as well as the de novo assemblies generated using the reference mapped reads.  

Using a \Tef database compiled from available reference sequences on the NCBI database, homologs of the \Tef barcode were identified in each TNAU isolate assembly through BLASTN (1e-6 cut-off). For each assembly, the locations of hits with greater than 70\% identity and 90\% coverage were recorded, and the sequence within this region was extracted using Samtools (Version 1.15.1). Using the extracted \Tef sequences the Fusariod-ID MSLT database and NCBI BLAST database were searched for similar sequences. For the S16 isolate, the Fusariod MSLT database's best scoring hits were for Fusarium fujikuroi species complex, in which F. sacchari can be found (Table 3). Further, a search of the NCBI database revealed that the \Tef sequence extracted from the S16 assembly's best scoring hits were for F. sacchari. Searches for the S6 isolate extracted \Tef sequences suggest this isolate belongs to the F. oxysporum species complex. Although there were matches for F. oxysporum f. sp. cubense  \Tef sequences, these were not in the top 3 results from searches of both databases for the S6 isolate. This may be due to the quality of the S6 assembly. No matches were found for the S32 extracted \Tef sequences in the Fusarioid-ID MSLT database, and hits against the NCBI GenBank database were for F. fujikuroi isolates.  

The \Tef regions from each TNAU assembly and the in-house \Tef database were also used to build a phylogenetic tree. \Tef sequences were concatenated into a single FASTA file. MAFFT (Katoh \et 2019) was used to construct a multiple sequence alignment with the flag --adjustdirectionaccurately to ensure correct alignment. IQ-TREE (Version 2.2.0.3) was used to infer a maximum-likelihood phylogeny using the ultrafast bootstrap setting for 1000 bootstrap replicates (Nguyen  2015) and was visualised using iTOL (Letunic and Bork, 2021).  

The S16 isolate sits within the same clade as the SY-2 isolate and reference F. sacchari species based on the \Tef phylogeny which, taken together with the BOWTIE2 raw read mapping data, suggests these isolates may be strains of F. sacchari pathogenic towards banana (Figure 9). Based on the \Tef phylogeny, the S32 isolate sits alongside the F. sacchari clade. The S6 isolate clusters within one the F. oxysporum f. sp. cubense clades which, alongside the higher mapping rate for to the F. oxysporum f. sp. cubense reference and Fusariod-DB and NCBI BLASTN results, suggests S6  is a F. oxysporum f. sp. cubense isolate (Figure 9). Interestingly, based on the \Tef phylogeny, S6 clusters within a F. oxysporum f. sp. cubense race 1 clade, but collaborators at TNAU suggest that S6 is highly virulent against Cavendish banana varieties. Further work must be undertaken to improve the quality of the S6 genome assembly and determine which race of F. oxysporum f. sp. cubense the S6 isolate may be, as well as virulence tests comparing S6 to a Race 1 reference with high-quality genome sequence available, such as F. oxysporum f. sp. cubense R1 strain 160527 published by Asai., et al. (2019), to identify changes associated with enhanced virulence.


\subsection{Identification of pathogen-specific regions}
CIRCOS PLOT

\begin{figure}[htp!]
    \centering
    \includegraphics[width=15cm]{Figures/circos.png}
    \caption[Circos Plot of FS66 vs UK0001, including mapping data form TNAU isolates]{\textbf{Circos Plot of FS66 vs UK0001, including mapping data form TNAU isolates}}
    \label{TNAUCircos}
\end{figure}


\subsection{Ortholog analysis?}
\subsection{SIX gene searches – isolates don’t display the same SIX gene profile.}
\subsection{EffectorP outputs, any effectors which have homology to Foc predicted effectors?}


%%%%%%%%%%%%%%%%%%%%%%%%%%%%%%%%%%%%%%%%%%%%%
%DISCUSSION
%%%%%%%%%%%%%%%%%%%%%%%%%%%%%%%%%%%%%%%%%%%%%

\section{Discussion and Conclusions}
\subsection{Strength of genotyping and phenotyping – why does genotyping say it is Foc?}
\subsection{\textit{De novo} Genome Assembly and Contamination Analysis}
Using the raw reads which appear to be contaminated to generate a de novo assembly may result in misassembled contigs which are chimeric (part target species, part non-target species). These contigs can be challenging to identify and may result in contigs which should remain in the assembly being filtered out, and contigs which do not belong to the target species being kept in the assembly, even when using BlobTools to separate out target and non-target contigs. We therefore considered a reference-guided assembly approach, however, as isolates S16 and S32 have a higher mapping rate to the F. saccahri reference and assemblies generated using all raw reads contained contigs which shared greater sequence similarity with other Fusarium species, but these isolates have been classified as F. oxysporum f. sp. cubense isolates by collaborators at TNAU, determining which reference species to use it challenging. Furthermore, these isolates display a highly-virulent phenotype, and a reference-guided assembly may lose any large-scale rearrangements in the genome which may play a role in this. Consequently, for isolates S6 and S32, we have decided to map to two reference genomes (F. oxysporum f. sp. cubense TR4 isolate UK0001; F. sacchari isolate FS66) and then create a de novo assemblies with the mapped reads only.  
\subsection{High level of contamination from Stenotrophomonas.}
\subsection{Novel taxon – Similar results from Philippines paper.}

