\section{Introduction}

Untargeted metabolomics (UM) is emerging as a useful tool in plant pathology and has been employed in the study of infection or plant resistance and susceptibility (Allwood \et 2021). For instance, Galeano Garcia \et (2018) used liquid chromatography–mass spectrometry (LC-MS) to identify biomarkers of \textit{Phythopthora infestans} infection in tomato, including infection in asymptomatic plants. Sambles \et  (2017) and Sidda \et (2020) identified through UM that some secoiridoid glycosides can be used as discriminatory metabolites of ash die-back susceptibility in both UK and Danish ash trees (\textit{Fraxinus excelsior}). 

Most banana metabolomic studies pertain to the fruit and diet, with very few investigating the \textit{Foc}-banana metabolome. Li \et (2013) assessed the virulence of \Foc isolates and used LC/MS/MS to determine the content of two mycotoxins consistently associated with Foc, beauvericin and fusaric acid, in the roots, fruits, pseudostems, and foliage of plants. They showed that virulence correlated well with toxin deposition and went on to investigate the natural occurrence of these toxins in field-grown plants displaying symptoms of \Foc infection. They found that, although present, the natural occurrence was too low to be of concern to human and animal health. 
Li \et (2017b) employed reversed-phase high-performance liquid chromatography (HPLC), alongside biochemical and gene expression analyses, to study early (3, 27, and 51 hours post inoculation (hpi)) infection responses in root tissue of the resistant \textit{Musa yunnanensis} and susceptible Giant Cavendish cultivar, BaXi (\Musa AAA) during Foc TR4 infection. They directly measured specific phytohormones and phenolics which have been well studied in \textit{Arabidopsis} defence response. They show that these metabolites were activated to coordinate the banana defence response.
\subsection{Types of metabolomics analysis}
Metabolomics studies are often categorised based on the type of analysis, targeted or untargeted. The \textit{Foc}-banana studies highlighted are targeted; a set of desired compounds are sought for analysis. This, typically, is applied when quantitive analysis is required and uses complex extraction protocols. The studies by Garcia \et (2018), Sambles \et  (2017) and Sidda \et (2020) adopted an untargeted approach, whereby an undefined set of features are identified. This allows for novel discoveries in a broad range of compounds, including unknown compounds and metabolites. UM uses simple extraction and detection procedures compared to targeted studies but results in highly complex data with an increased false discovery burden requiring significantly more effort in data analysis and interpretation. Our lab has developed state-of-the-art UM profiling methods (see: Sidda \et 2020) which have been applied to gain insight into the metabolomic profiles of banana plants exposed to biotic and abiotic stresses.
So far as we are aware, there are no studies where metabolomics has been employed to capture a global Foc-banana metabolomic profile. It is intended that the output from the UM studies be compared to imaging data, providing insight into the biological processes which influence differences observed when imaging, as well as identifying specific markers of each stress.


\section{Materials and Methods}

\subsection{Sample collection}
Samples from four treatments were collected. Plants were inoculated with \Foc or \Xvm or were exposed to drought stress (no watering from the time of inoculation). A water mock involution treatment group was also established (see ~\ref{sec:Chapter4_MM}). 
Leaf material was collected at one-, two- and three weeks post-inoculation (wpi). Three leaves were collected per plant, with three plants sampled from each treatment at each time point. A section of lamina on either side of the midrib was excised from the base of each leaf, snap-frozen in liquid N2, lyophilised, and ground to a powder. The powdered lamina were then homogenised to generate a single leaf sample for each plant. 

\subsection{Metabolite Extraction and LC-MS Setting}
Aliquots of 10mg from each homogenised sample were extracted on ice in 400\(\mu\)L 60\% LC-MS grade methanol, vortexed for 30 seconds every 10 minutes for 30 minutes, returning to ice in between, sonicated for 15 minutes in ice-water, and centrifuged at 13000rpm for 10 minutes at 4℃. The supernatant was transferred to a clean 2ml Eppendorf. The extraction process was then repeated. After overnight storage at 4℃, samples were filtered through a PVDF syringe filter[\textcolor{red}{WERE THEY?}], and the filtrate was transferred into a glass LC-MS vial for analysis. Quality control (QC) samples were generated by pipetting 10\(\mu\)L from each sample into an LC-MS vial. Analysis was conducted using the Dionex UltiMate 3000 UHPLC system and Agilent Eclipse Plus C18 UPLC column with outflow routed to a Bruker MaXis II Q-TOF with an electrospray source. Samples were run in both positive and negative ion mode. Samples and blanks were randomized.

\subsection{Data Processing and Statistical Analysis}
Only the positive ion mode LC-MS data have been analysed so far. Raw data were converted to the mzXML format using Bruker Compass DataAnalysis 4.4 SR1. The mzXML files were processed in R using the packages IPO 1.10.0 (Libiseller \et 2015), XCMS 3.6.2 (Benton \et 2010) and CAMERA (Kuhl \et 2012) 1.40.0 (See \textcolor{red}{appendix X} for XCMS settings). Samples were initially only grouped by treatment and time, (e.g., 1 wpi control, 1 wpi \Foc, 1 wpi \Xvm, etc.) with each sample group containing three biological replicates and one QC sample. 

Samples were grouped by time point and/or treatment and analyses were performed using MetaboAnalyst 5.0 (https://www.metaboanalyst.ca/), a custom R script or InteractiVenn (http://www.interactivenn.net/). Peak areas were normalised to the Na(NaCOOH)3 adduct (m/z=226.9515) of the sodium formate internal standard. 

\section{Results}
\subsection{Untargeted metabolomics: the banana-pathogen metabolome}

\section{Discussion and conclusion}
