Bananas (Musa spp.) are one of the world’s top agricultural commodities, but global banana production is threatened by \acf{fwb}. Historically caused by \acf{Focub}, \acf{fwb} devastated the global production of ‘Gros Michel’ in the 1900s, only to be saved by the resistant Cavendish sub-group. Cavendish bananas are susceptible to an emerging race of \acf{Focub}, \ac{tr4}, and, since its identification, \ac{Focub4} has spread internationally. Now present in Asia, Africa, Australia, Europe, and South America. 

\textit{Fusarium} isolates pathogenic towards Cavendish banana were collected by collaborators at \acf{tnau} and raw sequence data was shared. Subsequent phylogenetic analysis and virulence gene profiling suggest that two of the isolates shared are from the \acl{FFSC}, and closely related to \acl{Fs}. Further, our data suggest a third isolate is from the newly recorded species, \acl{Fm} sp. nov \parencite{Nozawa2023}. 

We also compiled a database of publicly available \acf{Fo} genomes and recently sequenced genomes of isolates pathogenic towards lettuce, celery, and coriander\footnote{in collaboration with Prof. John Clarkson, University of Warwick and Dr Helen Bates, \acl{niab}}. We developed a computational pipeline for novel virulence gene discovery using a family of commonly associated, short (~180bp) transposable elements, known as \aclp{mimp} \parencite{Schmidt2013}. For the \ac{Fo} genomes included in our analysis, we identified host-specific candidate effector profiles, including novel candidate effectors.

Finally, we the first employed \acl{um} to investigate banana interactions with \ac{Focub} and aid in the development of diagnostics. We identified features of interest that appeared to distinguish three different wilting stresses (\ac{fwb}, \textit{Xanthomonas} wilt of banana, and drought stress). These unique features of interest can be explored as putative biomarkers, and their role in wilt stress response investigated. 

Our interdisciplinary approach has advanced our understanding of \ac{fwb} and its interaction with banana. We have provided insights into pathogen diversity and virulence mechanisms in \ac{fwb}. The findings of this thesis highlight the urgency of ongoing surveillance and diagnostic development to combat \ac{fwb}, as well as the broader applicability of our methodologies in plant disease management. 
